%\begingroup
%\let\cleardoublepage\clearpage


% English abstract
\cleardoublepage
\chapter*{Abstract}
\markboth{Abstract}{Abstract}
\addcontentsline{toc}{chapter}{Abstract (English/Français/Deutsch)} % adds an entry to the table of contents
% put your text here
The development of ultrafast techniques, advancements in laser science and material science have pathed the way for manipulating properties of matter simply by irradiating them with light.
This realization enables not only the study of ultrafast out-of-equilibrium dynamics, but also the selective control of materials.
Particularly correlated matter stands out as a family of materials, that be employed in the next generation of technologies, from zero-loss currents, to ultrafast, efficient switches.
These correlated materials exhibit divers exotic phases, consisting of a complex interplay of various interactions and correlations.
On a fundamental level, studying the ultrafast out-of-equilibrium dynamics offers insights into the destruction and reformation processes of equilibrium phases.
This enables us to gain additional information of interactions and correlations present in complex materials, and disecting the various contributions, to gain a more detailed insight into there emergence.
Additionally, the light irradiation offers a platform for finding and testing metastable states emerging from the out-of-equilibrium state, which are pivotal for the control over the materials properties.\hfill\break

In this dissertation, I explore the existence of metastable states in correlated matter, and probe their electronic signature with time- and angle-resolved photoemission spectroscopy (trARPES).
Beyond this, I will elucidate what the existence of such states can tell us about the equilibrium states. \hfill\break

The thesis start with a description the experimental technique used to study the correlated materials.
First the fundamental principles of ARPES are explained, which will then be extended to the time domain.
Afterwards I will present three different studies performed on the high temperature superconductors \ce{Bi2Sr2CaCu2O8} and \ce{Bi2Sr2Ca2Cu3O10}, as well as the charge density wave (CDW) transition metal dicalchogenide (TMD) \ce{TaTe2}.

In \ce{Bi2Sr2CaCu2O8} a metastable Lifshitz transition is observed after ultrafast light-excitation with infrared (IR) pulses, which can be controlled via the employed pump fluence.
Specifically, a change of the Fermi arc topology is observed, imitating the transformation seen i doping dependent studies.
The light-induced changes are quantified in terms of energy and momentum shifts, and their causes discussed in the framework of photodoping and screening effects.
From there the fluence dependent Fermi surface change is modeled within a tight binding approach.
And finally the extremely long lifetime of more than \qty{100}{\micro\second} is discussed.

Furthermore a study on the tri-layer cuprate \ce{Bi2Sr2Ca2Cu3O10} is presented.
The aim of this study is to address the question of a layer-dependent critical temperature $T_c$, with the tri-layer showing the highest $T_c$ in cuprates.
For this purpose data from the SLS-ULTRA endstation at the PSI synchrotron are analyzed and discussed.
Changes to the Fermi surface in comparison to the bi-layer compound were observed, together with a potential tri-layer splitting.
The observation will help elucidating the increased $T_c$ in the compound.

In addition, the nature of the phase transition, resulting in the formation of CDW were explored in \ce{TaTe2}, as well as the formation of a metastable state at low temperature.
The ultrafast dynamics in the material revealed strong oscillation, which result from the electron-phonon coupling.
Performing Fourier analysis on this data allows for the mapping of the electron-phonon coupling, revealing the band specific coupling of three phonon modes.
Here a potential amplitude mode is identified for the first time.
It is shown that the phase transition is predominantly driven by structural changes.
And the existence of the metastable state is discussed.

Lastly, I will present the development of a new trARPES beamline, which extends the current high harmonic setup to the vacuum ultraviolet regime.
The new beamline offers tunability in the range from \qtyrange{7.2}{10.8}{\electronvolt} with high energy resolution, while sacrificing little in time resolution and \unit{\mega\hertz} repetition rates.
The interoperability with the high harmonic beamline is discussed, as well as further developments on the system. \hfill\break

This dissertation attempts to understand the microscopic properties of correlated materials by investigation their ultrafast out-of-equilibrium dynamics.
In addition, metastable states are explored for their role in controlling the materials.
And the existing metrology is improved to enable new studies in the field.
For each chapter a introduction into the topic is given, the consequences of the studies are discussed and new ideas for future studies are given that could provide further insights in the field.\hfill\break

\textbf{Keywords:} correlated matter, superconductivity, charge density wave (CDW), time- and angle-resolved ARPES, photoexcitation, metastability, control, phase transition, vacuum ultraviolet (VUV)


% German abstract
\begin{otherlanguage}{german}
\cleardoublepage
\chapter*{Kurzzusammenfassung}
\markboth{Kurzzusammenfassung}{Kurzzusammenfassung}
% put your text here
\lipsum[1-2]
\end{otherlanguage}




% French abstract
\begin{otherlanguage}{french}
\cleardoublepage
\chapter*{Résumé}
\markboth{Résumé}{Résumé}
% put your text here
\lipsum[1-2]
\end{otherlanguage}


%\endgroup			
%\vfill
