%\begingroup
%\let\cleardoublepage\clearpage


% English abstract
\cleardoublepage
\chapter*{Abstract}
\markboth{Abstract}{Abstract}
\addcontentsline{toc}{chapter}{Abstract (English/Français/Deutsch)} % adds an entry to the table of contents
% put your text here
Advancements of laser technologies and material science have revolutionized the ability to manipulate matter through irradiation with ultrafast light pulses.
This progress enables not only the study of out-of-equilibrium dynamics but also allows for the selective control of material properties.
On a fundamental level, the investigation of ultrafast dynamics reveals intricate details about the destruction and reformation of equilibrium phases, offering deeper insights into the interactions and correlations governing complex materials.
Dissecting these contributions helps unravel the mechanisms driving emergent exotic phases in condensed matter.

Crucially, out-of-equilibrium phenomena open a route to explore the free energy landscape of materials, often presenting numerous local minima, some of which exhibit long lifetimes.
Light irradiation thus provides a unique tool for accessing and stabilizing these metastable states, which can play a pivotal role in material control.
Correlated matter, due to the close proximity in energy between different phases, presents an ideal system for manipulating the ground state and inducing exotic phases using light pulses.
As a result, they stand out as a class of materials with the potential to drive next-generation technologies, from zero-loss current systems to ultrafast, energy-efficient switches.
Their diverse and exotic phase diagrams, arising from a complex interplay of interactions and correlations, make them particularly promising for these advanced applications.\hfill\break

In this dissertation, I investigate and characterize of metastable states in correlated matter, probing their electronic signatures using time- and angle-resolved photoemission spectroscopy (trARPES).
Beyond demonstrating their presence, I delve into the ultrafast dynamics of their origin, and the insights they provide into the microscopic properties of materials.\hfill\break

The thesis begins by detailing the experimental method, starting with the fundamental principles of ARPES and extending them into the time domain.
Three case studies follow, investigating high-temperature superconductors (SC), \ce{Bi2Sr2CaCu2O8} and \ce{Bi2Sr2Ca2Cu3O10}, as well as the charge density wave (CDW) material \ce{TaTe2}.

In \ce{Bi2Sr2CaCu2O8}, ultrafast IR excitation induces a metastable transformation of the Fermi surface topology, which is controlled by pump fluence, leading to a Lifshitz transition at high fluence.
This light-induced change mimics the doping-dependent evolution of the material.
The changes are quantified by energy and momentum shifts, and discussed within the frameworks of photodoping and screening effects.
The fluence-dependent changes are modeled using a tight-binding approach, and the remarkable lifetime of the metastable state, exceeding \qty{100}{\micro\second}, is explored.

Building on this, I present high-resolution ARPES data for the tri-layer cuprate \ce{Bi2Sr2Ca2Cu3O10}, addressing the layer-dependence of the critical temperature $T_c$ in cuprates.
The observed Fermi surface modifications and possible tri-layer splitting offer insights into the mechanisms driving the enhanced $T_c$, and are discussed in the framework of a composite layer model.

Lastly, the phase transition in \ce{TaTe2}, leading to CDW formation, is studied through its ultrafast dynamics.
Strong oscillations, driven by electron-phonon coupling, are mapped via Fourier analysis, revealing band-specific coupling to three phonon modes, including a previously unreported \qty{0.4}{\tera\hertz} mode, identified as a potential amplitude mode.
The analysis shows that structural changes predominantly drive the phase transition.

In the final chapter, I present the development of a new trARPES beamline extending into the vacuum ultraviolet (VUV) regime, with tunable photon energies from \qtyrange{7.2}{10.8}{\electronvolt}, combining high energy and time resolution (\qty{27}{\milli\electronvolt}, \qty{320}{\femto\second}) paired at \unit{\mega\hertz} repetition rates.
The interoperability with the existing high harmonic setup is addressed, alongside future development prospects.\hfill\break

\textbf{Keywords:} correlated matter, superconductivity, charge density wave (CDW), transition metal dichalcogenide (TMD), time- and angle-resolved ARPES, photoexcitation, metastability, phase transition, vacuum ultraviolet (VUV)


% German abstract
\begin{otherlanguage}{german}
\cleardoublepage
\chapter*{Kurzzusammenfassung}
\markboth{Kurzzusammenfassung}{Kurzzusammenfassung}
% put your text here
Fortschritte in der Lasertechnologie und Materialwissenschaft haben die Fähigkeit, Materie durch ultrakurze Lichtpulse zu manipulieren, revolutioniert.
Diese Entwicklungen ermöglichen nicht nur das Studium von Nicht-Gleichgewichtsdynamiken, sondern auch die selektive Kontrolle von Materialeigenschaften.
Auf fundamentaler Ebene offenbart die Untersuchung ultraschneller Prozesse vielschichtige Details über die Zerstörung und Neuformierung von Gleichgewichtszuständen, wodurch tiefere Einblicke in die Wechselwirkungen und Korrelationen, die komplexe Materialien beeinflussen, gewonnen werden.
Die Analyse dieser Beiträge hilft, die Mechanismen hinter der Entstehung exotischer Phasen in der kondensierten Materie zu entschlüsseln.

Nicht-Gleichgewichtseffekte eröffnen dabei einen Weg zur Erforschung der freien Energielandschaft von Materialien, die oft zahlreiche lokale Minima aufweist, von denen einige lange Lebensdauern zeigen.
Die Bestrahlung mit Licht bietet somit ein einzigartiges Werkzeug, um auf diese metastabilen Zustände zuzugreifen und sie zu stabilisieren, was eine entscheidende Rolle in der Kontrolle von Materialien spielen kann.
Korrellierte Materie, aufgrund der geringen Energiedifferenzen zwischen verschiedenen Phasen, stellt ein ideales System dar, um den Grundzustand zu manipulieren und exotische Phasen mit Lichtpulsen zu induzieren.
Dadurch heben sie sich als eine Klasse von Materialien hervor, die das Potenzial besitzen, die nächste Generation von Technologien anzutreiben, von verlustfreien Stromtransportsystemen bis hin zu ultraschnellen, energieeffizienten Schaltern.
Ihre vielfältigen und exotischen Phasendiagramme, die aus einem komplexen Zusammenspiel von Wechselwirkungen und Korrelationen entstehen, machen sie besonders vielversprechend für diese fortschrittlichen Anwendungen. \hfill\break

In dieser Dissertation untersuche und charakterisiere ich metastabile Zustände in korrelierter Materie, indem ich ihre elektronischen Signaturen mit zeit- und winkelaufgelöster Photoemissionsspektroskopie (trARPES) erfasse.
Neben dem Nachweis ihrer Existenz gehe ich auf die ultraschnelle Dynamik ihrer Entstehung ein und analysiere die Einblicke, die sie über die mikroskopischen Eigenschaften von Materialien bieten. \hfill\break

Die Arbeit beginnt mit einer detaillierten Beschreibung der experimentellen Methode, beginnend mit den grundlegenden Prinzipien von ARPES und deren Erweiterung in den Zeitbereich.
Es folgen drei Studien, die sich mit Hochtemperatur-Supraleitern (SC) \ce{Bi2Sr2CaCu2O8} und \ce{Bi2Sr2Ca2Cu3O10}, sowie dem Ladungsdichtewellen-Material (CDW) \ce{TaTe2} befassen.

In \ce{Bi2Sr2CaCu2O8} induziert die ultrakurze infrarot IR-Anregung eine metastabile Transformation der Fermi-Flächen-Topologie, die durch die Pump-Fluenz gesteuert wird und bei hoher Fluenz zu einem Lifshitz-Übergang führt.
Diese lichtinduzierte Veränderung ahmt die dopingabhängige Entwicklung des Materials nach.
Die Veränderungen werden durch ihre Energie- und Impulsverschiebungen quantifiziert und im Rahmen von Photodoping- und Abschirmeffekten diskutiert.
Die fluenzabhängigen Veränderungen werden mit einem Tight-Binding-Ansatz modelliert, und die bemerkenswerte Lebensdauer des metastabilen Zustands, die über \qty{100}{\micro\second} hinausgeht, wird untersucht.

Darauf aufbauend präsentiere ich hochauflösende ARPES-Daten für das Drei-Lagen Kuprat \ce{Bi2Sr2Ca2Cu3O10}, die die Schichtabhängigkeit der kritischen Temperatur $T_c$ in Kupraten thematisieren.
Die beobachteten Modifikationen der Fermi-Fläche und eine mögliche Drei-Lagen Aufspaltung liefern Einblicke in die Mechanismen, die zu der erhöhten $T_c$ führen, und werden im Rahmen eines Kompositlagen-modells diskutiert.

Abschließend wird der Phasenübergang in \ce{TaTe2}, der zur Bildung der CDW führt, durch seine ultraschnelle Dynamik untersucht.
Starke Oszillationen, die durch die Elektron-Phonon-Kopplung angetrieben werden, werden mittels Fourier-Analyse dargestellt, wobei eine bandspezifische Kopplung an drei Phononenmodi aufgezeigt wird, einschließlich einer zuvor nicht beobachteten \qty{0.4}{\tera\hertz}-Mode, welche als potenzielle Amplitudenmode identifiziert wird.
Die Analyse zeigt, dass strukturelle Veränderungen hauptsächlich den Phasenübergang antreiben.

Im abschließenden Kapitel präsentiere ich die Entwicklung einer neuen trARPES-Strahllinie, die den Gesamtaufbau in den Vakuum-Ultraviolett-(VUV)-Bereich erweitert, und eine verstellbare Photonenergie zwischen \qty{7.2}{\electronvolt} und \qty{10.8}{\electronvolt} aufweist, und eine hohe Energie- und Zeitauflösung (\qty{27}{\milli\electronvolt}, \qty{320}{\femto\second}) mit Wiederholraten im \unit{\mega\hertz}-Bereich kombiniert.
Die Interoperabilität mit dem bestehenden Hochharmonik-System wird ebenso behandelt wie zukünftige Entwicklungsperspektiven.\hfill\break

\textbf{Schlüsselwörter:} korrelierte Materie, Supraleitung, Ladungsdichtewelle (CDW), Übergangsmetalldichalkogenid (TMD), zeit- und winkelaufgelöste ARPES, Photoanregung, Metastabilität, Phasenübergang, Vakuum-Ultraviolett (VUV)

\end{otherlanguage}




% French abstract
\begin{otherlanguage}{french}
\cleardoublepage
\chapter*{Résumé}
\markboth{Résumé}{Résumé}
% put your text here
Les avancées dans les technologies laser et les sciences des matériaux ont révolutionné la capacité à manipuler la matière par irradiation avec des impulsions lumineuses ultrarapides. Ces progrès permettent non seulement l’étude des dynamiques hors équilibre, mais aussi le contrôle sélectif des propriétés des matériaux. À un niveau fondamental, l’investigation des dynamiques ultrarapides révèle des détails complexes sur la destruction et la réformation des phases d’équilibre, offrant ainsi une meilleure compréhension des interactions et des corrélations qui gouvernent les matériaux complexes. L’analyse de ces contributions aide à élucider les mécanismes à l'origine des phases exotiques émergentes dans la matière condensée.

Les phénomènes hors équilibre ouvrent une voie cruciale pour explorer le paysage énergétique libre des matériaux, qui présente souvent de nombreux minima locaux, certains d’entre eux ayant des durées de vie longues. L'irradiation lumineuse fournit ainsi un outil unique pour accéder et stabiliser ces états métastables, qui peuvent jouer un rôle clé dans le contrôle des matériaux. La matière corrélée, en raison de la proximité énergétique entre différentes phases, constitue un système idéal pour manipuler l'état fondamental et induire des phases exotiques par l’utilisation de pulsations lumineuses. En conséquence, ces matériaux se distinguent comme une classe capable de propulser les technologies de prochaine génération, allant des systèmes de transport de courant sans perte à des interrupteurs ultrarapides et efficaces sur le plan énergétique. Leurs diagrammes de phases divers et exotiques, issus d’un entrelacement complexe d’interactions et de corrélations, les rendent particulièrement prometteurs pour ces applications avancées.\hfill\break

Dans cette thèse, j’étudie et caractérise les états métastables dans la matière corrélée, en sondant leurs signatures électroniques à l'aide de la spectroscopie de photoémission résolue en temps et en angle (trARPES). Au-delà de la démonstration de leur présence, j'explore les dynamiques ultrarapides qui les génèrent et les informations qu’elles fournissent sur les propriétés microscopiques des matériaux. \hfill\break

La thèse commence par une description détaillée de la méthode expérimentale, en exposant d'abord les principes fondamentaux de l'ARPES, puis en les étendant au domaine temporel. Trois études de cas suivent, portant sur les supraconducteurs à haute température (SC) \ce{Bi2Sr2CaCu2O8} et \ce{Bi2Sr2Ca2Cu3O10}, ainsi que sur le matériau à onde de densité de charge (CDW) \ce{TaTe2}.

Dans \ce{Bi2Sr2CaCu2O8}, une excitation infrarouge ultrarapide induit une transformation métastable de la topologie de la surface de Fermi, qui est contrôlée par la fluence de la pompe, menant à une transition de Lifshitz à haute fluence. Ce changement induit par la lumière imite l'évolution dépendante du dopage du matériau. Les changements sont quantifiés par des décalages d’énergie et de moment, et sont discutés dans le cadre des effets de photodopage et de criblage. Les changements dépendant de la fluence sont modélisés à l’aide d’une approche tight-binding, et la durée de vie remarquable de l’état métastable, qui dépasse \qty{100}{\micro\second}, est explorée.

Ensuite, je présente des données ARPES à haute résolution pour le cuprate à trois couches \ce{Bi2Sr2Ca2Cu3O10}, en abordant la dépendance à la couche de la température critique $T_c$ dans les cuprates. Les modifications observées de la surface de Fermi et la possible séparation des couches fournissent des informations sur les mécanismes responsables de l’augmentation du $T_c$, et sont discutées dans le cadre d’un modèle de couche composite.

Enfin, la transition de phase dans \ce{TaTe2}, conduisant à la formation de la CDW, est étudiée à travers sa dynamique ultrarapide. De fortes oscillations, provoquées par le couplage électron-phonon, sont cartographiées via une analyse de Fourier, révélant un couplage spécifique aux bandes avec trois modes de phonons, y compris un mode à \qty{0.4}{\tera\hertz}, identifié pour la première fois comme un mode d’amplitude potentiel. L’analyse montre que des changements structurels dominent la transition de phase.

Dans le dernier chapitre, je présente le développement d’une nouvelle ligne de faisceau trARPES, s'étendant dans le domaine de l’ultraviolet sous vide (VUV), avec des énergies de photons ajustables de \qtyrange{7.2}{10.8}{\electronvolt}, combinant une haute résolution en énergie et en temps (\qty{27}{\milli\electronvolt}, \qty{320}{\femto\second}) avec des fréquences de répétition à l’échelle du \unit{\mega\hertz}. L’interopérabilité avec le système existant d’harmoniques élevés est abordée, ainsi que les perspectives de développement futur. \hfill\break

\textbf{Mots-clés:} matière corrélée, supraconductivité, onde de densité de charge (CDW), dichalcogénure de métal de transition (TMD), ARPES résolue en temps et en angle, photoexcitation, métastabilité, transition de phase, ultraviolet sous vide (VUV)
\end{otherlanguage}


%\endgroup			
%\vfill
