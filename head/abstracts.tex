%\begingroup
%\let\cleardoublepage\clearpage


% English abstract
\cleardoublepage
\chapter*{Abstract}
\markboth{Abstract}{Abstract}
\addcontentsline{toc}{chapter}{Abstract (English/Français/Deutsch)} % adds an entry to the table of contents
% put your text here
Advancements of ultrafast techniques, laser technologies and material science have revolutionized the ability to manipulate matter through light irradiation.
This progress enables not only the study of ultrafast, out-of-equilibrium dynamics but also allows for the selective control of material properties.
On a fundamental level, the investigation of these dynamics reveals intricate details about the destruction and reformation of equilibrium phases, offering deeper insights into the interactions and correlations governing complex materials.
Dissecting these contributions helps unravel the mechanisms driving emergence of exotic phases in these systems.

Crucially, out-of-equilibrium phenomena open a window to explore the free energy landscape of materials, often containing numerous local minima, some of which exhibit long lifetimes.
Light irradiation thus provides a unique platform for accessing and stabilizing these metastable states, which can play a pivotal role in material control.
Correlated matter stands out as a class of materials with the potential to drive the next generation of technologies, from zero-loss current systems to ultrafast and energy-efficient switches.
Their diverse and exotic phase diagrams, arising from a complex interplay of interactions and correlations, make them particularly promising for these advanced applications.\hfill\break

In this dissertation, I focus on the existence and characterization of metastable states in correlated matter, probing their electronic signatures using time- and angle-resolved photoemission spectroscopy (trARPES).
Beyond demonstrating their presence, I delve into the ultrafast dynamics that lead to these states, and the insights they provide into the microscopic properties of materials.\hfill\break

The thesis begins by detailing the experimental method, starting with the fundamental principles of ARPES and extending them into the time domain.
Three case studies follow, investigating high-temperature superconductors (SC), \ce{Bi2Sr2CaCu2O8} and \ce{Bi2Sr2Ca2Cu3O10}, as well as the charge density wave (CDW) material \ce{TaTe2}.

In \ce{Bi2Sr2CaCu2O8}, ultrafast IR excitation induces a metastable transformation of the Fermi surface topology, which is controlled by pump fluence, leading to a Lifshitz transition at high fluence.
This light-induced change mimics the doping-dependent evolution of the material,
The changes are quantified by energy and momentum shifts, and discussed within the frameworks of photodoping and screening effects.
The fluence-dependent changes are modeled using a tight-binding approach, and the remarkable lifetime of the metastable state, exceeding \qty{100}{\micro\second}, is explored.

Building on this, I present high-resolution ARPES data for the tri-layer cuprate \ce{Bi2Sr2Ca2Cu3O10}, addressing the layer-dependence of the critical temperature $T_c$ in cuprates, which is highest in the tri-layer compound.
The observed Fermi surface modifications and possible tri-layer splitting offer insights into the mechanisms driving the enhanced $T_c$, and are discussed in the framework of a composite layer model.

Lastly, the phase transition in \ce{TaTe2}, leading to CDW formation, is studied through its ultrafast dynamics.
Strong oscillations, driven by electron-phonon coupling, are mapped via Fourier analysis, revealing band-specific coupling to three phonon modes, including a previously unreported \qty{0.4}{\tera\hertz} mode, identified as a potential amplitude mode.
The analysis shows that structural changes predominantly drive the phase transition, and the existence of a metastable state at low temperature is also discussed.

In the final chapter, I present the development of a new trARPES beamline extending into the vacuum ultraviolet (VUV) regime, with tunable photon energies from \qtyrange{7.2}{10.8}{\electronvolt}, combining high energy resolution paired with few hundred \unit{\femto\second} time resolution at \unit{\mega\hertz} repetition rates.
The interoperability with the existing high harmonic setup is addressed, alongside future development prospects.\hfill\break

\textbf{Keywords:} correlated matter, superconductivity, charge density wave (CDW), transition metal dichalcogenide (TMD), time- and angle-resolved ARPES, photoexcitation, metastability, phase transition, vacuum ultraviolet (VUV)


% German abstract
\begin{otherlanguage}{german}
\cleardoublepage
\chapter*{Kurzzusammenfassung}
\markboth{Kurzzusammenfassung}{Kurzzusammenfassung}
% put your text here
\lipsum[1-2]
\end{otherlanguage}




% French abstract
\begin{otherlanguage}{french}
\cleardoublepage
\chapter*{Résumé}
\markboth{Résumé}{Résumé}
% put your text here
\lipsum[1-2]
\end{otherlanguage}


%\endgroup			
%\vfill
