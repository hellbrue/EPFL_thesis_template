\chapter{Photoinduced Lifshitz transition in \ce{Bi2Sr2CaCu2O8}}
\label{ch:bi2212}

Superconductors have intrigued scientists ever since Kamerlingh Onnes first observed a sudden drop in resistivity to zero in 1911 \cite{van_delft_discovery_2010}, followed by the discovery of perfect diamagnetism by Meissner and Ochsenfeld in 1933 \cite{meissner_neuer_1933}.
These groundbreaking discoveries highlighted the potential of superconductors for technological applications, such as lossless electrical transport and the creation of powerful magnets.
Today, superconductors are utilized in a wide range of fields, from electrical transformers and MRI machines to maglev trains, strong-field magnets, in next-generation nuclear fusion reactors, and quantum computing.

Following the early discoveries of superconducting materials like mercury, lead, niobium, and niobium nitride, the microscopic understanding of superconductivity was revolutionized by the Bardeen-Cooper-Schrieffer (BCS) theory \cite{bardeen_theory_1957}.
This theory explains superconductivity through the formation of Cooper pairs, paired charge carriers that condense into a macroscopic quantum state below a critical temperature.
In conventional superconductors, the pairing mechanism, often referred to as the "glue", is provided by phonons.
With the BCS framework as a foundation, researchers were able to design better superconductors by selecting materials with strong electron-phonon coupling and stiff phonons, achieving higher critical temperatures.
However, these efforts could not push the critical temperature beyond the boiling point of liquid nitrogen.
The highest $T_c$ for a BCS superconductor was achieved in \ce{MgB2} with a critical temperature of \qty{39}{\kelvin} \cite{nagamatsu_superconductivity_2001}.

\begin{figure}
	\centering
	\includegraphics[width=\linewidth]{images/bi2212/timeline_tc}
	\caption{The figure shows a timeline marking the year in which superconductors were discovered, together with their respective critical temperature $T_c$. From \cite{pjray_english_2015}}
	\label{fig:timeline}
\end{figure}

Instead, in 1986 the first high temperature superconductor was found by Berdnoz and Müller \cite{bednorz_possible_1986}.
This remarkable breakthrough reignited the search for room-temperature superconductors and sparked great interest in the unconventional pairing mechanisms behind these materials.
In the decades since, extensive research has led to the discovery of many classes of high-$T_c$ or unconventional superconductors, including iron pnictides, carbon-based systems, heavy-fermion compounds, hydrides, nickelates, and, most famously, the cuprate superconductors, the first materials to exhibit high-$T_c$ behavior.
Figure \ref{fig:timeline} illustrates the timeline of discovery of superconductors and their respective critical temperature.
To date, mercury-based cuprates hold the record for the highest critical temperature at ambient pressure, reaching \qty{133}{\kelvin} \cite{schilling_superconductivity_1993}.

Over the years, several theoretical models have been proposed to explain unconventional superconductors.
One of the earliest was Anderson's resonating valence bond (RVB) theory \cite{anderson_resonating_1973, anderson_resonating--valence-bond_1987}, an extension of Pauling's ideas on metals and molecular interactions \cite{pauling_nature_1938, pauling_nature_1948}.
In this model, singlets are bound by superexchange $J$, a virtual hopping mechanism.
While the RVB model provided valuable insights, such as predicting d-wave symmetry, no direct experimental evidence has confirmed the existence of an RVB-related state.
Another prominent theory focuses on antiferromagnetic fluctuations mediating the formation of Cooper pairs \cite{bickers_cdw_1987}.
This idea, supported by neutron scattering experiments \cite{mook_polarized_1993, hayden_structure_2004, dahm_strength_2009}, suggests that the formation of pairs is mediated by the exchange of particle-hole fluctuations.
Today, it is widely accepted that antiferromagnetic fluctuations and superexchange interactions are critical in the emergence of superconductivity, as demonstrated by recent experimental findings \cite{kowalski_oxygen_2021, omahony_electron_2022}.

Despite these advances, a complete microscopic theory of unconventional superconductivity is lacking.
Additionally, high-$T_c$ superconductors typically exhibit complex phase diagrams with competing and intertwined phases.
While some models succeed in describing specific phases, developing a universal framework to capture the interplay between these phases remains a significant challenge.

In recent years, new approaches have appeared in the quest to push critical temperatures closer to room temperature.
One potential route involves hydrides, which have shown near-room-temperature superconductivity at extreme pressures in the \unit{\giga\pascal} range \cite{duan_structure_2017}.
The goal is to stabilize superconductivity under high pressure, with the possibility of retaining the superconducting phase after the pressure is released.
Although these materials can potentially be described by BCS-Eliashberg theory, this classification remains a topic of ongoing debate.
Another approach involves using light to selectively drive and manipulate materials, potentially inducing superconductivity from an underlying high-temperature normal state.
Recent groundbreaking experiments have provided the first signs of light-induced or photo-enhanced superconductivity \cite{fausti_light-induced_2011, buzzi_phase_2021, fava_magnetic_2024}.

This chapter will provide an overview of high-$T_c$ superconductors, with a focus on cuprates.
I will present the characterization of the samples used in this study, followed by an introduction to the main findings, a light-induced shift of quasiparticle peaks in \ce{Bi}2212, which creates a metastable state lasting on millisecond timescales.
This observation will be expanded to include angle-dependent observation near both the node and anti-node.
I will then discuss the distinction between photodoping and screening effects and reconstruct the Fermi surface for various fluence levels using a tight-binding approach
Lastly I will discuss the time duration of the observed effect and close with a conclusion and outlook for the topic.

\section{General aspects of \ce{Bi2Sr2CaCu2O8}}
\label{sec:bscco_general}

\ce{Bi2Sr2CaCu2O8} (commonly referred to as \ce{Bi}2212) belongs to the family of \ce{Bi}-based cuprates.
Like all cuprates, its electronic properties are predominantly governed by the presence of \ce{CuO2} planes.
Cuprates form in perovskite crystal structures, and they are typically identified by the number of \ce{CuO2} planes in each unit cell.
Figure \ref{fig:bscco_structure} depicts the crystal structures of the single-, bi-, and trilayer \ce{Bi}-based cuprates.

\begin{figure}
	\centering
	\includegraphics[width=0.7\linewidth]{images/bi2212/BSCCO_structure}
	\caption{The figure shows the crystal strucutre of the single-, bi- and tri layer \ce{Bi}-based cuprates \ce{Bi2Sr2Ca_{n-1}Cu_nO_{4+2n}} (\ce{Bi}2201, \ce{Bi2212} and \ce{Bi}2223). Adapted from \cite{nazargulov_english_2009}.}
	\label{fig:bscco_structure}
\end{figure}

The \ce{CuO2} planes consist of \ce{Cu^{2+}} and \ce{O^-} ions, resulting in a $3d^9$ electronic configuration for the transition metal, with one hole per site.
Due to crystal field splitting places this hole is located in the $3d_{x^2-y^2}$ orbital, which is the highest-energy orbital.
The single-hole occupancy can be described using a Mott-Hubbard Hamiltonian at half-filling, representing the basic electronic structure of the system.
Due to the large on-site Coulomb repulsion $U$, which is approximately \qty{8}{\electronvolt}, the $d$-band splits into an upper and lower Hubbard band (UHB and LHB).
Under typical conditions, this would make \ce{Bi2212} a Mott-Hubbard insulator.
However, in the present case the charge transfer gap between the \ce{O} $2p$ and the \ce{Cu} $3d$ orbital is smaller, hence the classification as a Charge-Transfer insulator \cite{zaanen_band_1985,zegrodnik_superconductivity_2019}.
Figure \ref{fig:cuo_dos} shows a schematic of the \ce{CuO2} plaquette and the corresponding density of states (DOS) for doped and undoped cases.

\begin{figure}
	\centering
	\includegraphics[width=0.7\linewidth]{images/bi2212/fs_bi2212}
	\caption{The figure shows the Fermi surface of \ce{Bi}2212. The center of the BZ is marked with $\Gamma$, and a corner of the BZ is marked as ($\pi$, $\pi$), representative for all four corners. Additionally the nodal (gapless) region and anti-nodal (gapped) region are marked.}
	\label{fig:fsbi2212}
\end{figure}

In cuprates, the valence states, composed primarily of \ce{Cu} $d$-bands and \ce{O} $p$-bands, are largely responsible for their electronic properties.
Additionally, the physics of the cuprates is highly dependent on the amount of charges, which can be adjusted by adding holes or electrons to the system in a chemical doping process.
In this chapter, I will focus on the hole-doped case of \ce{Bi}2212, as all the measurements discussed were performed on hole-doped samples.
Introducing a small number of holes leads to their delocalization and the formation of the Zhang-Rice singlet (ZRS) \cite{zhang_effective_1988}, which resides at the Fermi level (see Fig. \ref{fig:cuo_dos}).
Angle-resolved photoemission spectroscopy (ARPES) reveals four Fermi arcs, which in the normal state leads to an enclosed hole-like Fermi pocket which is centered around the ($\pi$,$\pi$) points.
The shape of these arcs changes with varying hole doping, and at a doping level of approximately 0.22 holes, the Fermi surface transforms, becoming centered around the $\Gamma$ point, resulting in an electron-like pocket.

This change in Fermi surface topology is known as a Lifshitz transition, which signifies a sudden alteration in the electronic structure \cite{lifshitz_anomalies_1960}.
Lifshitz transitions occur in various correlated materials and are often associated with emergent quantum phenomena.
In both cuprate and pnictide superconductors, Lifshitz transitions have been observed as a function of doping.
In \ce{Bi}2212, this transition is linked to the presence of a flat band or extended saddle point near ($\pi$,$0$), which shifts towards the Fermi level with increasing doping and crosses it at around 0.22 holes \cite{campuzano_photoemission_2004, gofron_observation_1994}.
The Lifshitz transition in cuprates is often accompanied by the disappearance of the pseudogap state \cite{matt_electron_2015, benhabib_collapse_2015} and the onset of the strange metal phase, suggesting the proximity of a quantum critical point \cite{michon_thermodynamic_2019, chen_incoherent_2019, cooper_anomalous_2009, badoux_change_2016, putzke_reduced_2021}.

Two distinctive areas can be identified in the Fermi surface.
The region around the ($\pi$,$0$) points is called the anti-node and corresponds to the region in which a superconducting gap opens below the superconducting transition temperature $T_c$.
The second region corresponds instead to the area at the center of the arcs (approximately at ($\frac{\pi}{2}$,$\frac{\pi}{2}$)) and is called the nodal-area.
As its name suggests, no gap opens up in this region of the Fermi surface, even when entering the superconducting phase.

\begin{figure}[t]
	\centering
	\begin{subfigure}[b]{0.49\textwidth}
		\includegraphics[width=\textwidth]{bi2212/cuo}
		\caption{}
	\end{subfigure}
	\begin{subfigure}[b]{0.45\textwidth}
		\includegraphics[width=\textwidth]{bi2212/dos}
		\caption{}
	\end{subfigure}
	\caption{(a) shows a sketch of the \ce{CuO}-plane, consisting of the \ce{Cu} $3d_{x^2-y^2}$ and \ce{O} $2p_{x,y}$ orbitals. (b) shows a sketch of the density of states close to $E_F$ for the doped (top) and undoped (bottom) case.}
	\label{fig:cuo_dos}
\end{figure}

The electronic properties of cuprates are heavily influenced by the hole-doping level, leading to a rich phase diagram with multiple exotic phases. Figure \ref{fig:phase_diag} (a) presents a schematic of the hole-doped cuprate phase diagram.
In the undoped state, the material behaves as a Mott insulator due to strong electron correlations, characterized by long-range antiferromagnetic order below a Néel temperature of around \qtyrange{300}{400}{\kelvin}.
Increasing hole doping leads to the emergence of superconductivity between 0.05 and 0.25 holes per unit cell.
The superconducting phase exhibits a dome-like shape in the phase diagram, peaking at an optimal doping, where \ce{Bi2212} achieves a maximum critical temperature $T_c$ of approximately \qty{90}{\kelvin} at a doping level of 0.16.
Doping levels below and above this point are referred to as underdoped (UD) and overdoped (OD), respectively.

The superconducting (SC) phase in cuprates features a momentum-dependent $d$-wave gap symmetry, which has been observed using various experimental techniques, including SQUID, ARPES, STM, and neutron scattering \cite{wollman_experimental_1993, tsuei_pairing_1994, tsuei_pairing_1996, shen_anomalously_1993, ding_angle-resolved_1996, chen_unconventional_2022, renner_vacuum_1995, pan_microscopic_2001, gu_directly_2019, devereaux_electronic_1994, sacuto_nodes_1997, fong_neutron_1999}.
The amplitude of this gap is described by the equation
\begin{equation}
	\Delta(k_x,k_y) = \frac{\Delta_0}{2}\left[\cos(k_xa)-\cos(ky_a)\right]
\end{equation}
where $\Delta_0$ is the maximum gap value, and $a = \qty{3.83}{\angstrom}$ is the lattice constant for the \ce{Cu}-\ce{Cu} distance.
The result of this is the Fermi surface consisting of the four Fermi arcs, being divided into the nodal and anti-nodal regions.
This $d$-wave symmetry contrasts with the $s$-wave symmetry found in conventional BCS superconductors \cite{bardeen_theory_1957}, which exhibit a full gap across the Brillouin zone.

At temperatures above $T_c$, in the normal state, other phases emerge.
In the underdoped regime, the pseudogap (PG) phase appears, characterized by a partial gap with a $d$-wave symmetry similar to the superconducting gap, though no superconductivity occurs.
The maximum doping, as a function of temperature, for which the pseudogap phase still exists is defined as $p^*$, but variations for $p^*$ were recorded in the same material using different techniques.
A complete understanding of the pseudogap phase remains elusive, but evidence points to its connection with short-range antiferromagnetic correlations \cite{zhang_effective_1988, rice_theory_1973}.
The interplay and coexistence of the PG and SC phase is still a puzzling observation and the mechanism behind one of the big questions in cuprate superconductors.

Further increasing the doping level beyond $p^*$ leads to the formation of the "strange metal" phase.
Also the microscopic mechanisms responsible for the strange metal phase are largely unknown.
In this phase, the resistivity shows a linear temperature dependence, extending to temperatures as high as a few hundred kelvin in cuprates \cite{martin_normal-state_1990, daou_linear_2009, cooper_anomalous_2009}.
The strange metal phase is not unique to hole-doped cuprates and is observed across various material classes.
It typically emerges near a quantum critical point \cite{cooper_anomalous_2009,varma_singular_2002,marel_quantum_2003} and is closely linked to the phenomenon of Planckian dissipation, where the inelastic scattering time $\tau$ is inversely proportional to temperature
\begin{equation}
	\tau \approx \frac{\hbar}{k_BT}
\end{equation}
where $\hbar$ is the Planck constant and $k_B$ the Boltzmann constant \cite{ataei_electrons_2022,bruin_similarity_2013}.
A further increase in doping, into the heavily overdoped regime, leads to the appearance of a Fermi-liquid phase \cite{barisic_evidence_2019}, in which the linear temperature dependence of the strange metal changes to the quadratic dependence of the resistivity $\rho\sim T^2$ \cite{nakamae_electronic_2003,hussey_coherent_2003,plate_fermi_2005}.

\begin{figure}[t]
	\centering
	\begin{subfigure}[b]{0.49\textwidth}
		\includegraphics[width=\textwidth]{bi2212/phase_diag}
		\caption{}
	\end{subfigure}
	\begin{subfigure}[b]{0.45\textwidth}
		\includegraphics[width=\textwidth]{bi2212/fs_dwave}
		\caption{}
	\end{subfigure}
	\caption{(a) shows a sketch of the phase diagram of the hole doped cuprates. Adapted from \cite{keimer_quantum_2015}. (b) shows a sketch of the Fermi surface of \ce{Bi}2212. The red line symbolizes the momentum dependent gap magnitude, corresponding to the d-wave symmetry. Adapted from \cite{zhang_photoinduced_2017}.}
	\label{fig:phase_diag}
\end{figure}

Another notable phase in \ce{Bi}2212 and other cuprates is the charge density wave (CDW) phase.
The physics of charge density waves (CDWs) is explored in greater detail in the chapter focused on the CDW compound \ce{TaTe2}, found in Section \ref{sec:cdw},
This phase overlaps with the pseudogap and superconducting phases and is a topic of active research, especially regarding its competition with superconductivity \cite{arpaia_charge_2021}.
Recent ultrafast studies have revealed differing recovery times for CDW and superconducting features, showing a potential way of studying mutual effects between the two phases \cite{wandel_enhanced_2022}.

In this chapter, I will explore how femtosecond irradiation can manipulate the material's microscopic properties.
One key question is whether light-induced changes can transiently drive the system into new Fermi surfaces, imitating entirely different phases, and whether this approach can be used to transiently access different exotic phases.
This study could serve as a platform to determine whether exotic phases can still be transiently observed or, if not, to identify what additional factors might be necessary.

For this it is important to have a look at the fundamental physics that describe the \ce{CuO2} planes.
The three-band Hubbard model (sometimes called Emery model) can describe the contributions from multiple orbitals as well as their interaction\cite{avella_emery_2013,emery_theory_1987,ogata_tj_2008}.
The tight-binding Hamiltonian of the system can be written as
\begin{equation}
\begin{split}
	H &\quad = \epsilon_d \sum_{i}^{} n_i^d + \epsilon_p \sum_{i}^{} n_j^p \\
	  &\quad + \sum_{\braket{i,j}}^{} t_{pd}^{ij} \left(d_i^\dagger p_j + h.c.\right) + \sum_{\braket{j,j'}}^{} t_{pp}^{jj'} \left(p_i^\dagger p_j + h.c.\right)\\
	  &\quad + U_d \sum_{i}^{} n_{i,\uparrow}^d n_{i,\downarrow}^d + U_p \sum_{j}^{} n_{j,\uparrow}^p n_{j,\downarrow}^p + V_{pd} \sum_{\braket{i,j}}^{} n_i^d n_j^p
\end{split}
\end{equation}
where $\epsilon_d$ and $\epsilon_p$ are the respective energies of copper and oxygen orbitals, $n_d$ and $n_p$ are the corresponding occupation numbers, and $t_{pd}$ and $t_{pp}$ are the hopping parameters for copper-oxygen and oxygen-oxygen orbitals.
The last three term describe the on-site Coulomb interaction of the copper and oxygen orbitals with their respective potential $U_d$, $U_p$, as well as the contribution from copper-oxygen Coulomb interaction $V_{pd}$.

This model can be simplified to a single-band Hamiltonian that describes interactions between copper sites via superexchange, mediated by oxygen orbitals \cite{anderson_new_1959, zhang_effective_1988, emery_mechanism_1988}.
The superexchange $J$ is given by
\begin{equation}
	J= \frac{4t_{pd}^4}{\left(\Delta + V_{pd} \right)^2} \left[\frac{1}{U_d} + \frac{2}{2\Delta + U_p}\right]
\end{equation}
where $\Delta$ represents the charge transfer gap.
This leads to the Heisenberg Hamiltonian
\begin{equation}
	H_J = J \sum_{\braket{i,j}}^{} \mathbf{S}_i \mathbf{S}_j
\end{equation}
mapping the system onto a square lattice of localized spins, with exchange restricted to the nearest neighbors.
For cuprates, $J$ is typically around \qty{0.1}{\electronvolt}, which results in the antiferromagnetic order that was already mentioned earlier.

The models described above apply to the undoped insulating state.
Upon hole doping, the added hole delocalizes and forms a ZRS.
This delocalization is driven by the strong Coulomb repulsion on the copper sites, which prevents double occupancy.
Zhang and Rice showed that an additional hole leads to the formation of a singlet state that is energetically favored \cite{zhang_effective_1988}.
The singlet is formed from a symmetric superposition of four oxygen $2p_{x,y}$ orbitals hybridized with the \ce{Cu} $3d_{x^2-y^2}$ orbital.
One can think of the ZRS as a hole moving through the antiferromagnetic square lattice, described by the Heisenberg-Hamiltonian, and whose movement is expressed by the superexchange $J$.
The low-energy nature of the singlet state allows to reduce the previous three band model, to an effective one band Hamiltonian, in which the hopping of the ZRS is expressed by an effective hopping parameter $t_{i,j}$.
This model is commonly referred to as the t-J-model
\begin{equation}
	H_\text{eff} = - \sum_{\braket{i,j}\sigma} P \left( t_{ij} c_{i\sigma}^\dagger c_{j\sigma} + h.c. \right) P + J \sum_{\braket{i,j}} \left[\mathbf{S}_i \mathbf{S}_j - \frac{1}{2} n_i n_j\right]
\end{equation}
where P is the Gutzwiller projector, which excludes any double occupancy on a site $i$.
Apart from the nearest neighbor hopping $t$, higher orders can be included as well, with the next nearest neighbor hopping $t'$ etc.

The t-J model emphasizes the significance of hopping parameters in determining the electronic properties of \ce{CuO2} planes.
ARPES measurements allow for the estimation of these parameters by fitting the Fermi surface \cite{norman_phenomenological_1995}.
In this chapter, I will discuss how intense femtosecond infrared pulses can modify the Fermi surface and alter these microscopic parameters.

\section{Sample characterization}

In this study, we assessed \ce{Bi}2212 single crystals grown at the EPFL crystal growth facility in collaboration with László Forró's research group.
The sample preparation involved synthesizing crystals from a powder mixture containing \ce{Bi2O3}, \ce{SrCO3}, \ce{CaCO3}, and \ce{CuO}.
After thorough manual mixing, the powder charge, typically weighing between \qtyrange{200}{300}{\gram}, was placed in a high-purity alumina crucible. The crucible was then inserted into a furnace and heated to \qty{1020}{\degreeCelsius} over approximately 4 hours.

Once the powders fully melted, which took around 6 hours, the temperature was held steady until a completely homogeneous and slightly viscous fluid was confirmed via visual inspection.
To ensure further homogeneity, the mixture was stirred using an alumina rod.
The cooling process commenced by reducing the temperature from \qty{1020}{\degreeCelsius} to \qty{840}{\degreeCelsius} at a controlled rate of \qty{1.6}{\degreeCelsius/\hour}, taking just over four days.
Following this, the crucible was rapidly cooled (about \qty{10}{\degreeCelsius/\hour}) from \qty{840}{\degreeCelsius} to room temperature. Once cooled, the crucible was opened, revealing mica-like flakes with glossy crystal faces. Using this method, crystals as large as \qty{1}{\centi\meter} were successfully grown.

\begin{figure}
	\centering
	\includegraphics[width=\textwidth]{images/bi2212/sample_res.pdf}
	\caption{Electrical resistance of the sample Bi2212 used in the experiment. $T_{c,\text{onset}}=\qty{92.3}{\kelvin}, T_{c,\text{50\%}}=\qty{90.1}{\kelvin}, T_{c,\text{zero}}=\qty{85.7}{\kelvin}$}
	\label{fig:sample_res}
\end{figure}

The crystal growth process was conducted in air.
It is important to note that annealing or heat treatment of \ce{Bi}2212 samples significantly affects their $T_c$.
The initial superconducting transition was observed at approximately \qty{75}{\kelvin} (for zero resistance).
However, after subjecting the cleaved crystals to air heating at temperatures up to \qty{600}{\degreeCelsius} for as little as one minute, $T_c$ increased to typical values, ranging from \qtyrange{85}{90}{\kelvin}, as shown in Fig. \ref{fig:sample_res}.

Sample characterization involved electrical resistivity measurements on thin, well-shaped single crystals. Gold wires, with diameters of \qty{12}{\micro\meter}, were used for electrical contacts, which were secured with silver epoxy and cured at \qty{600}{\degreeCelsius}. The contact resistances were typically below \qty{1}{\ohm}. The resistivity in the a-b plane was measured using line contacts in a standard four-probe arrangement \cite{kendziora_composition_1992}.

\section{Photoinduced change of Fermi momenta k$_F$}
\label{sec:larger_effect}

In this section, I will present the fundamental changes observed in the band structure of \ce{Bi}2212 after pump excitation.
The discussion is based on the electronic structure of optimally doped \ce{Bi}2212 at equilibrium (see Sec. \ref{sec:bscco_general}).
All measurements discussed in this chapter were conducted at a temperature of approximately \qty{70}{\kelvin}, well within the superconducting phase.
The Harmonium beamline \cite{arrell_harmonium_2017} was used to probe the band structure with a photon energy of \qty{28}{\electronvolt}, and the excitation was induced by an infrared pump pulse at \qty{1.6}{\electronvolt}, resonant with the charge-transfer gap.

The focus of this section is on how the quasiparticle peaks and a parabolic band-like feature behave under intense light excitation.
For this analysis, the sample was oriented so that the anti-nodal region was parallel to the detector slit, yielding a Fermi surface (FS) as shown in Fig. \ref{fig:fs_cut} (a).
A cut was taken between the node and anti-node, marked by a dashed black line in Fig. \ref{fig:fs_cut} (a), and the corresponding bandmap is shown in Fig. \ref{fig:fs_cut} (b).
This bandmap includes the key features discussed in Sec. \ref{sec:bscco_general}, such as the quasiparticle peaks near $E_F$, the parabolic low-energy feature, the higher-energy \ce{Cu} d-bands, and the incoherent "waterfall" feature connecting them.
It is important to note that the superconducting gap is not observable in these measurements, mainly due to the limited energy resolution of the Harmonium beamline (around \qty{150}{\milli\electronvolt}) and the residual spectral weight from the pseudogap phase.
As a result, the quasiparticle peak is seen at the Fermi level, along with its corresponding Fermi momentum $k_F$.

\begin{figure}[t]
	\centering
	\begin{subfigure}[b]{0.49\textwidth}
		\includegraphics[width=\textwidth]{bi2212/fermi_surface_hhg}
		\caption{}
	\end{subfigure}
	\begin{subfigure}[b]{0.45\textwidth}
		\includegraphics[width=\textwidth]{bi2212/fs_cut}
		\caption{}
	\end{subfigure}
	\caption{(a) Fermi surface of \ce{Bi}2212 (show real FS). The nodal and anti-nodal regions are marked. The black dashed line indicates a cut for which the bandmap in (b) is taken.}
	\label{fig:fs_cut}
\end{figure}

Accurate determination of the $k_F$ points is essential for the discussion in this chapter.
Therefore, careful alignment of the Fermi surface with the analyzer slit was achieved by manually rotating the sample using a wobble stick, with a typical error margin of $\pm$\qty{5}{\degree}.
The Fermi arcs were fitted using Lorentzian peaks to correct for any angular misalignment or tilt due to sample termination not being perfectly parallel to the manipulator surface.

\begin{figure}[th!]
	\centering
	\includegraphics[width=1\textwidth]{images/bi2212/EDM_collection}
	\caption{The figure shows a collection of bandmaps all taken at the same parallel momentum, which is marked by the black dashed line in Fig. \ref{fig:fs_cut} (a). Each bandmap represents a scan with a different pump fluence, from the pump being blocked to a pump fluence of \qty{3.33}{\milli\joule/\centi\meter\squared}. The evolution shows a closing of the parabola which also shows in the quasiparticle peaks moving closer to \qty{0}{\per\angstrom}. Additionally the leading edge of the \ce{Cu} d-bands (at \qty{1.2}{\electronvolt}) shifts slightly upwards with increasing fluence. At a fluences above \qty{1.74}{\milli\joule/\centi\meter\squared} a slight upward shift of the Fermi level is visible.}
	\label{fig:edm_collection}
\end{figure}

A momentum cut at \qty{-0.423}{\angstrom^{-1}} was selected for studying the out-of-equilibrium response in the high pump-fluence regime.
To this end, the pump fluence was varied from \qtyrange{0}{3.33}{\milli\joule/\centi\meter\squared}, and changes in the band structure were recorded.
A pump-probe delay of \qty{166}{\micro\second} was used, an unusually long timescale that will be discussed further in section \ref{sec:meta}.
The resulting bandmaps are shown in Fig. \ref{fig:edm_collection}.
Clear changes to the parabolic feature, as well as to the $k_F$ points and the \ce{Cu} d-bands, can be observed.
The parabola narrows continuously, and the changes in Fermi momenta become especially noticeable when the MDC is plotted as a function of fluence (see Fig. \ref{fig:fluence_map}).
This colormap shows that the $k_F$ points for positive and negative $k_\parallel$ shift toward \qty{0}{\per\angstrom}, with the change in positive momentum being more pronounced due to the planar offset $\chi$.

\begin{figure}[t]
	\centering
	\includegraphics[width=1\textwidth]{images/bi2212/MDC_fits}
	\caption{The figure shows a collection of the MDCs corresponding to the bandsmaps in Fig. \ref{fig:edm_collection}. The MDCs is a result of the integration at $E_F$ in a $\pm$\qty{100}{\milli\electronvolt} window. Triple or Double Lorentzians were used to fit the peak position. The two outer Lorentzians correspond to the position of the quasiparticle peak, and the third corresponds to the Umklappband crossing the spectrum (see Fig. \ref{fig:fs_cut} (a)). The MDCs are fitted for a fluence between \qtyrange{0}{3.33}{\milli\joule/\centi\meter\squared} with the individual plots labeled as such. The distance of the quasiparticle peaks $\Delta k_\parallel$ is given in the respective MDCs.}
	\label{fig:mdc_fits}
\end{figure}

In addition to visualizing the changes in the bandmaps, it is important to quantify the pump-induced shifts in the parabola by extracting the $k_F$ positions.
MDCs were plotted over a $\pm$\qty{100}{\milli\electronvolt} range around $E_F$ (see Fig. \ref{fig:mdc_fits}).
Two peaks, corresponding to the positive and negative $k_F$ points, are visible in each MDC.
A triple Lorentzian fit was applied, with two peaks representing the $k_F$ points and a third accounting for the umklapp band crossing the anti-nodal region (see Fig. \ref{fig:fs_cut} (a)).
This band arises from a lattice superstructure caused by a slight distortion of the unit cell along one high-symmetry direction, which also generates replicas in momentum space.

Repeating this fitting process for each fluence level revealed a linear relationship between the distance of the $k_F$ points and the pump fluence, as shown in Fig. \ref{fig:fluence_map} (b).
The linear trend is evident not only in the peak distance but also in the individual peak positions (see Fig. \ref{fig:fluence_map} (b) and (c)). Figure \ref{fig:fluence_map} (c) also highlights the asymmetric change in the positions of the positive and negative peaks, as previously discussed.

\begin{figure}[t]
	\centering
	\begin{subfigure}[b]{0.25\textwidth}
		\includegraphics[width=\textwidth]{bi2212/fluence_map}
		\caption{}
	\end{subfigure}
	\begin{subfigure}[b]{0.35\textwidth}
		\includegraphics[width=\textwidth]{bi2212/Peak_distance_fluence}
		\caption{}
	\end{subfigure}
	\begin{subfigure}[b]{0.35\textwidth}
		\includegraphics[width=\textwidth]{bi2212/Peak_position_fluence}
		\caption{}
	\end{subfigure}
	\caption{(a) The figure shows the intensity distribution at $E_F$ for a fluence range from \qtyrange{0}{1.74}{\milli\joule/\centi\meter\squared}. The figure visualizes the closing of the parabola seen in Fig. \ref{fig:edm_collection}. (b) The graph shows a the distance between the two quasiparticle peak as determined by the MDC fitts from Fig. \ref{fig:mdc_fits}. The error is determined from the standard deviation of the fitted central momentum. (c) The graph shows the peak position individually for positive and negative momenta. A linear but asymmetric change for each peak can be observed, which is also visible in the fluence map of (a).}
	\label{fig:fluence_map}
\end{figure}

Similar shifts in $k_F$ have been observed previously in bandmaps taken at the nodal position \cite{rameau_photoinduced_2014}.
In those measurements, the Fermi momentum shift was significantly smaller for similar pump fluences, indicating a possible momentum dependence of the effect.
This dependence is also seen when varying the hole-doping content, with stronger changes in the anti-nodal region due to alterations in the anti-bonding band, while the nodal region remains relatively stable \cite{drozdov_phase_2018}.
This comparison motivates an extension of the studies to a larger momentum space.

\section{Nodal vs anti-nodal behavior}
\label{sec:angle}

\ce{Bi}2212 is well-known for its anisotropy between the nodal and anti-nodal regions, which is clearly evident from its d-wave superconductivity, which results in a superconducting gap at the anti-node but not at the node.
Additionally, the chemical doping-dependent evolution of the Fermi surface highlights this anisotropy, with significant changes observed near the anti-node while the nodal region remains relatively unchanged.
In this section, I explore whether a similar anisotropic behavior is evident in the light-induced effects presented in the previous section (Sec. \ref{sec:larger_effect}).

\begin{figure}[t]
	\centering
	\includegraphics[width=0.5\textwidth]{images/bi2212/fermi_map_ang_dep}
	\caption{The figure shows the FS of Bi2212 as in Fig. \ref{fig:fs_cut}. The black dashed lines indicate the series of cuts for which a fluence dependence was recorded.}
	\label{fig:fermimap_angdep}
\end{figure}

In this context, a series of cuts between the nodal and anti-nodal regions were analyzed.
These cuts were taken at parallel momentum $k_p$ values of \qtylist{0.41;0.452;0.494;0.535}{\per\angstrom}.
Figure  \ref{fig:fermimap_angdep} shows the same Fermi surface as in Fig. \ref{fig:fs_cut} (a), with black dashed lines marking the $k_p$ positions of the selected cuts.
For each cut, the fluence dependence was measured similarly to section \ref{sec:larger_effect}, with fluences ranging from \qtyrange{0.09}{1.15}{\milli\joule/\centi\meter\squared}.
The resulting bandmaps from this fluence and momentum-dependent study are shown in figure \ref{fig:effect_angle}.

\begin{figure}[b!]
	\centering
	\includegraphics[width=0.6\linewidth]{images/bi2212/ang_dep_deltak}
	\caption{The figure shows the distance of the quasiparticle peaks at the Fermi level $E_F$ as a function of fluence, for each of the four selected cuts. In each fluence series a linear dependence of the peak distance $\Delta k_\parallel$ to the pump fluence can be observed. The magnitude of the effect increases when approaching the anti-node.}
	\label{fig:angdep_deltak}
\end{figure}

\begin{figure}[t!]
	\centering
	\begin{subfigure}[b]{0.95\textwidth}
		\includegraphics[width=\textwidth]{bi2212/Dispersion_8deg}
		\caption{}
	\end{subfigure}
	\\
	\begin{subfigure}[b]{0.95\textwidth}
		\includegraphics[width=\textwidth]{bi2212/Dispersion_9deg}
		\caption{}
	\end{subfigure}
	\\
	\begin{subfigure}[b]{0.95\textwidth}
		\includegraphics[width=\textwidth]{bi2212/Dispersion_10deg}
		\caption{}
	\end{subfigure}
	\\
	\begin{subfigure}[b]{0.95\textwidth}
		\includegraphics[width=\textwidth]{bi2212/Dispersion_11deg}
		\caption{}
	\end{subfigure}
	\caption{The figure shows a series of bandmaps for four different cuts corresponding to (a) \qty{0.41}{\per\angstrom} (b) \qty{0.452}{\per\angstrom} (c) \qty{0.494}{\per\angstrom} (d) \qty{0.535}{\per\angstrom}, as indicated in Fig. \ref{fig:fermimap_angdep}. For each cut a series of fluences between \qtyrange{0.09}{1.15}{\milli\joule/\centi\meter\squared} is shown.}
	\label{fig:effect_angle}
\end{figure}

As with the previous measurements, a closing of the parabolic feature can be observed at higher momentum cuts, whereas little to no change is visible at \qty{0.41}{\per\angstrom}.
Additionally, a slight shift of the leading edge is evident across all four cuts.
The same analysis as in the previous section was performed for each momentum cut, using MDCs taken at $E_F$ over a $\pm$\qty{100}{\milli\electronvolt} range.
Triple Lorentzians were fitted to the resulting MDCs to extract the peak positions, quantifying the shift of the quasiparticle peak and the closing of the parabola.
As fluence increases, the distance between the peaks decreases for each cut, with the degree of this decrease varying substantially depending on the proximity to the anti-node.

At a parallel momentum of \qty{0.535}{\per\angstrom}, the change is most pronounced, showing a $20\%$ reduction in the peak distance.
In contrast, the reductions are smaller for the other cuts, with $5\%$, $3\%$, and $2.7\%$, respectively, as the fluence increases from \qty{0.09}{\milli\joule/\centi\meter\squared} to \qty{1.15}{\milli\joule/\centi\meter\squared}.
The peak distances for each momentum cut are plotted as a function of fluence in Fig. \ref{fig:angdep_deltak}, revealing the increasing shift in quasiparticle peaks as the cuts approach the anti-nodal region.
This plot also demonstrates that the closing of the distance between the two peaks remains linear with respect to pump fluence, consistent with the behavior observed in the previous section.

From the MDC fits, it is possible to locate the momentum positions of each peak on the Fermi surface.
To validate the results, the data points corresponding to a fluence of \qty{0.09}{\milli\joule/\centi\meter\squared} were compared to previously published literature on the Fermi surface of optimally doped \ce{Bi}2212.
This comparison shows strong agreement between the extracted data points and the literature values.
The data points for each cut at various fluences were plotted on a constant energy map, alongside literature data.
Instead of displaying the entire Fermi surface, the zoomed-in regions of the anti-node (Fig. \ref{fig:FS_points_zoom} (a)) and node (Fig. \ref{fig:FS_points_zoom} (b)) are shown for better visualization of the quasiparticle movement induced by pump excitation.
When compared to previously published Fermi surface data for \ce{Bi}2212 samples with varying hole doping, a similarity can be seen between the behavior observed at higher pump fluences and the Fermi surface changes associated with increased hole doping via chemical methods \cite{drozdov_phase_2018}.

\begin{figure}[b!]
	\centering
	\begin{subfigure}[b]{\textwidth}
		\includegraphics[width=\textwidth]{bi2212/doping_dep_fs}
		\caption{}
	\end{subfigure}
	\\
	\begin{subfigure}[b]{0.49\textwidth}
		\includegraphics[width=\textwidth]{bi2212/FS_points_antinode}
		\caption{}
	\end{subfigure}
	\hfill
	\begin{subfigure}[b]{0.49\textwidth}
		\includegraphics[width=\textwidth]{bi2212/FS_points_node}
		\caption{}
	\end{subfigure}
	\caption{(a) The figure shows three different Fermi surfaces for different amount of hole doping. The evolution highlight the strong change of the bands close to the anti-node (anti-bonding band, dashed line), whereas the node remains unchanged. Adapted from \cite{drozdov_phase_2018}. (b) and (c) show the region of the anti-node and node respectively, together with the extracted Fermi momenta, for all five fluences, together with literature data for comparison.}
	\label{fig:FS_points_zoom}
\end{figure}

The evolution of the quasiparticle peaks at different momenta represents a key finding of this chapter and will form the basis for further analysis.
It is important to note that additional scans were performed at higher momenta, closer to the anti-node, to increase the number of data points.
However, it was not possible to obtain reliable MDCs for these scans.
This difficulty arose due to the broad distribution of spectral intensity associated with the incoherent quasiparticle peaks that form the Fermi arcs, due to the presence of the pseudo gap \cite{norman_destruction_1998}.
This broadening, combined with the relatively low energy resolution of the Harmonium beamline, made it impossible to resolve the distinct peaks and the extraction of reliable momentum coordinates.

In a following section, I will present a full reconstruction of the Fermi surface based on these four momentum cuts to provide a more detailed description of the material’s evolution under intense femtosecond light illumination.
Before doing so, however, an additional quantity can be extracted from the bandmaps, which will greatly help in understanding the Fermi surface evolution.

\section{Distinguishing effects of photodoping and screening}
\label{sec:mu}

Additional insights can be gained from the fluence-dependent series of bandmaps.
When analyzing the change in the $k_F$ points, keeping in mind that these points represent the endpoints of the parabolic dispersion, two distinct mechanisms could explain the observed momentum shift.
The first possibility is that a true shift in $k$ across the parabolic feature causes the quasiparticle to appear at momenta closer to \qty{0}{\per\angstrom}.
The second possibility is an upward shift of the entire band structure, which would result in smaller $k_F$ values if the same energy slice is used in the MDC fitting procedure.

Although both mechanisms would lead to a similar observed result, $k_F$ points moving towards the center, the underlying physical causes are quite different.
Screening results in a reduction of the electric field, caused by mobile charges, as experienced by the system.
A change in screening could result from an induced charge imbalance between electrons and holes, leading to a modification of the Coulomb potential.
In the band structure such a charge imbalance would result in a rearrangement of the momentum of the charges.

On the other hand, a potential energy shift of the band structure could have two primary causes.
The first is space charge effects, which occur when a dense cloud of photoelectrons is emitted simultaneously, causing broadening and a shift in the detected photoelectron energy due to Coulomb repulsion.
A more detailed discussion of space charge effects is provided in Section \ref{sec:space_charge}.
To ensure that the observed energy shift is not due to space charge effects, the Fermi level is checked for any movement, as space charge would also affect its position.
This is done by integrating over a momentum range where no band is present, in this case from \qtyrange{0.4}{0.5}{\per\angstrom}, and fitting the resulting energy distribution curve (EDC) using a Fermi-Dirac equation broadened by a Gaussian function, given by
\begin{equation}
	I = \left[ C + \frac{1}{e^{(E-E_F)/k_BT}+1} \right] * g(E,\Delta E)
\end{equation}
where $C$ is a constant background, $E_F$ is the Fermi level, and $k_B$ is the Boltzmann constant.
The symbol $*$ denotes the convolution operation, and $g(E, \Delta E)$ represents a Gaussian function with $\Delta E$ being the energy resolution.
An example of an EDC fit can be seen in Fig. \ref{fig:fermi_fit_bi2212} (a).
Fig. \ref{fig:fermi_fit_bi2212} (b) shows the extracted Fermi level for a fluence-dependent measurement series, demonstrating that the Fermi level remains stable within the energy resolution of \qty{150}{\milli\electronvolt}.
Only for the highest fluence of \qty{1.15}{\milli\joule/\centi\meter\squared} can a broadening and small energy shift be observed.

\begin{figure}[t]
	\centering
	\begin{subfigure}[b]{0.33\textwidth}
		\includegraphics[width=\textwidth]{bi2212/fermi_fit_1354}
		\caption{}
	\end{subfigure}
	\begin{subfigure}[b]{0.33\textwidth}
		\includegraphics[width=\textwidth]{bi2212/fermi_fit_fluence_1354}
		\caption{}
	\end{subfigure}
	\caption{(a) The figure shows an EDC for the fluence of \qty{0.09}{\milli\joule/\centi\meter\squared} with the corresponding fit determining the Fermi level. (b) Shows the same EDC for all recorded fluences as a colormap. Red markers shows the center of the Fermi level as determined from the respective fit.}
	\label{fig:fermi_fit_bi2212}
\end{figure}

The second possible cause for an upward shift in the band structure is photodoping, which is a change in the number of electrons occupying the material’s electronic states.
In ARPES measurements, the system is electronically grounded, and the Fermi level remains fixed.
Therefore, a change in the occupation number would manifest as an upward or downward shift of the band structure relative to $E_F$.
An upward shift, which would cause the quasiparticle peak to move toward the center, reflects a reduction in the chemical potential and a decrease in the number of electrons occupying the states.
In this section, I will discuss how to quantify this upward band shift and why it cannot fully explain the shift of the quasiparticle peaks.

\begin{figure}[b!]
	\centering
	\begin{subfigure}[b]{0.27\textwidth}
		\includegraphics[width=\textwidth]{bi2212/mu_parab_example}
		\caption{}
	\end{subfigure}
	\begin{subfigure}[b]{0.33\textwidth}
		\includegraphics[width=\textwidth]{bi2212/mu_edge_example}
		\caption{}
	\end{subfigure}
	\caption{The figure shows part of the bandmap, with a zoom of (a) the parabolic feature at $E_F$ and (b) of the \ce{Cu} d-band. A step function broadened by a Gaussian is used to determine the leading edge of each feature. Red markers indicate the extracted center of the edge. Both bandmaps show the position of the leading edge for a pump fluence of \qty{0.09}{\milli\joule/\centi\meter\squared}.}
	\label{fig:mu_center}
\end{figure}

Several spectral features could be used to quantify the photodoping effect, but due to the lower energy resolution, the analysis is restricted to three options, the bottom of the parabolic dispersion, and the center and edge of the \ce{Cu} d-band.
In each case, a broadened step function is fitted to the feature to determine the leading edge, and the energy shift is tracked as a function of fluence.
Fig. \ref{fig:mu_center} illustrates how these positions are determined, with (a) showing the fit of the bottom of the parabola, and (b) showing the fitted leading edge of the \ce{Cu} d-band.
In the first case, the bottom of the parabola is determined by taking the center of the two quasiparticle peaks, while the step function fit extracts the leading edge position.
This method assumes a parabolic feature, but in reality, there is asymmetric spectral weight distribution, which complicates the analysis.
Additionally, the asymmetric closing of the parabola due to the slight planar misalignment (see Sec. \ref{sec:larger_effect}) makes this problem worse

Thus, using the leading edge of the \ce{Cu} d-bands provides a more reliable alternative, benefiting from a higher signal-to-noise ratio.
The momentum range is divided into \qty{10}{\per\angstrom} bins to improve signal quality, and the center position of the leading edge is extracted using a broadened step function across the entire momentum range.
The extracted position, shown in Fig. \ref{fig:mu_center} (b), displays a curved distribution that is relatively flat under the incoherent waterfall feature.
For this reason, the leading edge of the \ce{Cu} d-band center is used to quantify the shift as a function of pump fluence.
Similar analysis is performed for the parabola and the d-band edges to confirm that the shift is consistent across the band structure.

\begin{figure}[b!]
	\centering
	\begin{subfigure}[b]{0.33\textwidth}
		\includegraphics[width=\textwidth]{bi2212/mu_shift_cuband}
		\caption{}
	\end{subfigure}
	\begin{subfigure}[b]{0.33\textwidth}
		\includegraphics[width=\textwidth]{bi2212/mu_shift_all}
		\caption{}
	\end{subfigure}
	\begin{subfigure}[b]{0.33\textwidth}
		\includegraphics[width=\textwidth]{bi2212/fermi_fluence_dep}
		\caption{}
	\end{subfigure}
	\caption{(a) The figure shows the average relative (to the lowest fluence case) shift of the leading edge of the \ce{Cu} d-bands for the center region of the band map \qtyrange{-0.2}{0.2}{\per\angstrom} and the edge of the bandmap. There is virtually no difference between the two cases. (b) The figure shows the same graph as (a), but including the relative shift extracted from the parabolic feature. A bigger errorbar can be observed, as well as a strong deviation at the highest fluence. A linear regression for all cases is shown (excluding the highest fluence point for the parabolic feature).}
	\label{fig:mu_shift}
\end{figure}

The analysis of the leading edge shift as a function of fluence shows a linear shift, with an amplitude of approximately \qty{40}{\milli\electronvolt} at the maximum fluence of \qty{1.15}{\milli\joule/\centi\meter\squared} (Fig. \ref{fig:mu_shift}).
Fig. \ref{fig:mu_shift} (a) also shows that both the center and edges of the \ce{Cu} d-band shift by the same amount.
The parabolic feature behaves similarly, except at the highest fluence, where the feature becomes less clear and the fit procedure fails.
The fitting error for the parabolic feature is also significantly larger than for the \ce{Cu} d-band.

Overall, the analysis shows a clear upward shift of the band structure, which is linear with fluence and consistent across the bandmap.
A linear regression of the data points yields a chemical potential shift of \qty{33}{\frac{\milli\electronvolt}{\milli\joule/\centi\meter\squared}}.
This combined with the fact that the Fermi level does not show a linear upwards shift (see Fig. \ref{fig:mu_shift} (c))) means that the chemical potential $\mu$ changes and scales linearly with the fluence.
Therefore, we attribute the observed shift to photodoping, which plays a significant role in the out-of-equilibrium properties of cuprates.

The next step is to determine whether photodoping is the dominant cause of the observed quasiparticle shift or if screening effects must also be considered.

\section{Modeling the Fermi surface: A tight binding approach}
\label{sec:tb}

This section focuses on the role of screening in the observed quasiparticle shifts after light irradiation.
To address this, I will use the results from the photodoping analysis and attempt to recreate the Fermi surface (FS) using a tight binding approach, which is a standard, but phenomenological method for plotting the FS of \ce{Bi}2212.
Despite its simplicity, the tight binding approach successfully reproduces the FS across a wide doping range \cite{markiewicz_one-band_2005}.
The model parameters were compared to literature values for the unpumped state, with minor adjustments made to account for the out-of-equilibrium FS at different fluences.

\begin{figure}
	\centering
	\includegraphics[width=0.6\linewidth]{images/bi2212/Cu_plackett}
	\caption{The figure shows the \ce{CuO} placket, which is mostly responsible for the electronic properties of \ce{Bi}2212. Additionally the main microscopic charge properties are marked, which are responsible for the change of the FS, including the Coulomb potential $U$, the charge transfer energy CT, and the nearest and next nearest neighbor hopping $t$ \& $t'$.}
	\label{fig:cuplackett}
\end{figure}

The tight binding model generally includes a series of cosine terms, with their amplitude depending on the hopping parameters.
In the model used here, terms up to the 3rd order are used
\begin{equation}
	E = \frac{\mu}{t} + \frac{1}{2} \left(\cos(k_xa)+\cos(k_ya)\right) + \frac{t'}{t} \cos(k_xa)\cos(k_ya) + \frac{t''}{2t} \left(\cos(2k_xa)+\cos(2k_ya)\right)
\end{equation}
with the nearest neighbor hopping parameter $t$, the next nearest neighbor hopping parameter $t'$ and the second next nearest neighbor hopping parameter $t''$, the chemical potential $\mu$ and the lattice parameter $a=\qty{3.83}{\angstrom}$, reflecting the \ce{Cu}-\ce{Cu} distance.
Higher order terms, especially terms accounting for the splitting of bonding and anti-bonding bands were ignored, as it is not possible to individually resolve the two branches with the Harmonium beamline.

The tight binding model was fitted to the extracted $k_F$ points at the Fermi level.
Several assumptions were made to reduce the number of free parameters.
First, $t'$ and $t''$ were fixed based on values from previous studies on optimally doped \ce{Bi}2212 \cite{kondo_hole-concentration_2004}.
The chemical potential was also treated as a fixed parameter, again using values from the literature \cite{kondo_hole-concentration_2004}.
This leaves only the nearest neighbor hopping parameter $t$ as a free fitting parameter.
For higher fluences, $t'$ and $t''$ were kept constant, justified by results showing that the band structure is well described by holding these parameters fixed across different doping levels \cite{drozdov_phase_2018}.
The chemical potential $\mu$ was adjusted according to the relative shift extracted in the previous section, while the nearest neighbor hopping $t$ remained the only free parameter, with its value from the lower fluence case used as the initial guess.
This iterative process was used to reconstruct the FS for the full fluence evolution.

\begin{figure}[b!]
	\centering
	\begin{subfigure}[t!]{0.33\textwidth}
		\includegraphics[width=\textwidth]{bi2212/single_fluence_tb}
		\caption{}
	\end{subfigure}
	\begin{subfigure}[t!]{0.33\textwidth}
		\includegraphics[width=\textwidth]{bi2212/fs_tb_full}
		\caption{}
	\end{subfigure}
	\caption{The figures shows the tight binding FS of (a) a fluence of \qty{0.09}{\milli\joule/\centi\meter\squared} together with optimally doped \ce{Bi}2212 literature data from \cite{kondo_hole-concentration_2004} and (b) for all measured fluences. The full FS is recreated by utilizing the materials four-fold symmetry. The measured datapoints are marked respectively.}
	\label{fig:fs_tb}
\end{figure}

To visualize the full FS, additional points were added based on the four-fold lattice symmetry.
Fig. \ref{fig:fs_tb} (a) shows the fit results for the lowest fluence case, which compares well with literature data.
The full FS evolution is shown in Fig. \ref{fig:fs_tb} (b), along with the experimental data points.

The figure clearly shows that the FS changes both in curvature and the size of the enclosed area.
At the lowest fluence (\qty{0.09}{\milli\joule/\centi\meter\squared}), the FS encloses a hole pocket centered around the ($\pi$,$\pi$)-point.
As fluence increases, significant changes occur near the anti-node ($\pi$,$0$), as previously observed in Sec. \ref{sec:angle}.
At a fluence of \qty{0.88}{\milli\joule/\centi\meter\squared}, the tight binding fit predicts that the enclosed FS area shifts from being centered around ($\pi$,$\pi$) to being centered around $\Gamma$, indicating a transformation from a hole pocket to an electron pocket.
This transformation is referred to as a Lifshitz transition, a well-known phenomenon in various hole-doped cuprates, including \ce{Bi}2212, \ce{Bi}2201, LSCO, and \ce{Nd}-LSCO \cite{kaminski_change_2006,matt_electron_2015,ding_disappearance_2019,kondo_hole-concentration_2004}.
Under equilibrium conditions, the Lifshitz transition occurs as hole doping increases, with the transition happening at $p=0.22$ in \ce{Bi}2212 \cite{kaminski_change_2006}.

It is important to note that the exact shape of the reconstructed FS depends on the fixed parameters used in the fit.
Although substantial effort was made to optimize the values of $t'$ and $t''$, different parameter sets can fit the data points similarly well, especially in the anti-nodal region.
Therefore, the reconstructed FS as a function of fluence should be viewed as a qualitative rather than a definitive representation of the pump-induced changes.
Regardless, the nodal region remains largely fixed, and while the magnitude of change at the anti-node may vary with the fit parameters, the bands consistently move towards the ($\pi$,$0$) position, eventually resulting in a Lifshitz transition.
Thus, irradiation with femtosecond infrared light induces changes in the FS that mimic the behavior observed with increased hole doping.

\begin{figure}
	\centering
	\includegraphics[width=0.4\linewidth]{images/bi2212/mu_vs_t}
	\caption{The graph shows a comparison between the relative change in chemical potential $\mu$, as discussed in section \ref{sec:mu}, and the change in the nearest neighbor hopping $t$. It shows that the change in hopping is nearly 3 times as high as the shift of the chemical potential.}
	\label{fig:mu_t}
\end{figure}

In addition to visualizing the FS evolution, the fit provides an estimate of the change in the nearest neighbor hopping parameter.
Plotting the relative change in $\mu$ alongside the relative change in $t$ reveals that the change in $t$ is larger than the shift in $\mu$.
Since $t$ strongly influences the Coulomb potential $U$ at the \ce{Cu} site, this also affects screening.
The difference between the changes in $\mu$ and $t$ is shown in Fig. \ref{fig:mu_t}.

Having discussed the quasiparticle shifts and the reconstruction of the FS, the next section will focus on the time duration of the observed effects.

\section{Effect duration and metastability}
\label{sec:meta}

In many cases, trARPES measurements are conducted on the timescale of a few picoseconds to study the ultrafast dynamics following femtosecond pulse excitation.
Similarly, the early experiments in this study focused on observing ultrafast excitation and the subsequent decay within a picosecond window.
However, surprisingly, no change in the quasiparticle peak was detected during the time scan, but instead only when compared to a different scan, in which the system was not excited.
This implies that not only does the system fail to return to equilibrium within the observed delay window, but it also does not return to equilibrium between successive probe pulses.
Therefore, a lower limit for the effect duration can be placed by the inverse repetition rate of the laser, which corresponds to $\tau=\frac{1}{\qty{6}{\kilo\hertz}}=\qty{166.66}{\mu\second}$.

\begin{figure}
	\centering
	\includegraphics[width=0.7\linewidth]{images/bi2212/pulse_scheme}
	\caption{The figure shows a sketch of the pulse scheme. A fixed delay position of \qty{-2}{\pico\second} was used for the fluence dependent scans. This results in an overall time span of $\qty{166.66}{\micro\second}-\qty{2}{\pico\second}$ between the excitation of the sample and the probe.}
	\label{fig:pulsescheme}
\end{figure}

To confirm the effect duration, the fluence-dependent scans were performed using a somewhat unconventional pump-probe configuration.
Instead of measuring the time between the pump and probe pulses at a positive delay, negative delays were used, meaning that the probe pulse arrived before the excitation pulse.
When considering the entire sequence of pulses arriving at a rate of \qty{6}{\kilo\hertz}, this effectively results in an observed time delay that corresponds to the time interval between the pump pulse and the subsequent probe pulse.
This corresponds to a delay of $\Delta t = \qty{166.66}{\micro\second} - \qty{1}{\pico\second}$, reflecting the inverse repetition rate minus the negative delay between the pump and probe pulses.

Furthermore, the effect was confirmed to be reversible.
Repeating the scan without any infrared pump after the fluence-dependent measurement showed that the band structure had returned to its equilibrium state.
This places an upper limit on the effect duration, which is on the order of tens of seconds, based on the time it takes to acquire a bandmap with a reasonable signal-to-noise ratio.

This result contrasts with observations reported by Rameau et al. \cite{rameau_photoinduced_2014}, where the effect duration was found to be on the order of \qty{1}{\pico\second}, with slight variations depending on the pump fluence.
Such a discrepancy hints towards a sample depended influence on the observed effect.

\section{Conclusion and Outlook}

The primary observation in this study is the shift in the quasiparticle peaks when the \ce{Bi}2212 sample is illuminated with intense femtosecond infrared light.
It was shown that the magnitude of this shift varies depending on the angle at which the measurement was performed.
Specifically, the shift towards \qty{0}{\per\angstrom} becomes more pronounced at higher parallel momenta, or in other words, closer to the anti-node.
By mapping the FS at multiple angles, it was possible to recreate its evolution using a tight binding approximation.
The FS evolution across different fluences revealed that light-induced effects mimic the results typically seen in doping-dependent measurements, with light apparently inducing higher hole doping in the sample.

Further analysis of spectral features revealed a rigid band shift caused by changes in the chemical potential.
It was possible to quantify this shift and with the information it was possible to fit the hopping parameter in the recreation of the FS.
These shifts in chemical potential and hopping parameter correspond to two distinct physical mechanisms.
The change in chemical potential is attributed to photodoping, where light illumination removes charges, leading to a reduction in chemical potential.
On the other hand, the shift in the hopping parameter, along with the Coulomb potential, is a result of screening due to the presence of long-lived photodoped carriers.
Comparison of these contributions showed that the change in the hopping parameter represents the largest energy change within the tight binding Hamiltonian.

Additionally, it was shown that the effect persists for more than \qty{166.66}{\micro\second}, indicating the formation of a metastable state that returns to equilibrium after stopping the optical excitation.
Comparing the time duration to a similar observation by Rameau et al. \cite{rameau_photoinduced_2014} shows that the time duration is rather sample dependent, as there measurements showed a decay of the quasiparticle shift within a picosecond.
Which shows that while the magnitude and duration of the effect are sample dependent, the nature of the effect, with the ability to manipulate the microscopic electronic properties is rather fundamental.

Several possible mechanisms could explain this photoinduced phenomenon, including photodoping, reordering of oxygen vacancies, or a dynamic Hubbard $U$ \cite{baykusheva_ultrafast_2022}.
A parallel can be drawn to the phenomenon of persistent photosuperconductivity (PPS) observed in YBCO, where an enhanced superconducting critical temperature persists for over \qty{200}{\minute}.
The cause of PPS is still debated, with arguments revolving around photodoping and oxygen vacancy reordering \cite{gilabert_photodoping_2000}.
Long photodoping lifetimes can be explained by electron-hole pair excitation followed by electron trapping by charged defects \cite{el_hage_disentangling_2024, kudinov_persistent_1993, kudinov_mechanisms_1994}.
Alternatively, PPS might be driven by photoinduced oxygen vacancy reordering, which increases carrier density in the \ce{CuO2} planes.
A recent study \cite{el_hage_disentangling_2024} argued that oxygen reordering, not photodoping, is the primary mechanism behind enhanced superconducting properties.
The reasoning in this report is, that enhanced superconducting properties were only observed together with an increased carrier mobility, which they observed to be uncorrelated to the trapping of electrons.
Whether these insights apply to \ce{Bi}2212 remains an open question for future research, but a recent XPS study \cite{puntel_out--equilibrium_2024} provides initial evidence suggesting that oxygen reordering is relevant in infrared-excited \ce{Bi}2212.

It is essential to emphasize that the current data does not definitively support any particular mechanism, leaving room for further investigations.
One avenue could be additional trARPES measurements using lower repetition rates, which would extend the time delay between pump and probe pulses but at the cost of significantly reducing the signal-to-noise ratio, making an already challenging experiment even harder.
Another option could be changing the pump energy to a value below the charge transfer gap, which might influence the photoinduced effect differently.
Beyond trARPES, other techniques like time resolved transport measurements or ultrafast electron diffraction, which might be sensitive to any light induced oxygen reordering, could help in further understand the microscopic mechanisms at play.

The observed effect has intriguing and promising implications.
First, the ability to transiently replicate the FS and, by extension, the electronic properties of samples with higher hole doping opens up new possibilities.
This capability could advance the understanding of the material by allowing doping-dependent studies on a single sample, reducing uncertainty compared to switching between different samples.
Moreover, the demonstrated light-based control of the material's properties could be relevant for applications, such as controlled modification of a superconductor’s transport properties.
Additionally, the ability to alter electronic properties suggests potential applications in electronic devices such as switches.

Here, an important consideration is the effect on the material’s superconducting properties.
Typically, intense femtosecond pulses transiently destroy superconductivity, which then reforms on a picosecond timescale.
The present effect strongly outlasts this time frame, so it is interesting to pose the question if the FS of the material has changed, and the superconducting properties have reformed again?
Or do strong modifications to the material prevent this from happening?
If SC can reform, it may be possible to transiently manipulate the phase diagram, moving horizontally across different phases.
In materials science, metal-to-insulator transitions are of great interest, and the present scenario could pave the way for driving an insulator-to-superconductor transition solely through light illumination.

\begin{figure}
	\centering
	\includegraphics[width=0.7\linewidth]{images/bi2212/phase_manip.pdf}
	\caption{The figure shows the generic phase diagram of hole-doped cuprates. \cite{keimer_quantum_2015}. The red arrow marks a potential lightinduced insulator-to-superconductor transition, as a Gedankenexperiment to the presented light-induced effects.}
	\label{fig:phasemanip}
\end{figure}

The long-lasting nature of the effect is particularly promising in this regard.
Many ultrafast or light-induced phenomena suffer from rapid decay, meaning the effect must be reapplied frequently to sustain the driven state.
At present, the factors determining the effect’s duration are unclear, as Rameau et al. \cite{rameau_photoinduced_2014} observed an effect lasting only picoseconds in their samples.
By further investigating the fundamental mechanisms behind the photoinduced shifts, it may be possible to understand the sample requirements and even design materials tailored for specific needs.