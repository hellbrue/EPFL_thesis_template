\chapter{Preliminary study on the cuprate \ce{Bi}2223}

In previous chapter, I provided an overview of the significance of superconductors, discussing the general theoretical framework and discussing the experimental investigation of photo-induced effects on the two-layered cuprate \ce{Bi2Sr2CaCu2O8} (\ce{Bi}2212).
This material has been extensively studied, especially using Angle-Resolved Photoemission Spectroscopy (ARPES).
However, despite its importance, \ce{Bi}2212 does not exhibit the highest critical temperature $T_c$ among cuprates, nor within the bismuth-based family.
The three-layered counterpart, \ce{Bi2Sr2Ca2Cu3O10} (\ce{Bi}2223), surpasses \ce{Bi}2212 with a critical temperature of approximately \qty{108}{\kelvin}.

The dependence of $T_c$ on the number of \ce{CuO2} layers is a crucial factor in unraveling the origin of high-$T_c$ superconductivity \cite{feng_electronic_2002, scott_layer_1994, chakravarty_explanation_2004,iyo_tc_2007,chu_hole-doped_2015,luo_electronic_2023}.
In the bismuth-based cuprate family \ce{Bi2Sr2Ca_{n-1}Cu_nO_{2n+4}}, $T_c$ increases systematically from \qty{33}{\kelvin} for $n=1$, to \qty{92}{\kelvin} for $n=2$, peaking at \qty{108}{\kelvin} for $n=3$.
Interestingly, this trend reverses beyond $n=3$, where further increases in the number of \ce{CuO2} layers result in a decrease in $T_c$.
This behavior is mirrored across several cuprate families, where tri-layered compounds such as \ce{Bi}2223, \ce{TlBa2Ca2Cu3O9} (with $T_c$ of \qty{125}{\kelvin}), and \ce{HgBa2Ca2Cu3O8} (with $T_c$ of \qty{133}{\kelvin}) demonstrate the highest critical temperatures.
Figure \ref{fig:tcdep} (a) shows the dependence of the critical temperature on the amount of \ce{CuO2} layers, for three different cuprate families.
In addition to their elevated $T_c$, these tri-layered compounds exhibit an intriguing phase diagram, deviating from the characteristic dome-shaped superconducting phase observed in other cuprates \cite{fujii_doping_2002,piriou_effect_2008}.
Instead, they display a relatively constant $T_c$ in the overdoped regime, suggesting an unusual doping dependence that demands further investigation (see Fig. \ref{fig:tcdep} (b)).

\begin{figure}
	\centering
	\includegraphics[width=1\linewidth]{images/bi2223/tc_dep}
	\caption{(a) The figure shows the dependence of the maximum critical temperature on the amount of \ce{CuO2} layers for three different cuprate families. In each case the tri-layer compound exhibits the highest $T_c$. (b) Shows the doping dependence of the critical temperature, with the doping being normalized to the optimally doped case. \ce{Bi}2212 and \ce{Bi}2223 are compared with each other, showing the different behavior of $T_c$ in the overdoped range for three-layer cuprates. From \cite{luo_electronic_2023}}
	\label{fig:tcdep}
\end{figure}


The mechanisms driving both the layer dependence of $T_c$ and the unique behavior at higher doping levels remain elusive.
Understanding why tri-layered cuprates possess the highest $T_c$, and what differentiates them from their single- and bi-layer counterparts, could provide valuable insights into the underlying physics of high-$T_c$ superconductivity.
Among the tri-layered cuprates, \ce{Bi}2223 stands out as the only compound studied using ARPES \cite{feng_electronic_2002,muller_fermi_2002,sato_low_2002,matsui_bcs-like_2003,ideta_angle-resolved_2010, ideta_enhanced_2010, ideta_energy_2012,kunisada_observation_2017,ideta_hybridization_2021,luo_electronic_2023}, albeit significantly less frequently than \ce{Bi}2212.
One of the primary challenges has been the scarcity of high-quality \ce{Bi}2223 samples, though recent advancements have made these samples more readily available, with commercial sources now supplying them.

In bi-layer cuprates, ARPES measurements have revealed the splitting of Fermi arcs into bonding and anti-bonding bands, arising from the interlayer interactions between the two \ce{CuO2} planes.
By analogy, one would expect a similar splitting into three bands in \ce{Bi}2223 due to the interactions between the three \ce{CuO2} layers.
However, until recently, ARPES measurements on \ce{Bi}2223 had only revealed two Fermi sheets.
A tri-layer band splitting was observed for the first time in overdoped (OD) \ce{Bi}2223 \cite{luo_electronic_2023}, pointing to the existence of weak interlayer couplings and a charge imbalance between the inner and outer \ce{CuO2} layers as potential explanations for the absent third band.

In this chapter, I will provide a summary of the current understanding of \ce{Bi}2223, with a focus on recent experimental observations of overdoped samples.
This will be followed by a preliminary analysis of the data obtained from our measurements on optimally doped (OP) \ce{Bi}2223 at the SLS-ULTRA endstation at the PSI synchrotron, which offers new insights into the electronic structure of this promising high-$T_c$ superconductor.

\section{Previous Publications}

As introduced earlier, three-layer band splitting in the cuprate \ce{Bi}2223 has so far only been observed in its overdoped (OD) state \cite{luo_electronic_2023}.
Recent ARPES measurements have provided the first clear evidence of this splitting, capturing the electronic structure in a limited window around one Fermi arc.
The data revealed two bands similar to the bi-layer splitting seen in \ce{Bi}2212, as well as a third, well-separated band (see Fig. \ref{fig:arpes_sketch} (a)).
Energy distribution curves (EDCs) symmetrized around $E_F$ allowed for the extraction of superconducting gaps, with a gap amplitudes of approximately \qtylist{17;29;62}{\milli\electronvolt}, for the three bands respectively.
Notably, the gap evolution demonstrated a subtle deviation from the typical d-wave symmetry (see Fig. \ref{fig:arpes_sketch} (b)).

\begin{figure}
	\centering
	\begin{subfigure}[b]{0.59\textwidth}
		\includegraphics[width=\linewidth]{images/bi2223/fermi_surface}
		\caption{}
	\end{subfigure}
	\hfill
	\begin{subfigure}[b]{0.4\textwidth}
		\includegraphics[width=\linewidth]{images/bi2223/gap_size}
		\caption{}
	\end{subfigure}
	\caption{(a) Shows the Fermi surface of \ce{Bi}2223, showing the splitting in three separate Fermi surface sheets. (b) Shows the gap amplitude for the three different bands. The gap amplitude is extracted from the symmetrized EDC curves. A slight deviation from the d-wave gyp symmetry can be observed. From \cite{luo_electronic_2023}.}
	\label{fig:arpes_sketch}
\end{figure}

To complement the experimental results, the measured band structure was modeled using a tight-binding approach, not dissimilar to the methodology employed in previous chapters \cite{luo_subtle_2021}.
In the case of the three-layer \ce{Bi}2223, contributions from the electronic states in both the inner and outer \ce{CuO2} planes were taken into account.
In addition to intralayer hopping parameters, interlayer hopping and pairing terms were incorporated into the model.
Due to the presence of three layers, two distinct types of interlayer interactions are relevant: those between the inner and outer planes ($t_{io}$) and those between the two outer planes ($t_{oo}$).

In previous studies, the interlayer coupling between the two outer planes was often approximated as negligible, based on the assumption that the distance between them is too large for significant interaction \cite{kunisada_observation_2017,ideta_hybridization_2021}.
This rationale was consistent with the absence of the third band in earlier measurements.
However, the observation of this band in the recent study \cite{luo_electronic_2023} necessitated the inclusion of hopping and pairing between the outer layers in the model.
The analysis revealed an unusual behavior of the interlayer hopping parameters.
Firstly, the hopping between the inner and outer planes exhibits a dome-shaped dependence, with a peak near the nodal region and suppression at the antinodes.
Secondly, contrary to expectations based on the unit cell geometry, the hopping between the two outer planes was found to be stronger than that between the inner and outer planes.
Several possible explanations for these findings were proposed.
One suggestion is that intercell hopping, particularly in the antinodal region, is responsible for the suppression of ($t_{io}$).
Additionally, it was hypothesized that increased hole doping in the outer planes could enhance the hopping between these layers, implying that both the distance and the doping levels govern the interlayer coupling in \ce{CuO2} planes.

\begin{figure}
	\centering
	\includegraphics[width=0.5\linewidth]{images/bi2223/parameter}
	\caption{Schematic of the three \ce{CuO2} layers, highlighting the different microscopic parameters. The intralayer hopping parameters $t$, $t'$ \& $t''$. As well as the interlayer hopping parameter between the inner and outer plane $t_{io}$ and between the two outer planes $t_{oo}$, as well as the corresponding pairings $\Delta_{io}$ \& $\Delta_{oo}$. From \cite{luo_electronic_2023}.}
	\label{fig:parameter}
\end{figure}


Two principal theories have been proposed to explain the higher $T_c$ observed in tri-layer cuprates.
One hypothesis attributes the enhanced $T_c$ to interlayer hopping, suggesting that these interactions play a direct role in increasing the critical temperature \cite{chakravarty_explanation_2004,nishiguchi_superconductivity_2013}.
The second, which is favored by the recent report on the OD \ce{Bi}2223, is a composite model that differentiates the roles of the inner and outer planes \cite{kivelson_making_2002,berg_route_2008,okamoto_enhanced_2008}.
In this model, the inner plane is thought to be underdoped, characterized by a large superconducting gap and strong pairing strength, while the outer planes are overdoped, contributing to phase stiffness, which is beneficial for maintaining a high $T_c$ \cite{emery_importance_1995} 33.
This separation of responsibilities allows the material to independently optimize pairing strength and phase coherence, thereby maximizing $T_c$.
Such optimization is not possible in single- or bi-layer cuprates, where both parameters must be balanced within the same planes.
However, the question remains as to why this mechanism fails for cuprates with more than three \ce{CuO2} layers, where this separation would be also possible.

Interestingly, the recent study suggests that three-layer splitting is absent in optimally doped (OP) \ce{Bi}2223, likely due to the lack of interlayer hopping between the outer planes \cite{luo_electronic_2023}.
The data indicate that higher doping levels are necessary for the appearance of $t_{oo}$ and thereby splitting to occur, implying that additional doped holes are primarily located in the outer layers.
This finding suggests that the strong interlayer hopping $t_{oo}$ is responsible for maintaining a high $T_c$ in the overdoped regime, though it does not explain why the tri-layer system exhibits a higher $T_c$ compared to \ce{Bi}2212 in the first place.

In the following section, I will present our data on optimally doped \ce{Bi}2223, comparing it to the existing data on overdoped samples, and explore potential insights into the mechanisms driving the superconductivity in this system.

\section{preliminary analysis on optimally doped \ce{Bi}2223}

The data presented in this section was collected as part of two short beamtimes at the PSI synchrotron.
The measured samples were provided in a collaboration with Enrico Giannini at the University of Geneva, and they were confirmed to be optimally doped \ce{Bi}2223 samples by measuring the magnetic susceptibility as a function of temperature in a vibrating sample magnetometer (VSM) in the crystal growth facility at EPFL.
In total four samples were explored during the beamtime, together with a \ce{Bi}2212 sample for comparison in equal conditions.
For the original proposal it was planned to perform a full Fermi surface characterization, both at room temperature (RT) and at \qty{12}{\kelvin}, with a focus on exploring the possible tri-layer splitting.
Additionally, we proposed to perform temperature sweeps across the transition temperature, well into the pseudogap phase and observe any possible changes in self energy, scattering rate or effective mass.
But at the time of the beamtime the temperature control was not working, therefore only scans at room temperature and a base temperature of \qty{12}{\kelvin} were performed.
Only a rudimentary analysis of the collected data is presented, as the project is only in the beginning stages.

During the beamtime it was possible to successfully map the First Brillouin zone of the material.
Figure \ref{fig:bi2223_fs} (a) shows the obtained Fermi surface at a temperature of \qty{12}{\kelvin}.
At a first glance the band structure appears to be extremely similar to that of \ce{Bi}2212, but small changes can be observed.
First of all no clear splitting of the Fermi arc can be observed similar to the bi-layer splitting in \ce{Bi}2212, but the Fermi arcs do broaden as they approach the anti-node, which could potentially be a consequent of a tiny splitting (see Fig. \ref{fig:bi2223_fs} (b)).
A further difference can be spotted in the zoomed in map displayed in figure \ref{fig:bi2223_fs} (c).
There on the inside of the Fermi arc, a second faint arc like feature can be observed.
This spectral feature shows a strong similarity to the band marked in red in figure \ref{fig:arpes_sketch} (a), from the OD \ce{Bi}2223 data in \cite{luo_electronic_2023}.

\begin{figure}[t!]
	\centering
	\begin{subfigure}[b]{0.6\textwidth}
		\includegraphics[width=\linewidth]{images/bi2223/Bi2223_FS_wMarkers}
		\caption{}
	\end{subfigure}
	\\
	\begin{subfigure}[b]{0.6\textwidth}
		\includegraphics[width=\linewidth]{images/bi2223/Bi2223_FS_bilayer}
		\caption{}
	\end{subfigure}
	\begin{subfigure}[b]{0.3\textwidth}
		\includegraphics[width=\linewidth]{images/bi2223/Bi2223_FS_trilayer}
		\caption{}
	\end{subfigure}
	\caption{(a) Shows the Fermi surface of \ce{Bi}2223, showing the splitting in three separate Fermi surface sheets. (b) Shows the gap amplitude for the three different bands. The gap amplitude is extracted from the symmetrized EDC curves. A slight deviation from the d-wave gyp symmetry can be observed. From \cite{luo_electronic_2023}.}
	\label{fig:bi2223_fs}
\end{figure}

Before commenting further on the tri-layer splitting, I will present the analysis of the gap in the main Fermi arc.
For that figure \ref{fig:edc_comparison} (a) shows an energy distribution curve (EDC) taken from the antinodal region of the arc and compares it with an EDC from the node.
To evaluate the gap correctly, a reference measurement was performed on a polycrystalline Gold sample, placed on the same sample mount as the \ce{Bi}2223 sample.
The comparison shows that a small gap exists at the antinode, but not at the node, as is expected from the d-wave symmetry of the gap amplitude.
The same comparison is done for a measurement on \ce{Bi}2212, which can be seen in figure \ref{fig:edc_comparison} (b).
As in the previous case, EDC of the Gold reference sample and the EDC taken at the node coincide with each other, showing the absence of a gap in that region.
Looking at the EDC at the anti-node a clear gap can be observed, and the gap size appears to have a larger amplitude than in the case of \ce{Bi}2223.

\begin{figure}
	\centering
	\begin{subfigure}[b]{0.49\textwidth}
		\includegraphics[width=\linewidth]{images/bi2223/Bi2223_EDCs}
		\caption{}
	\end{subfigure}
	\hfill
	\begin{subfigure}[b]{0.49\textwidth}
		\includegraphics[width=\linewidth]{images/bi2223/Bi2212_EDC}
		\caption{}
	\end{subfigure}
	\caption{(a) Shows the Fermi surface of \ce{Bi}2223, showing the splitting in three separate Fermi surface sheets. (b) Shows the gap amplitude for the three different bands. The gap amplitude is extracted from the symmetrized EDC curves. A slight deviation from the d-wave gyp symmetry can be observed. From \cite{luo_electronic_2023}.}
	\label{fig:edc_comparison}
\end{figure}


The amplitude of the gap and the symmetry can be further explored by plotting the EDCs along one Fermi arc and mirror them at the Fermi level $E_F$.
Figure \ref{fig:gap_bilayer} (a) shows this analysis for \ce{Bi}2223.
The EDCs were taken along one Fermi arc, and their exact position is highlighted with red crosses in the Fermi surface of Fig. \ref{fig:bi2223_fs} (a).
The left panel, corresponding to the scan taken at \qty{12}{\kelvin}, clearly shows the anisotropy of the gap amplitude, as is expected from the d-wave anisotropy.
A gap of approximately \qty{15}{\milli\electronvolt} can be extracted from the position of the quasiparticle peak.
In the right panel the EDCs for the same momentum points are shown for RT with the gap being fully closed.
Comparing this with the EDCs along a Fermi arc of \ce{Bi}2212 shows a similar anisotropic behavior (see Fig. \ref{fig:gap_bilayer} (b)), but also a larger gap amplitude of approximately \qty{30}{\milli\electronvolt} near the anti-node, similar to the observations already seen in Fig. \ref{fig:edc_comparison}.

A further observation can be made regarding the structure of the EDCs in figure \ref{fig:edc_comparison}.
In the case of \ce{Bi}2212, a strong difference between node and anti-node can observed at higher energies, beyond the quasiparticle peak (see Fig. \ref{fig:edc_comparison} (b)).
Here, at the anti-node, the onset of the so called peak-dip-hump structure is visible \cite{kordyuk_origin_2002}.
This peak-dip-hump structure is absent in the anti-nodal EDC of \ce{Bi}2223 (see Fig. \ref{fig:edc_comparison} (a)).
Additionally, the difference between the node and anti-node, apart from the superconducting gap, seem to be only marginal.

\begin{figure}
	\centering
	\begin{subfigure}[b]{0.49\textwidth}
		\includegraphics[width=\linewidth]{images/bi2223/Bi2223_gap_small}
		\caption{}
	\end{subfigure}
	\begin{subfigure}[b]{0.25\textwidth}
		\includegraphics[width=\linewidth]{images/bi2223/Bi2212_gap}
		\caption{}
	\end{subfigure}
	\caption{(a) Shows the Fermi surface of \ce{Bi}2223, showing the splitting in three separate Fermi surface sheets. (b) Shows the gap amplitude for the three different bands. The gap amplitude is extracted from the symmetrized EDC curves. A slight deviation from the d-wave gyp symmetry can be observed. From \cite{luo_electronic_2023}.}
	\label{fig:gap_bilayer}
\end{figure}

Coming back to the faint spectral feature seen in Fig. \ref{fig:bi2223_fs} (c), representing the additional band due to the tri-layer structure.
The feature can be visualized better when looking at the band cutting the additional arc like feature.
The results can be seen in figure \ref{fig:trilayer_splitting} for two different cuts.
A faint split band can be observed in Fig. \ref{fig:trilayer_splitting} (a), which appears to have similar as the more intense main band.
Closer to the anti-node the difference in dispersion is bigger, resulting in a bigger kink for the additional band (see Fig. \ref{fig:trilayer_splitting} (b)).

A difference between the two bandmaps lies in the polarization of the probe pulse.
While the probe of (a) is linearly polarized, in (b) a circular polarized probe was used.
Therefore it seems like the intensity of this band is dependent on the light polarization of the probe, and matrix element effects play a role.
Furthermore figure \ref{fig:trilayer_splitting} shows that the additional band has a bigger gap amplitude than the main Fermi arc, similar to the observations made in OD \ce{Bi}2223 \cite{luo_electronic_2023}.
The amplitude of the gap can be analyzed in greater detail, similar to the discussion for the main arc.
Several points are selected from close to the node to increasingly close to the anti-node.

The result of this can be seen in figure \ref{fig:trilayer_splitting}.
Similar to the main arc, an anisotropy of the gap amplitude is observed.
Closer to the anti-node a gap size of more than \qty{30}{\milli\electronvolt} is present, indicating a larger gap than in \ce{Bi}2212.
At room temperature has closed as expected (see Fig. \ref{fig:trilayer_splitting} right panel).

\begin{figure}
	\centering
	\begin{subfigure}[b]{0.3\textwidth}
		\includegraphics[width=\linewidth]{images/bi2223/Bi2223_EDM_splitting}
		\caption{}
	\end{subfigure}
	\begin{subfigure}[b]{0.3\textwidth}
		\includegraphics[width=\linewidth]{images/bi2223/Bi2223_EDM_splitting2}
		\caption{}
	\end{subfigure}
	\begin{subfigure}[b]{0.39\textwidth}
		\includegraphics[width=\linewidth]{images/bi2223/Bi2223_gap_large}
		\caption{}
	\end{subfigure}
	\caption{(a) Shows the Fermi surface of \ce{Bi}2223, showing the splitting in three separate Fermi surface sheets. (b) Shows the gap amplitude for the three different bands. The gap amplitude is extracted from the symmetrized EDC curves. A slight deviation from the d-wave gyp symmetry can be observed. From \cite{luo_electronic_2023}.}
	\label{fig:trilayer_splitting}
\end{figure}

The last feature I want to address can be seen in a zoom on a Fermi arc, along which the umklappbands of the superstructure overlap at the node position (see Fig. \ref{fig:fs_hybrid}).
In this specific configuration where the sample was aligned along the node and horizontally polarized light was used, a strong suppression of spectral weight at the node can be observed.
Similar observation could be made with sample alignment along the anti-node and linear polarized light, always resulting in a strong suppression of the spectral weight.
Most likely matrix element effects are the result of this suppression, but the reason why the node suddenly shows this polarization dependence is unclear.
Hybridization between the several bands crossing each other at this point could be a reason, but further investigation is necessary to understand is origin.

\begin{figure}[h!]
	\centering
	\includegraphics[width=0.5\linewidth]{images/bi2223/FS_hybrid_zoom}
	\caption{}
	\label{fig:fs_hybrid}
\end{figure}

\section{Conclusion}

To summarize the preliminary results from this chapter.
The band structure of OP \ce{Bi}2223 was successfully mapped and was shown to exhibit strong similarities compared to the bi-layer compound.
In the Fermi surface the main differences stem from a smaller anti-bonding splitting of the Fermi arc.
This splitting was more pronounced in the OD case, and it was discussed that a higher doping is necessary to see the splitting.
Even in the OD case only a faint splitting was observed, which is rather observed as a broadening of the band near the anti-node.
Similarly, a slight broadening can be observed here at the anti-node (see Fig. \ref{fig:bi2223_fs} (b)).
A more detailed analysis, especially with a focus on the anti-node is necessary to further discuss this feature.

The second difference to FS of \ce{Bi}2212 is the existence of a third band, whose origin is most likely the inner \ce{CuO2} layer of the unit cell (Fig. \ref{fig:bi2223_fs} (c)).
It shows a similar arc shaped dispersion as the main band and its spectral weight might be strongly matrix element dependent (Fig. \ref{fig:trilayer_splitting} (a)\&(b)).
The band shows a similar dispersion to the main abnd close to the node, with the kink becoming stronger closer to the anti-node.

A typical d-wave symmetry was found for both the main main and the side Fermi arc, with an ungapped nodal position and the gap opening progressively towards the anti-node.
A more careful and quantitative analysis on the gap size along the Fermi arc has to be made, to observe any potential deviations from the d-wave symmetry as observed in the OD compound \cite{luo_electronic_2023}.
But from the preliminary analysis it can be shown that the gap of the main arc, which is the one similar to the bi-layer compound, has a smaller gap amplitude than in \ce{Bi}2212.
Instead the additional arc from the inner layer shows a larger gap than both the main arc in \ce{Bi}2223 and the arc in \ce{Bi}2212.
This fact is supported by the composite picture, where the inner layer show a larger pairing strength and therefore a larger gap, and the outer layers are responsible for the phase stiffness.

Furthermore it was possible to make out differences between the bi- and tri-layer regarding the energies beyond the quasiparticle peak.
Here, the absence of the typical peak-dip-hump structure in the tri-layer stands out, but the implications are not yet obvious and a more detailed analysis is needed.

Lastly, a polarization dependent loss of spectral weight at the node was observed, at the point where the umklappbands overlap.
The reason for this is currently unknown, it is possible to speculate that hybridization between the several bands play a role, which alter the character of the band, but further investigation is needed to interpret the result.

In total, this preliminary analysis showed several interesting observations that might help in characterizing the OP \ce{Bi}2223.
A more detailed analysis on the presented data is necessary and warranted, especially with the comparison to the results on the OD compound in mind.
The discussed data only represents a fraction of the datasets taken at the beamtime.
Four different \ce{Bi}2223 samples were measured in total, with different combinations of sample alignment and probe polarizations as well as photon energy dependencies.
Here a complete study of the available datasets, with all the different configurations is necessary to further understand the material.

In addition, depending on the outcome of the analysis further studies could be envisioned.
With the access to commercial suppliers it is possible to perform more measurements, although the quality of those samples is yet to be determined.
Possible measurements that could be performed within the LACUS facility could focus on temperature dependent scan across the superconducting phase transition or the first time-resolved measurements.