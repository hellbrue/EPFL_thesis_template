\chapter{Preliminary study on the cuprate \ce{Bi}2223}

In the previous chapter, I discussed the significance of superconductors, providing a general overview of the topic along with an introduction to the system and its basic theoretical aspects.
The focus was on experimental studies of the two-layered cuprate superconductor \ce{Bi}2212, a widely studied material in Angle-Resolved Photoemission Spectroscopy (ARPES).
While \ce{Bi}2212 is a significant compound, it does not exhibit the highest critical temperature ($T_c$) among cuprates, nor even within the \ce{Bi}-based family.
The three-layered compound \ce{Bi2Sr2Ca2Cu3O10} (\ce{Bi}2223) shows a higher $T_c$ of approximately \qty{108}{\kelvin}.

The dependence of $T_c$ on the number of \ce{CuO} layers is crucial for understanding the mechanism behind high-$T_c$ superconductivity.
In the bismuth-based cuprate family \ce{Bi2Sr2Can−1CunO2n+4}, $T_c$ increases from \qty{33}{\kelvin} for $n=1$, to \qty{92}{\kelvin} for $n=2$, and reaches \qty{108}{\kelvin} for $n=3$.
However, when the number of \ce{CuO} layers exceeds $n=3$, $T_c$ decreases.
Among various cuprate families, the tri-layered compounds exhibit the highest $T_c$, including \ce{Bi2Sr2Ca2Cu3O10} ($T_c=\qty{108}{\kelvin}$), \ce{TlBa2Ca2Cu3O9} ($T_c\qty{125}{\kelvin}$), and \ce{HgBa2Ca2Cu3O8} ($T_c=\qty{133}{\kelvin}$).
