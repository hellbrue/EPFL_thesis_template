\chapter{Equilibrium study on the tri-layer cuprate \ce{Bi}2223}

In previous chapter, I provided an overview of the significance of superconductors, discussing the general theoretical framework and discussing the experimental investigation of photo-induced effects on the two-layered cuprate \ce{Bi2Sr2CaCu2O8} (\ce{Bi}2212).
This material has been extensively studied, especially using Angle-Resolved Photoemission Spectroscopy (ARPES), thanks to the availability of high quality single-crystal samples and the relative ease to expose an atomically clean surface by cleaving the topmost layers.
However, \ce{Bi}2212 does not exhibit the highest critical temperature $T_c$ among cuprates, nor within the bismuth-based family.
The three-layered counterpart, \ce{Bi2Sr2Ca2Cu3O10} (\ce{Bi}2223), surpasses \ce{Bi}2212 with a critical temperature of approximately \qty{108}{\kelvin}.

The dependence of $T_c$ on the number of \ce{CuO2} layers is a crucial factor in unraveling the origin of high-$T_c$ superconductivity \cite{feng_electronic_2002, scott_layer_1994, chakravarty_explanation_2004,iyo_tc_2007,chu_hole-doped_2015,luo_electronic_2023}.
In the bismuth-based cuprate family \ce{Bi2Sr2Ca_{n-1}Cu_nO_{2n+4}}, $T_c$ increases systematically from \qty{33}{\kelvin} for $n=1$, to \qty{92}{\kelvin} for $n=2$, peaking at \qty{108}{\kelvin} for $n=3$.
Interestingly, this trend reverses beyond $n=3$, where further increases in the number of \ce{CuO2} layers result in a decrease in $T_c$.
This behavior is mirrored across several cuprate families, where tri-layered compounds such as \ce{Bi}2223, \ce{TlBa2Ca2Cu3O9} (with $T_c$ of \qty{125}{\kelvin}), and \ce{HgBa2Ca2Cu3O8} (with $T_c$ of \qty{133}{\kelvin}) demonstrate the highest critical temperatures.
Figure \ref{fig:tcdep} (a) shows the dependence of the critical temperature on the amount of \ce{CuO2} layers, for three different cuprate families.
In addition to their elevated $T_c$, these tri-layered compounds exhibit an intriguing phase diagram, deviating from the characteristic dome-shaped superconducting phase observed in other cuprates \cite{fujii_doping_2002,piriou_effect_2008}.
Instead, they display a relatively constant $T_c$ in the overdoped regime, suggesting an unusual doping dependence that demands further investigation (see Fig. \ref{fig:tcdep} (b)).

\begin{figure}
	\centering
	\includegraphics[width=0.8\linewidth]{images/bi2223/tc_dep}
	\caption{(a) Dependence of the maximum critical temperature ($T_c$) on the number of \ce{CuO2} layers for three different cuprate families. In each case, the tri-layer compound exhibits the highest $T_c$. (b) Doping dependence of the critical temperature, with doping normalized to the optimally doped state. \ce{Bi}2212 and \ce{Bi}2223 are compared, highlighting the different behavior of $T_c$ in the overdoped regime for the tri-layer cuprates. Adapted from \cite{luo_electronic_2023}.}
	\label{fig:tcdep}
\end{figure}

The mechanisms driving both the layer dependence of $T_c$ and the unique behavior at higher doping levels remain elusive.
Understanding why tri-layered cuprates possess the highest $T_c$, and what differentiates them from their single- and bi-layer counterparts, could provide valuable insights into the underlying physics of high-$T_c$ superconductivity.
Among the tri-layered cuprates, \ce{Bi}2223 stands out as the only compound studied using ARPES \cite{feng_electronic_2002,muller_fermi_2002,sato_low_2002,matsui_bcs-like_2003,ideta_angle-resolved_2010, ideta_enhanced_2010, ideta_energy_2012,kunisada_observation_2017,ideta_hybridization_2021,luo_electronic_2023}.
One of the primary challenges has been the scarcity of high-quality \ce{Bi}2223 samples, though recent advancements have made these samples more readily available, with commercial sources now supplying them.

In bi-layer cuprates, ARPES measurements have revealed the splitting of Fermi arcs into bonding and anti-bonding bands, arising from the interlayer interactions between the two \ce{CuO2} planes.
By analogy, one would expect a similar splitting into three bands in \ce{Bi}2223 due to the interactions between the three \ce{CuO2} layers.
However, until recently, ARPES measurements on \ce{Bi}2223 had only revealed two Fermi sheets.
A tri-layer band splitting was observed for the first time in overdoped (OD) \ce{Bi}2223 \cite{luo_electronic_2023}, pointing to the existence of weak interlayer couplings and a charge imbalance between the inner and outer \ce{CuO2} layers as potential explanations for the absent third band.

In this chapter, I will provide a summary of the current understanding of \ce{Bi}2223, with a focus on recent experimental observations of overdoped samples.
This will be followed by an analysis of the data obtained from our measurements on optimally doped (OP) \ce{Bi}2223 at the SLS-ULTRA endstation at the PSI synchrotron, which offers new insights into the electronic structure of this promising high-$T_c$ superconductor.

\section{Previous Publications}

As introduced earlier, three-layer band splitting in the cuprate \ce{Bi}2223 has so far only been observed in its overdoped (OD) state \cite{luo_electronic_2023}.
Recent ARPES measurements have provided the first clear evidence of this splitting, capturing the electronic structure in a limited window around one Fermi arc.
The data revealed two bands similar to the bi-layer splitting seen in \ce{Bi}2212, as well as a third, well-separated band (see Fig. \ref{fig:arpes_sketch} (a)).
Energy distribution curves (EDCs) symmetrized around $E_F$ allowed for the extraction of superconducting gaps, with a gap amplitudes of approximately \qtylist{17;29;62}{\milli\electronvolt}, for the three bands respectively.
Notably, the gap evolution demonstrated a subtle deviation from the typical d-wave symmetry (see Fig. \ref{fig:arpes_sketch} (b)).

\begin{figure}[t]
	\centering
	\begin{subfigure}[b]{0.4\textwidth}
		\includegraphics[width=\linewidth]{images/bi2223/fermi_surface}
		\caption{}
	\end{subfigure}
	\begin{subfigure}[b]{0.3\textwidth}
		\includegraphics[width=\linewidth]{images/bi2223/gap_size}
		\caption{}
	\end{subfigure}
	\caption{(a) The Fermi surface of \ce{Bi}2223, displaying the splitting into three separate Fermi surface sheets. (b) The gap amplitude for the three different bands. The gap amplitude is extracted from the symmetrized EDC curves, with a slight deviation from typical d-wave symmetry observed. Adapted from \cite{luo_electronic_2023}.}
	\label{fig:arpes_sketch}
\end{figure}

To complement the experimental results, the measured band structure was modeled using a tight-binding approach, not dissimilar to the methodology employed in previous chapters \cite{luo_subtle_2021}.
In the case of the three-layer \ce{Bi}2223, contributions from the electronic states in both the inner and outer \ce{CuO2} planes were taken into account.
In addition to intralayer hopping parameters, interlayer hopping and pairing terms were incorporated into the model.
Due to the presence of three layers, two distinct types of interlayer interactions are relevant: those between the inner and outer planes ($t_{io}$) and those between the two outer planes ($t_{oo}$).

In previous studies, the interlayer coupling between the two outer planes was often approximated as negligible, based on the assumption that the distance between them is too large for significant interaction \cite{kunisada_observation_2017,ideta_hybridization_2021}.
This rationale was consistent with the absence of the third band in earlier measurements.
However, the observation of this band in the recent study \cite{luo_electronic_2023} necessitated the inclusion of hopping and pairing between the outer layers in the model.
The analysis revealed an unusual behavior of the interlayer hopping parameters.
Firstly, the hopping between the inner and outer planes exhibits a dome-shaped dependence, with a peak near the nodal region and suppression at the antinodes.
Secondly, contrary to expectations based on the unit cell geometry, the hopping between the two outer planes was found to be stronger than that between the inner and outer planes.
Several possible explanations for these findings were proposed.
One suggestion is that intercell hopping, particularly in the antinodal region, is responsible for the suppression of ($t_{io}$).
Additionally, it was hypothesized that increased hole doping in the outer planes could enhance the hopping between these layers, implying that both the distance and the doping levels govern the interlayer coupling in \ce{CuO2} planes.

\begin{figure}
	\centering
	\includegraphics[width=0.5\linewidth]{images/bi2223/parameter}
	\caption{Schematic of the three \ce{CuO2} layers, highlighting the different microscopic parameters. The intralayer hopping parameters $t$, $t'$ \& $t''$. As well as the interlayer hopping parameter between the inner and outer plane $t_{io}$ and between the two outer planes $t_{oo}$, as well as the corresponding pairings $\Delta_{io}$ \& $\Delta_{oo}$. Adapted from \cite{luo_electronic_2023}.}
	\label{fig:parameter}
\end{figure}


Two principal theories have been proposed to explain the higher $T_c$ observed in tri-layer cuprates.
One hypothesis attributes the enhanced $T_c$ to interlayer hopping, suggesting that these interactions play a direct role in increasing the critical temperature \cite{chakravarty_explanation_2004,nishiguchi_superconductivity_2013}.
The second, which is favored by the recent report on the OD \ce{Bi}2223, is a composite model that differentiates the roles of the inner and outer planes \cite{kivelson_making_2002,berg_route_2008,okamoto_enhanced_2008}.
In this model, the inner plane is thought to be underdoped, characterized by a large superconducting gap and strong pairing strength, while the outer planes are overdoped, contributing to phase stiffness, which is beneficial for maintaining a high $T_c$ \cite{emery_importance_1995}.
This separation allows to independently optimize pairing strength and phase coherence, thereby maximizing $T_c$.
Such optimization is not possible in single- or bi-layer cuprates, where both parameters must be balanced within the same planes.
However, the question remains as to why this mechanism fails for cuprates with more than three \ce{CuO2} layers, where this separation would be also possible.

Interestingly, a recent study suggests that three-layer splitting is absent in optimally doped (OP) \ce{Bi}2223, likely due to the lack of interlayer hopping between the outer planes \cite{luo_electronic_2023}.
The data indicate that higher doping levels are necessary for the appearance of $t_{oo}$ and thereby splitting to occur, implying that additional doped holes are primarily located in the outer layers.
This finding suggests that a strong interlayer hopping $t_{oo}$ is responsible for maintaining a high $T_c$ in the overdoped regime, though it does not explain why the tri-layer system exhibits a higher $T_c$ compared to \ce{Bi}2212 in the first place.

In the following section, I will present our data on optimally doped \ce{Bi}2223, comparing it to the existing data on overdoped samples, and explore potential insights into the mechanisms driving the superconductivity in this system.

\section{Preliminary analysis on optimally doped \ce{Bi}2223}

The experimental data presented in this section were collected during two beamtimes at the PSI synchrotron.
The \ce{Bi}2223 samples used in the study were provided through a collaboration with Enrico Giannini at the University of Geneva, and their optimal doping level was confirmed by measuring the magnetic susceptibility as a function of temperature using a vibrating sample magnetometer (VSM) at the crystal growth facility at EPFL.
In total, four optimally doped \ce{Bi}2223 samples were studied, alongside a \ce{Bi}2212 sample for comparison under identical experimental conditions.

The initial proposal for the experiment involved a Fermi surface characterization of \ce{Bi}2223 at both room temperature (RT) and the base temperature of the endstation of \qty{12}{\kelvin}, with a particular focus on exploring the possible tri-layer splitting in the electronic structure.
Additionally, we proposed temperature sweeps across the superconducting transition, extending into the pseudogap phase, to observe changes in key observables such as self-energy, scattering rate, and effective mass.
Unfortunately, due to issues with temperature control during the beamtime, only measurements at RT and the base temperature of \qty{12}{\kelvin} were performed.
This report provides a preliminary analysis of the collected data, as the project is still in its early stages.

We successfully mapped the band structure within the first Brillouin zone of the \ce{Bi}2223 samples.
The Fermi surface at \qty{12}{\kelvin}, shown in Fig. \ref{fig:bi2223_fs} (a), exhibits a band structure that closely resembles that of \ce{Bi}2212, although subtle differences are apparent.
It was first expected to observe a clear tri-layer splitting, somewhat resembling the \ce{Bi}2212 bonding/anti-bonding splitting.
Instead the observed arcs of \ce{Bi}2223 exhibit a broadening near the anti-node, which may suggest the presence of a small splitting that remains unresolved in these measurements (see Fig. \ref{fig:bi2223_fs} (b)).

\begin{figure}[t!]
	\centering
	\begin{subfigure}[b]{0.6\textwidth}
		\includegraphics[width=\linewidth]{images/bi2223/Bi2223_FS_wMarkers}
		\caption{}
	\end{subfigure}
	\\
	\begin{subfigure}[b]{0.6\textwidth}
		\includegraphics[width=\linewidth]{images/bi2223/Bi2223_FS_bilayer}
		\caption{}
	\end{subfigure}
	\begin{subfigure}[b]{0.3\textwidth}
		\includegraphics[width=\linewidth]{images/bi2223/Bi2223_FS_trilayer}
		\caption{}
	\end{subfigure}
	\caption{(a) The Fermi surface of \ce{Bi}2223, with red crosses marking the positions where EDCs were taken for Fig. \ref{fig:gap_bilayer}. (b) A zoomed view of the lower right Fermi arc in (a), highlighting the broadened region near the anti-node. (c) A zoomed view with adjusted contrast to better observe the additional Fermi sheet from the inner \ce{CuO2} layer, indicated by arrows in both (b) and (c)}
	\label{fig:bi2223_fs}
\end{figure}

A closer examination of the Fermi surface, presented in Fig. \ref{fig:bi2223_fs} (c), reveals an additional faint arc-like feature, neighboring the primary Fermi arc.
This feature is reminiscent of the band identified in overdoped \ce{Bi}2223 data ($\gamma$ sheet, solid red line, see Fig. \ref{fig:arpes_sketch} (a)), as reported by Luo et al. \cite{luo_electronic_2023}.
This observation hints at the presence of the expected tri-layer band splitting with two near degenerate bands, although further analysis is required to confirm its exact nature.

\begin{figure}[t]
	\centering
	\begin{subfigure}[b]{0.49\textwidth}
		\includegraphics[width=\linewidth]{images/bi2223/Bi2223_EDCs}
		\caption{}
	\end{subfigure}
	\begin{subfigure}[b]{0.49\textwidth}
		\includegraphics[width=\linewidth]{images/bi2223/Bi2212_EDC}
		\caption{}
	\end{subfigure}
	\caption{(a) Comparison of three different EDCs: one taken near the anti-node, one near the node of the primary arc of \ce{Bi}2223, and one from a polycrystalline gold reference sample. The Fermi level of the gold and the node coincide, indicating the absence of a gap, while a small gap is observed near the anti-node. (b) Similar plot as in (a), but with EDCs taken from the anti-node and node of \ce{Bi}2212, where a larger gap is observed at the anti-node.}
	\label{fig:edc_comparison}
\end{figure}

Next, I want to focus on discussing the superconducting gap along the primary Fermi arc.
Fig. \ref{fig:edc_comparison} (a) compares the energy distribution curves (EDCs) from the anti-nodal and nodal regions of the arc.
A reference measurement was conducted on a polycrystalline gold sample placed on the same mount as \ce{Bi}2223 to accurately determine the Fermi level and hence the gap size.
The comparison reveals the presence of a small gap at the anti-node but none at the node, consistent with the expected d-wave symmetry of the superconducting gap in cuprates.
For comparison, a similar analysis was performed on the \ce{Bi}2212 sample (Fig. \ref{fig:edc_comparison} (b)), which shows a more pronounced gap at the anti-node, with a larger amplitude than in \ce{Bi}2223.

Further analysis of the gap was conducted by plotting the EDCs along the Fermi arc and symmetrizing them around the Fermi level $E_F$.
Figure \ref{fig:gap_bilayer} (a) shows this analysis for \ce{Bi}2223 at \qty{12}{\kelvin}.
The gap amplitude $\Delta$ is extracted by fitting the center of the two peaks, and their distance corresponding to $2\Delta$.
The anisotropy of the gap is clearly visible, with a gap size of approximately \qty{15}{\milli\electronvolt} near the anti-node.
At RT, the gap is fully closed, as expected.
A similar anisotropic behavior is observed for \ce{Bi}2212, although the gap amplitude in \ce{Bi}2212 is significantly larger, reaching around \qty{30}{\milli\electronvolt} at the anti-node (see Fig. \ref{fig:gap_bilayer} (b)).

Another noteworthy difference between \ce{Bi}2223 and \ce{Bi}2212 is the absence of the characteristic peak-dip-hump structure in the \ce{Bi}2223 EDCs.
In \ce{Bi}2212, this feature is clearly visible at higher energies beyond the quasiparticle peak, particularly at the anti-node (see Fig. \ref{fig:edc_comparison} (b)), consistent with previous reports \cite{kordyuk_origin_2002}.
However, this structure is absent in the corresponding EDCs for \ce{Bi}2223 (see Fig. \ref{fig:edc_comparison} (a)), suggesting significant differences in the incoherent part of the electronic structure of these two materials.

\begin{figure}
	\centering
	\begin{subfigure}[b]{0.4\textwidth}
		\includegraphics[width=\linewidth]{images/bi2223/Bi2223_gap_small}
		\caption{}
	\end{subfigure}
	\begin{subfigure}[b]{0.2\textwidth}
		\includegraphics[width=\linewidth]{images/bi2223/Bi2212_gap}
		\caption{}
	\end{subfigure}
	\caption{(a) Evolution of EDCs across the primary Fermi arc, with the positions of the EDCs corresponding to the markers in Fig. \ref{fig:bi2223_fs} (a). The left panel shows the EDCs at \qty{12}{\kelvin}. From the anti-node at (\qty{-0.55}{\per\angstrom}, \qty{0.18}{\per\angstrom}) across the arc to the opposite anti-node near (\qty{-0.2}{\per\angstrom}, \qty{0.48}{\per\angstrom}), a superconducting gap is observed, which continuously closes towards the node and reopens afterward. The right panel shows the same EDCs at room temperature (RT), where the gap has disappeared. (b) The same EDCs along the Fermi arc of \ce{Bi}2212, where, in general, a larger gap is observed compared to the primary arc of \ce{Bi}2223.}
	\label{fig:gap_bilayer}
	\centering
	\begin{subfigure}[b]{0.27\textwidth}
		\includegraphics[width=\linewidth]{images/bi2223/Bi2223_EDM_splitting}
		\caption{}
	\end{subfigure}
	\begin{subfigure}[b]{0.3\textwidth}
		\includegraphics[width=\linewidth]{images/bi2223/Bi2223_EDM_splitting2}
		\caption{}
	\end{subfigure}
	\begin{subfigure}[b]{0.39\textwidth}
		\includegraphics[width=\linewidth]{images/bi2223/Bi2223_gap_large}
		\caption{}
	\end{subfigure}
	\caption{(a) Bandmap of a cut near the node, showing the band connected to the quasiparticle peak of the primary arc. A faint additional split band, corresponding to the Fermi sheet of the inner \ce{CuO2} layer, is visible. (b) Similar bandmap taken closer to the anti-node, where a stronger kink in the additional band, compared to the primary sheet, is apparent. The intensity difference between (a) and (b) may result from matrix element effects, as different light polarizations were used: linear in (a) and circular in (b). (c) Symmetrized EDCs along the additional Fermi sheet seen in Fig. \ref{fig:bi2223_fs} (c), progressing from near the node to the anti-node. The left panel shows scans taken at \qty{12}{\kelvin}, where the gap widens towards the anti-node and has a larger amplitude than for the primary arc in \ce{Bi}2223 and \ce{Bi}2212. The right panel shows the same EDCs at room temperature, where the gap has closed.}
	\label{fig:trilayer_splitting}
\end{figure}

After examining the gap amplitude, I will focus on the faint spectral feature seen in Fig. \ref{fig:bi2223_fs} (c), which is likely associated with the additional band from the tri-layer structure.
Fig. \ref{fig:trilayer_splitting} shows bandmaps along different cuts through the Brillouin zone.
A faint split band is visible in Fig. \ref{fig:trilayer_splitting} (a), with greater separation near the antinode (see Fig. \ref{fig:trilayer_splitting} (b)).
In addition the lower band shows a more pronounced kink at the anti-node, compared to the upper main band.

Interestingly, the intensity of this split band appears to depend on the polarization of the probe light.
In Fig. \ref{fig:trilayer_splitting} (a), linearly polarized light was used, while in Fig. \ref{fig:trilayer_splitting} (b) circularly polarized light was employed, leading to a stronger intensity of the additional band.
This suggests that matrix element effects play a role in the visibility of the tri-layer splitting, and further polarization-dependent studies are needed to fully understand this behavior.
The gap amplitude was analyzed in further detail, following a similar approach to that used for the main arc.
Data points were selected from regions near the node and moving progressively closer to the anti-node.
As shown in figure \ref{fig:trilayer_splitting} (c), the gap exhibits noticeable anisotropy, much like the main arc.
Near the anti-node, the gap amplitude exceeds \qty{30}{\milli\electronvolt}, indicating a larger gap compared to \ce{Bi}2212.
As expected, the gap is fully closed at room temperature (see right panel of Fig. \ref{fig:trilayer_splitting} (c).

One further feature to note is the suppression of spectral weight observed in the nodal region under certain polarization conditions.
Fig. \ref{fig:fs_hybrid} shows a zoom-in on the Fermi arc where the umklapp bands of the superstructure overlap at the node.
In this configuration, when the sample is aligned with the node and horizontally polarized light is used, a strong suppression of spectral weight is observed.
This suppression is also seen when the sample is aligned with the antinode under linearly polarized light, either vertical or horizontal, indicating that matrix element effects are responsible.
The exact origin of this polarization dependence at the node remains unclear, though it could be related to hybridization between bands crossing at this point.
Further investigation is required to understand the underlying mechanism.
In the next section, I will provide a comparative analysis with existing data on overdoped \ce{Bi}2223, contextualizing the findings.

\begin{figure}[h!]
	\centering
	\includegraphics[width=0.5\linewidth]{images/bi2223/FS_hybrid_zoom}
	\caption{The figure shows a region of the Fermi surface around the node, where multiple umklapp bands overlap with the main arc. A significant reduction in spectral weight is observed, which varies depending on the polarization of the incoming probe.}
	\label{fig:fs_hybrid}
\end{figure}

\section{Conclusion}

In summary, the preliminary results presented in this chapter offer valuable insights into the electronic structure of optimally doped (OP) \ce{Bi}2223.
The band structure was successfully mapped, revealing strong similarities to the bi-layer \ce{Bi}2212 compound.
The primary differences in the Fermi surface arise from the smaller anti-bonding/bonding splitting of the Fermi arc in \ce{Bi}2223, which was more pronounced in overdoped (OD) samples \cite{luo_electronic_2023}.
In OP \ce{Bi}2223, this splitting manifests as a slight broadening near the anti-node, similar to what was observed in the OD case, although more detailed analysis is required to fully understand this feature.

Another significant difference is the presence of a third band/Fermi sheet, likely originating from the inner \ce{CuO2} layer of the unit cell.
This band displays a similar arc-shaped dispersion as the main band and its intensity appears to be dependent on matrix element effects, with its spectral weight showing variation under different experimental conditions.
The dispersion of this third band resembles that of the main band near the node, but with an increasingly pronounced kink closer to the anti-node.

A typical d-wave symmetry was observed for both the main and side Fermi arcs, with the gap closing at the nodal point and opening progressively towards the anti-node.
Preliminary analysis suggests that the gap amplitude of the main arc, which closely resembles that of \ce{Bi}2212, is smaller than in the bi-layer compound.
In contrast, the additional arc, attributed to the inner \ce{CuO2} layer, exhibits a larger gap than both the main arc in \ce{Bi}2223 and the arc in \ce{Bi}2212.
This observation aligns with the composite model, where the inner layer is believed to contribute to pairing strength, thus showing a larger gap, while the outer layers govern phase stiffness.
A more careful and quantitative analysis on the gap size along the Fermi arc has to be made, to observe any potential deviations from the d-wave symmetry as observed in the OD compound \cite{luo_electronic_2023}.

Moreover, the absence of the typical peak-dip-hump structure in the quasiparticle spectrum of \ce{Bi}2223, compared to \ce{Bi}2212, highlights another key distinction between the bi- and tri-layer systems, though further analysis is needed to clarify the implications of this difference.

Additionally, a polarization-dependent loss of spectral weight at the node was observed, possibly linked to band hybridization where the umklapp bands overlap.
The precise mechanism behind this effect remains unclear and requires further investigation.

Overall, this preliminary analysis reveals several intriguing observations that can contribute to the understanding of OP \ce{Bi}2223, when properly evaluated and interpreted.
A more detailed analysis of the collected data, particularly in comparison to OD \ce{Bi}2223, is needed.
The presented data represents only a portion of the measurements obtained during the beamtime, which includes multiple samples, different alignments, probe polarizations, and photon energy dependencies.
A comprehensive study of the complete datasets is necessary to properly characterize the material.

Continuing on the project, further studies are encouraged, particularly with the availability of commercially sourced samples, though the quality of these samples for ARPES remains to be evaluated.
Future experiments, potentially within the LACUS facility, could include temperature-dependent scans across the superconducting transition and the pseudogap phase, polarization dependent scans to elucidate the potential hybridization at the node, or the first time-resolved measurements on \ce{Bi}2223.