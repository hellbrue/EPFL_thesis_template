\chapter{High resolution trARPES in the VUV regime}

The study of strongly correlated quantum matter is a well-established field, with the discovery of high-temperature superconductors dating back more than 30 years \cite{bednorz_possible_1986,anderson_resonating_1987}.
Since then, the field has evolved significantly, with the advent of ultrafast spectroscopies being one of the most notable advancements.
Historically, research has focused predominantly on ground-state properties, while out-of-equilibrium studies were considered too messy for these already intricate systems.
However, a paradigm shift has taken place, largely due to several groundbreaking studies \cite{eesley_relaxation_1990, han_femtosecond_1990, giannetti_revealing_2011, smallwood_tracking_2012}, positioning ultrafast, out-of-equilibrium investigations as powerful tools in the study of correlated quantum systems \cite{orenstein_ultrafast_2012, maiuri_ultrafast_2020, boschini_time-resolved_2024, giannetti_ultrafast_2016, lloyd-hughes_2021_2021}.

These out-of-equilibrium techniques not only offer insight into the formation of phases and pairings relevant to ground-state phenomena but also reveal metastable excited states.
Moreover, they provide a means of controlling material properties through light-induced excitation.
The previous chapters on metastable states in \ce{Bi}2212 and \ce{TaTe2} exemplify this potential.

In the history of physics, many groundbreaking discoveries have been enabled by advancements in measurement technologies.
As tools improve, new features are revealed, expanding our understanding of complex phenomena.
Therefore, the development and refinement of experimental tools is a vital part of advancing the study of strongly correlated quantum systems, and of science as a whole.
Ultrafast techniques, in particular, have seen substantial progress, especially in the development of ultrafast light sources \cite{keller_recent_2003}.
From the early days of maintenance-intensive dye lasers, to the introduction of few-femtosecond titanium-doped sapphire (Ti:Sa) lasers, and now the more commonly used ultrafast fiber-based turn-key solutions, light sources have become brighter, more stable, and user-friendly.
Equally important are advances in detection schemes, vacuum systems, and sample quality, all of which contribute to the overall performance of these experimental setups.
The challenge lies in integrating these advancements into cohesive tools that improve the capabilities of ultrafast spectroscopic techniques.

In this chapter, I will introduce a technical development in time-resolved angle-resolved photoemission spectroscopy (trARPES), focusing on the expansion of the technique into the vacuum ultraviolet (VUV) regime at high repetition rates and high energy resolution, while still allowing for the exploration of the time domain with reasonably high detail.
The technique and instrumentation have been previously published \cite{hellbruck_high-resolution_2024}, but here I will expand on the rationale behind this particular trARPES setup, discussing its advantages, limitations, and its role within the LACUS facility, as well as ongoing improvements.

\section{The Harmonium beamline}

\begin{figure}
	\centering
	\begin{subfigure}[b]{0.45\textwidth}
		\includegraphics[width=\textwidth]{images/vuv/HHG_spectrum_30}
		\caption{}
	\end{subfigure}
	\begin{subfigure}[b]{0.45\textwidth}
		\includegraphics[width=\textwidth]{images/vuv/HHG_spectrum_100}
		\caption{}
	\end{subfigure}
	\caption{(a) Spectrum of Harmonium beamline operated in the \qtyrange{28}{48}{\electronvolt} range. The higher harmonics are generated in Argon. (b) Spectrum of Harmonium with higher harmonics generated in Neon. Both spectra have been taken from \cite{ojeda_harmonium_2015}.}
	\label{fig:hhgspectrum}
\end{figure}


Before delving into the new capabilities, I will give a short overview on the setup that existed before the newest extensions.
The trARPES at LACUS used a higher hamonics based laser system, paired with a Phoibos 150 2D hemispherical analyzer and a 4-axis manipulator for experiments.
This beam line is called Harmonium and is based on the higher harmonic generation (HHG) effect \cite{arrell_harmonium_2017}.
Figure \ref{fig:beamline_cad} shows a top and isometric view of the full beamline with specification of the various light sources.
A Titanium Sapphire (Ti:Sa) oscillator ("Vitara" by Coherent) is used to generate IR (\qty{1.55}{\electronvolt} or \qty{800}{\nano\meter}) pulses at a repetition rate of \qty{80}{\mega\hertz}.
These pulses are amplified in one of the two existing amplifier systems (KMLabs Wyvern or Coherent Astrella) to a pulse energy of \qty{2}{\milli\joule} (Wyvern) or \qty{1}{\milli\joule} (Astrella) at a repetition rate of \qty{6}{\kilo\hertz}.
A high peak power is necessary to drive the highly non-linear process of HHG.
In such a process, an ultrashort light pulse is used to ionize a noble gas, typically Argon or Neon \cite{rudawski_high-flux_2013}.
After ionization, the electron is accelerated in the electric field of the IR driving pulse, gaining kinetic energy.
Some electrons recombine with the ionized parent atom,emitting an extreme ultraviolet (EUV) photon.
The EUV spectrum generated in such a process consists of multiple harmonics with an energy of an odd multiple of the seed pulse $(2n+1) \hbar\omega$, thereby forming an harmonic comb.

Mainly three different methods of gas delivery for HHG exist, with the generation process either happening in a gas cell, gas jet or a gas filled fiber \cite{paul_quasi-phase-matched_2003, heckl_high_2009, sudmeyer_femtosecond_2008, roser_131w220fs_2005}.
In all cases, a comb of harmonics is emitted from the gas with the spectrum of the harmonics depending on phase matching conditions, which can be adjusted to optimize for different energies.
Phase matching is achieved by minimizing the phase mismatch between the n-th harmonic and the fundamental
\begin{equation}
	\Delta k = k_n - k_1
\end{equation}
with the respective wavenumber k.
The ideal phase matching condition is reached for $\Delta k=0$.
For example, in different gases different phase matching conditions can be realized, allowing for better access to different spectral ranges.
Typically, Argon is used to create energies between \qtylist{20; 40}{\electronvolt} and Neon between \qtylist{40; 110}{\electronvolt} (see Fig. \ref{fig:hhgspectrum}) \cite{rudawski_high-flux_2013}.
Furthermore gas pressure, laser intensity and target position relative to the focus are all parameters that can be adjusted for the phase matching conditions and optimize the output of the desired harmonic.
The small cross section of the HHG process, due to the low probability of the electron recombining with the ionized gas atom, as wells as the non-perturbative nature of the process are the reason for the need of high peak fluences at high repetition rates, furthermore for PES studies, high repetition rates are desirable due to space charge effects, as can be seen in section \ref{sec:space_charge}.

From the many harmonics emitted simultaneously, one is selected for a trARPES measurement, with the help of a monochromator, which consists of two toroidal mirrors, a grating and slit module
After the EUV generation, a first toriodal mirrir is used to collimate the beam onto a grating, from which the harmonic comb is dispersed and after refocused by the second toriodal mirror onto an adjustable slit.
One grating is selected from a set of multiple different gratings each with different amounts of grooves per \unit{cm} (\unit{g\per\cm}), with the different gratings being used for different energy ranges.
The gratings are illuminated at grazing incidence, and are aligned with their grooves parallel to the propagation direction to reduce the pulse front tilt of the wavefront \cite{poletto_time-preserving_2010}.
A third The single harmonic probe beam is then focused by a second torroidal mirror onto the sample.

\begin{figure}
	\centering
	\begin{subfigure}[b]{0.45\textwidth}
		\includegraphics[width=\textwidth]{images/vuv/laser_systems_top}
		\caption{}
	\end{subfigure}
	\begin{subfigure}[b]{0.45\textwidth}
		\includegraphics[width=\textwidth]{images/vuv/beamline_iso}
		\caption{}
	\end{subfigure}
	\caption{(a) Top and (b) Iso view of the trARPES setup including both Harmonium and the VUV light source.}
	\label{fig:beamline_cad}
\end{figure}


Part of the amplified IR beam is split before the HHG process and used to excite the sample.
After compensating for the beam path of the probe pulse a telescope consisting of two curved mirrors is used to focus the pump beam onto the sample and adjust the spot size at the sample position.
Pump and probe beam co-propagate with a small angle between each other, due to geometrical constraints.
A $\beta$-bariumborate (BBO) crystal can be inserted into the pump beam path to double the pump energy from \qty{1.55}{\electronvolt} to \qty{3.1}{\electronvolt}.

The HHG beamline can produce a high flux of up to \qty{2.3e11}{\pps} (at \qty{36}{\electronvolt}) for harmonics between \qtyrange{20}{110}{\electronvolt}.
An energy resolution of $\approx$\qty{150}{\milli\electronvolt} can be achieved, which is mainly determined by the chosen harmonic and grating combination, as well as vacuum space charge effects, which will be discussed in greater detail in the next section.
The time resolution amounts to $<\qty{100}{\femto\second}$, as calculated from the FWHM of the temporal cross correlation, measured with the laser assisted photoemission (LAPE) effect \cite{ojeda_harmonium_2015}.
While the HHG pulses have an intrinsically short pulse duration, only limited by the spread in recombination times within the gas jet, the probe beam experiences a pulse front tilt due to the single grating monochromator.
The pulse front tilt leads to a increase in pulse duration over the spot size, which together with the pump pulse duration of $\approx$\qty{50}{\femto\second} results in the above time resolution \cite{arrell_harmonium_2017}.
The high time resolution is due to the monochromator design, using a time preserving single grating type monochromator \cite{poletto_time-preserving_2010}, which compensates the pulse front tilt, typically induced a reflective grating.
Time preserving monochromators and the effect of pulse front tilt are discussed in a following section.

\section{Vacuum space charge effects}
\label{sec:space_charge}

One of the main limiting factors for the energy resolution in high resolution or time resolved ARPES experiments are vacuum space charge effects.
This effect refers to the fact that electrons traveling in vacuum experience a repulsive force between each other due to coulomb interaction.
If a certain density of electrons is reached, the space charge effect becomes strong enough to significantly change the distribution of the electron cloud, which alters the energy, momentum and time information carried by the individual electrons, resulting in a reduced measurement resolutions and artifacts.

\begin{figure}[b!]
	\centering
	\includegraphics[width=0.6\linewidth]{images/vuv/space_charge_lit}
	\caption{The Figure shows the attainable sample current, corresponding to the amount of emitted photoelectrons, while maintaining a certain energy resolution. The dashed lines represent the highest possible sample currents for different repetition rate scenarios at a spot size of \qty{1}{\milli\meter} on the sample. The figure was taken from \cite{corder_ultrafast_2018} and markers represent different published ARPES setups.}
	\label{fig:spacechargelit}
\end{figure}

The resulting energy spread due to space charge effects can strongly differ depending on the kinetic energy of the electrons and the time duration of the light pulse generating the electron cloud.
Many studies have been carried out to quantify the influence of vacuum space charge on the energy resolution for light pulses of different energy and pulse duration, as well as the influence of an additional pump pulse\cite{corder_ultrafast_2018,plotzing_spin-resolved_2016,hellmann_vacuum_2009,graf_vacuum_2010,frietsch_high-order_2013}.
Generally speaking, electron bunches resulting from one-photon photoemission processes in the range of \qtyrange{5}{100}{\electronvolt} and pulse duration in the sub \unit{\pico\second} regime, a linear dependence of the space charge effects to the electron density can be found.
This linear dependence refers both to shifts and broadening of the measured photoemission spectra \cite{corder_ultrafast_2018,plotzing_spin-resolved_2016}
The emitted photoelectron density being $\rho=N/D$, with $N$ the amount of emitted electrons and $D$ the spot size of the light on the sample.
$N$ can be translated into a measurable current $I_{sample}$ at the sample, with which the broadening and shift in energy can be calculated by
\begin{equation}
	\Delta E_{s,b} = m_{s,b} \frac{I_{sample}}{e f_{rep} D}
	\label{eq:spacecharge}
\end{equation}
with $f_{rep}$ the light source repetition rate and $m_{s,b}$ an empirical scaling factor for the energy broadening (b) and shift (s) respectively.
The scaling factors have been determined as $m_b=$\qty{2.1e-6}{\electronvolt} and $m_s=$\qty{2.1e-6}{\electronvolt} \cite{plotzing_spin-resolved_2016}.
The space charge effects can also occur due to photoemission from the pump laser pulses.
Here the effect of a high fluence pump pulse results in the additional creating of photoelectrons due to multiphoton processes \cite{al-obaidi_ultrafast_2015}.
Further experiments and calculations were performed by Graf et al. \cite{graf_vacuum_2010} in the low energy (\qty{6}{\electronvolt}) and few \unit{\pico\second} regime.
Similarly to eq. \ref{eq:spacecharge} a linear dependence between the photoelectron density and the energy shift as well as the broadening was observed.
The energy shift $\Delta_E$ and the energy broadening $\Gamma_E$ can be expressed as

\noindent\begin{minipage}{.5\linewidth}
	\begin{equation}
		\Delta_E(N,D)=0.5\times \frac{N^{1.1}}{D^{1.1}}
	\end{equation}
\end{minipage}%
\begin{minipage}{.5\linewidth}
	\begin{equation}
		\Gamma_E(N,D)=13\times \frac{N^{0.7}}{D^{1.2}}
	\end{equation}
\end{minipage}

Keeping the number of electrons equal for a period of \qty{1}{\second}, it becomes quite clear that the repetition rate has a strong effect on energy spread, which can also be seen from the inverse proportionality in the equation above.
Dividing the N generated photoelectrons into 1 million bunches (corresponding to a \unit{\mega\hertz} system) greatly reduces the space charge effects when compared to only dividing them into 6000 bunches, as would be the case for a typical HHG system like Harmonium.
Therefore a transition of trARPES experiments, or ARPES in general, to include high repetition rate light sources is a fundamental step in improving the overall resolution of these systems.
Figure \ref{fig:spacechargelit}, taken from \cite{corder_ultrafast_2018}, shows the highest possible photocurrent depending on the energy resolution, when evaluating equation \ref{eq:spacecharge} for different repetition rates, at a beam spot size of \qty{1}{\milli\meter}.
Different markers are added from multiple light sources referenced in \cite{corder_ultrafast_2018}.
The figure illustrates very well the strong effect higher repetition rates have on the highest attainable number of photoelectrons for high energy resolutions.

Due to the space charge effects a majority of the EUV photons has to be removed in a trARPES experiment in the Harmonium beamline, which provides a further reason to move to high repetition rate systems.
Apart from the space charge effect which impacts the energy resolution, further limits regarding the time resolution have to be addressed.
For example, the time resolution of Harmonium is rather limited by the pump pulse duration, which could be shortened via pulse compression to the few \unit{\femto\second} regime \cite{nisoli_compression_1997}.
Further limits are imposed by the pulse front tilt introduced by the monochromator.
The latter will be described in greater detail in the following section.
	
\section{Time resolution limit due to pulse front tilt}

In any trARPES measurement the time resolution of the experiment is determined by the cross-correlation of pump and probe pulses.
Therefore it is important to keep the pulse duration of each pulse as short as possible.
It has already been mentioned that short pump pulses are relatively easy to achieve by pulse compression via white light generation and subsequent re-compression \cite{nisoli_compression_1997}.
Additionally the probe pulse in the case of a typical HHG beamline like Harmonium is intrinsically shorter than the initial seed pulse, typically also used as the optical pump.
In the case of a Gaussian pulse not experiencing any change in phase or frequency the time duration of such pulses can be expressed as

\begin{equation}
	\Delta\tau = \frac{2 \ln{2}}{\pi c}\frac{\lambda^2}{\Delta\lambda} = \frac{0.44}{c}\frac{\lambda^2}{\Delta\lambda}
\end{equation}

with the central wavelength $\lambda$ and the spectral width at half-height $\Delta\lambda$.
But in the case of HHG a full comb of harmonics is generated instead of a single pulse, which necessitates the use of a monochromator to select a single harmonic for the ARPES experiment.
Therefore the influence of a monochromator on the time duration has to be taken into account.

Since HHG beamlines provide a wide spectrum of harmonics, it is desirable to employ monochromators that can take advantage of this fact and whose output can be tuned in a wide frequency range.
Typically, such monochromators use reflection gratings, which contain a large spectral band when used in grazing incident.
The problem of gratings arises from the fact that they induce a tilt of the pulse wavefront, after a light pulse is diffracted off of them.
The pulse front tilt is a result of the light rays being diffracted off of different neighboring grooves, which creates a path delay equal to $m\lambda$, with $m$ corresponding to the order of diffraction.
The total path difference across the full spatial pulse distribution can be summed to $m\lambda N$, with N totaling the number of illuminated grooves.
Depending on the properties of the chosen harmonic and the employed grating this contribution can be quite substantial, especially when the goal is a time resolution in the femtosecond regime.
Therefore it is important to compensate or minimize this pulse front tilt, when designing a time preserving monochromator.

The compensation can be achieved by simply introducing a second grating, which exactly compensates the tilt introduced by the first grating \cite{villoresi_compensation_1999}.
The downside of this method is the added complexity to the overall monochromator, as well as the reduced transmission.
A different option is the use of a single grating, for which it can be shown that the total path difference $m\lambda N=\lambda^2/\Delta\lambda$ \cite{samson_j_a_vacuum_1998} and therefore the pulse duration at its half-width equates to

\begin{equation}
	\Delta\tau = \frac{0.5}{c}\frac{\lambda^2}{\Delta\lambda}
\end{equation}

which is only slightly higher than the Fourier-limited case \cite{poletto_time-preserving_2010,nugent-glandorf_laser-based_2002,poletto_time-compensated_2004,poletto_time-delay_2006}.
A downside of this version however is that the energy resolution is reduced, the shorter the pulse duration of the pulse is before monochromatization.
Nonetheless this option is used in the Harmonium beamline, with the goal of a high temporal resolution, while only sacrificing some energy resolution.

\section{High repetition rate VUV trARPES}
\label{sec:high_rep_vuv}

Recent advancements in HHG and their extension into the multi-\unit{\mega\hertz} regime \cite{mills_xuv_2012,hadrich_high_2014,pronin_high-power_2015,saraceno_toward_2015,hadrich_single-pass_2016,carstens_high-harmonic_2016,zhao_efficient_2018} lead to their adaptation in trARPES systems.
Since then new HHG based trARPES systems have emerged at high repetition rates of up to \qty{88}{\mega\hertz} \cite{corder_ultrafast_2018,mills_cavity-enhanced_2019}.
Such systems are based on cavity enhanced seed pulses utilizing resonators, with which it is possible to operate at higher repetition rates, despite the lower peak power and still maintain a high photon flux at high energies.
The clear upside of higher repetition rate HHG systems is strongly increasing the statistics collected within the same measurement time, as well as reducing space charge effects, as previously stated.
On the other hand two downsides emerge with this approach.
One is the higher technical difficulty to continuously operate the system compared to a simple gas jet or fiber based HHG system.
But while technical challenges can be addressed, a fundamental challenge emerges when including the high repetition rate pump pulses into the picture.

Here, the relevant quantity for out of equilibrium phenomena is the energy density in \unit{\milli\joule\per\centi\meter^2}.
Keeping this parameter constant, it is obvious that the average power increases by 5 orders of magnitude when moving from \qty{1}{\kilo\hertz} to \qty{100}{\mega\hertz} (assuming all other beam parameters remain the same), at which point thermalization becomes the limiting factor.
Strongly increasing the average power is necessarily accompanied with a higher thermal load on the sample, and heat dissipation from pulse to pulse becomes a problem, leading to a temperature build up and accelerated sample degradation.
Thermalization puts a hard limitation on the use case of these high repetition trARPES setups.
Either by limiting the amount of samples that can be measured at high average powers, or by the necessity to reduce the peak power per pulse to a degree at which the out of equilibrium effect may not be observable anymore.
Depending on the use case, an optimal compromise lays in the reduction of the repetition rate, but still operate at high enough rates, so that a high amount of statistics can still be acquired.
The new addition to trARPES setup at LACUS was planned to operate at repetition rates between \qtyrange{0.5}{2}{\mega\hertz}, allowing for high statistics, reasonably high pump fluences to explore the high perturbation regime and reduced space charge effects.

Another challenge in trARPES in particular lies in the accessible spectral range for the employed probes.
As previously mentioned, HHG can easily access the spectral range between \qtyrange{15}{110}{\electronvolt} \cite{chini_coherent_2014,weissenbilder_how_2022}.
Whereas the range below \qty{7}{\electronvolt} can be accessed by frequency conversion via sum frequency generation in nonlinear crystals like BBO and KBBF, even at high repetition rates.
This leaves a gap in the spectral range between \qtyrange{7}{15}{\electronvolt}.
A way to access that range so far in trARPES has been performing third harmonic generation (THG) in a crystal converting IR pulse with an energy around \qty{1.2}{\electronvolt} to UV pulses of \qty{3.6}{\electronvolt}.
These are then again frequency tripled in a Xenon filled gas cell, creating the 9th harmonic, with an energy of usually around \qty{11}{\electronvolt}.
This leaves the problem of only offering small tunability depending on the IR input pulse energy, and a resulting gap within the \qtyrange{7}{20}{\electronvolt} range.
But information from ARPES, like any other spectroscopic technique, heavily depends on the employed probe energy due to the dipole matrix element, making it very desirable to tune the probe energy.
Additionally, the tunability of photon energy allows for selection of different momentum regions. 
The new VUV light source can address this problem and close this gap significantly, by allowing for tunability in the range from \qtyrange{7}{10.8}{\electronvolt} with a \qty{1.2}{\electronvolt} spacing.
This is possible by using a highly cascaded harmonic generation process (HCHG) instead of a double THG process \cite{couch_ultrafast_2020}.

\begin{table}[t]
	\centering
	\begin{tabular}{ *{4}{c} }
		\hline
		Rep. rate (\unit{\kilo\hertz})		& Osc. Power (\unit{\milli\watt}) 	& Booster Current (\unit{\ampere}) 		 & Booster Power (\unit{\milli\watt})	\\ \hline\hline
		1000								& 250							  	& 2.69								  	 & 1000									\\ \hline
		Rod Current (\unit{\ampere}) 		& Rod Power (\unit{\watt})			& IR Input VUV (\unit{\milli\watt})		 & Green Input VUV (\unit{\milli\watt})	\\ \hline\hline
		8.17								& 12					  			& 1600								   	 & 1600									\\ \hline
	\end{tabular}
	\caption{Table of laser parameters in every day use. The repetition rate can be set in the VUV laser software. Oscillator output power, and diode currents for both the booster and rod amplifier can then be read out from the software. The average power at different positions (for booster, rod and the VUV input) have to be measured manually with the help of a powermeter.}
	\label{tab:laser_param}
\end{table}

The HCHG process requires two driving pulses, one pulse being the second harmonic of the other.
In our case a fundamental light pulse with a wavelength of \qty{1030}{\nano\meter} is used, which is a standard output wavelength of an \ce{Yb}-doped fiber-based oscillators.
The oscillator ("YLMO" by Menlo Systems) provides pulses at a repetition rate of \qty{50}{\mega\hertz}, $<$\qty{80}{\femto\second} pulse duration and with a pulse energy of $>$\qty{40}{\nano\joule}.
In order to drive the HCHG process higher pulse energies are necessary, which are reached in two subsequent amplification steps.
The oscillator output is monitored within the laser software and a typical average power of \qtyrange{220}{260}{\milli\watt} can be observed.
After the pulses leave the fiber oscillator they pass through a first stretcher, increasing the pulse duration, allowing for better amplification via chirped pulse amplification (CPA) \cite{strickland_compression_1985,maine_generation_1988,strickland_chirped_2021}.
CPA allows pulses to be amplified without damaging the gain medium, by distributing the pulse energy in a greater temporal range. Additionally, an electro-optic pulse picker reduces the repetition to \qtyrange{0.5}{2}{\mega\hertz}, depending on the needs of the trARPES experiment.
The pulses are the pre-amplified by the fiber based booster amplifier from \unit{\nano\joule} to a few \unit{\micro\joule}, at a boster diode current of \qty{2.69}{\ampere}.

A second stretcher is used to further increase the pulse duration of the IR pulses with the help of a transmission grating.
Typically an average power of \qtyrange{950}{1050}{\milli\watt} can be measured, corresponding to a pulse power of \qtyrange{0.95}{1.05}{\micro\joule}.
A fiber based rod-amplifier ("AeroGain" by NKT Photonics) then increases the pulse energies from few \unit{\micro\joule} to $>$\qty{10}{\micro\joule}.
The amplification is done by coupling the pre-amplified pulses as a seed into a single-mode gain fiber, which is pumped by a counter-propagating continuous-wave (CW) diode at \qty{976}{\nano\meter} with current of \qty{8.17}{\ampere}.
The amplified pulses are then re-compressed by a pair of gratings to \qty{200}{\femto\second} at up to \qty{12}{\micro\joule}.
Table \ref{tab:laser_param} contains an overview of the laser parameters, at various stages of amplification.

\begin{figure}
	\centering
	\includegraphics[width=0.7\linewidth]{images/vuv/vuv_laser}
	\caption{Picture of the open VUV laser system containing the various modules.}
	\label{fig:vuvlaser}
\end{figure}


Approximately \qty{5}{\micro\joule} are used in the HCHG process in a typical experiment, with the rest being available for optical pumping.
The \qty{5}{\micro\joule} are split into two beams, with one arm containing a BBO crystal,that creates the second harmonic at \qty{515}{\nano\meter}.
The second harmonic beam is passing over a delay stage before the two arms are overlapped again and co-propagate to the fiber.
This delay stage ensures that fundamental and second harmonic pulses arrive in the fiber at the same time, which is necessary to create the third harmonic and start the HCHG process.
A waveplate - polarizing prism pair is used to adjust the relative power between the two arms.
Typically the highest VUV output is achieved by having an equal amount of average power for the fundamental and second harmonic before entering the fiber.
Additionally a waveplate in the green arm ensures that both beams a co-linearly polarized before entering the fiber.
While the cascaded process has also been demonstrated with capillary waveguides with large diameters ($>$\qty{100}{\micro\meter}) \cite{misoguti_generation_2001}, those setups were based on Ti:Sa oscillators, working with much shorter pulse durations and higher pulse energies \cite{misoguti_generation_2001,durfee_phase_2002,misoguti_nonlinear_2005}.
The light source here uses negative-curvature hollow-core photonic crystal fibers (HCPCF) of smaller diameters (\qty{50}{\micro\meter}), which enables the use of the fiber laser with longer pulse durations \cite{couch_ultrafast_2020}.
HCPCF are fibers with a complex structure, using photonic crystals to strongly localize the light within the photonic band gap of the material. \cite{kolyadin_negative_2015, wei_negative_2017}
To further confine the light, the fiber is structured in a way, that core boundary shows a negative curvature, hence the term negative-curvature HCPCF \cite{wei_negative_2017}.
Figure \ref{fig:hchgsketch} (a) (taken from \cite{couch_ultrafast_2020}) shows a sketch of the generation process together with a view of the structured face of the HCPCF.

The HCPCF is filled with xenon gas at a pressure of $\approx$\qty{100}{\kilo\pascal} (measured from vacuum) in which the third harmonic is created in a four-wave-mixing process, combining two photons of the second harmonic and creating one photon of the fundamental and third harmonic.
Once the third harmonic is generated it can combine with the fundamental or second harmonic in additional four-wave mixing processes, successively creating the higher harmonics up to the 15th order (see Fig. \ref{fig:hchgsketch} (b)).
For the generation of the fourth harmonic onward multiple options of combining photons arise to create the respective harmonics.
For example the 7th harmonic $\omega_7$ can be created by either:
\begin{equation}
\begin{aligned}
	\omega_7 &= \omega_6 + \omega_2 - \omega_1 \\
	\omega_7 &= \omega_5 + \omega_3 - \omega_1 \\
	\omega_7 &= \omega_4 + \omega_4 - \omega_1
\end{aligned}
\label{eq:vuv_pathway}
\end{equation}
With further examples being listed in the report by Couch et al. \cite{couch_ultrafast_2020}.

For a harmonic to be created a phase matching conditions has to be first fulfilled in each case.
As an example the phase mismatch for a four-wave-mixing process for the third harmonic can be calculated for the case of a gas filled hollow core fiber by
\begin{equation}
	\Delta k = k_1 + k_3 - 2k_2 = 2\pi N \left( \frac{\delta_3}{\lambda_3} + \frac{\delta_1}{\lambda_1} - \frac{2\delta_2}{\lambda_2}\right) - \frac{u}{4\pi a^2} (\lambda_3 + \lambda_1 -2\lambda_2)
	\label{eq:phasematching_vuv}
\end{equation}
with the first term representing the contribution which depends on the xenon pressure in the HCPCF, due to the pressure dependent refractive index, and the second term stemming from the confinement in the waveguide.
The refractive index of noble gases is close to 1 and can be described by the simplified Sellmeier expression \cite{bideau-mehu_measurement_1981}
\begin{equation}
	n-1\simeq\frac{Ne^2}{8\pi^2\epsilon_0mc^2}\sum_{i}^{}\frac{f_i}{\lambda_i^{-2}-\lambda^{-2}}
\end{equation}
Equation \ref{eq:phasematching_vuv} can be generalized for any harmonic, depending on the creation pathway mentioned in equation \ref{eq:vuv_pathway}, with $\lambda_n$ representing the wavelength of the nth harmonic, $k_n$ the corresponding wavevector, N the number density of the gas (in this case xenon), $\delta_n$ the wavelength dependent gas-dispersion \cite{bideau-mehu_measurement_1981}, u a mode dependent parameter \cite{couch_ultrafast_2020,durfee_iii_guided-wave_1999}) and a the waveguide diameter (here \qty{50}{\micro\meter}). 
The phase mismatch can be minimized for each harmonic by a slightly varied xenon gas pressure.
The creation of the third harmonic requiring the highest pressure, with a slightly lower needed pressure for each subsequent higher harmonic.
In order to fulfill the condition for all harmonics simultaneously, a pressure of typically \qtyrange{100}{125}{\kilo\pascal} is applied to one side of the fiber, with a resulting pressure gradient existing over the length of the fiber.
To confirm the $\chi^3$ dependent nature of the HCHG process Couch et al. performed numerical simulations using nonlinear Schrödinger equations within the PyNLO package, calculating the output flux of the harmonics \cite{couch_ultrafast_2020,hult_fourth-order_2007,ycas_g_pynlopynlo_2024}.
The simulations matched roughly the measured output, which is consistent with a $\chi^3$ dependent nonlinear process.

\begin{figure}
	\centering
	\includegraphics[width=0.7\linewidth]{images/vuv/hchg_sketch}
	\caption{(a) Sketch of the VUV system. IR pulses are provided by an \ce{Yb} fiber oscillator paired with fiber amplifier. A BBO generates the second harmonic from fundamental IR pulse, and both are focused into the Xenon filled HCPCF, in which the higher harmonics are generated in a HCHG process. A monochromator then selects the desired harmonic. (b) Scheme of possible four-wave mixing pathways creating the harmonics up to the 9th order in a HCHG process. Taken from \cite{couch_ultrafast_2020}.}
	\label{fig:hchgsketch}
\end{figure}


Couch et al. \cite{couch_ultrafast_2020} have also reported that high resolution spectra of the higher harmonics have a higher bandwidth full-width at half-maximum (FWHM) than the driving pulse (\qty{40}{\milli\electronvolt} instead of \qty{8}{\milli\electronvolt}), indicating that the HCHG process creates shorter pulse duration than the one of the driving pulse, similar to HHG \cite{gagnon_soft_2007}.
In our paper \cite{hellbruck_high-resolution_2024} we found that the time resolution measured in the ARPES experiment is quite longer than the one of the driving pulse, but found that the resolution and with it the VUV pulse duration is limited by the pulse front tilt of the monochromator grating, suggesting that the pulse duration is indeed shorter than the one of the driving pulse.
Measurements with a time compensated monochromator could help identifying the time duration of the pulse, at the output of the fiber.

Lastly, a big upside of this laser system lies in its reliability and ease of use.
The VUV laser uses a fiber based oscillator and two fiber based amplifiers, this does not only make the source more compact, but also significantly increase its stability.
For example it is not necessary to realign the oscillator at all, and the amplifier only needs small adjustments every few months.
This greatly reduces the time spent re-aligning the system, which can be used to focus on other parts of the experiment.
Furthermore, the implementation of the fibers in cartridges without adjustability also makes the VUV generation process quite stable.
The photon flux can be re-optimized with only a few touch-ups after a days of continuous usage.

Therefore the new light source enables trARPES measurements in an energy range previously not accessible with the HHG beamline.
It provides higher energy resolution than Harmonium while compromising some time resolution, complementing the characteristics of the HHG beamline.
The higher repetition rates allow for faster acquisition for the same amount of statistics, while still allowing for high pump fluence measurements.
But the light source is most suitable to explore the low perturbation regime due to the high repetition rates, something that was previously very challenging with Harmonium.

\section{Paper and Supplementary}

While the previous sections introduced the capabilities of each instrument this chapter will focus on the implementation of the new light source within the previously existing lab space and the full characterization of the light source.
A paper on the fully implemented trARPES setup, functioning with both light sources, containing a full characterization of the VUV light source has already been published.
This section will therefore contain the paper to address the characterization and expand on the specifics of the implementation.
Furthermore the section will contain supplementary information to the paper.

trARPES is a versatile technique, which is also highly dependent on the light source used due to the matrix element.
The previous section already delves into the various different options for light sources and the desired parameters for trARPES.
There I already argued about why the VUV laser source was chosen and its complementary properties compared to the Harmonium beamline.
Both light sources combined span a big range of desired options for trARPES, but in order to truly exploit this characteristic it is necessary to operate them interchangeably. 
The interoperability of the two beamlines was one of the challenges, when planning and constructing the new setup.
With the priorities being the ability to switch between the two beamlines within a minute, while avoiding downtime and additional time for realignments, as well as ensuring no modifications to the existing Harmonium beamline.
In order to fulfill these goals, a new vacuum chamber was inserted into the HHG beamline, with the purpose hosting insertable optics that direct the VUV pump and probe beams to the sample.
These optics consist of a flat IR mirror for the pump beam, and two \ce{Al} mirrors with \ce{MgF2} coating for focusing and redirecting the probe beam.
All three optics are mounted on individual, motorized stages, that can bring the optics in position for operation of the VUV beamline or retract them in order to use Harmonium.
This allows switching between the two light sources within a minute, without breaking vacuum or additional realignment beyond the usual use case.

Another challenge is posed by the xenon gas, which leads to a pressure of \qty{1e-4}{\milli\bar} in the monochromator.
It is important to reduce this pressure in the vacuum parts closer to the ARPES chamber in order to operate in the low \qty{1e-10}{\milli\bar} range and ensure low sample degradation.
For this purpose multiple differential pumping stages have been introduced, including a turbo pump between the recombination chamber and the laser, a turbo pump on the recombination chamber itself and two ion pumps, one of which is a ion and non-evaporative getter (NEG) pump-combination, directly attached to the beamline tupe connecting the recombination chamber with the ARPES chamber.
Additionally a gate valve can be inserted in front of the recombination chamber to completely separate the two vacuum systems.
The gate valve is equipped with \ce{MgF2} window which can transmit the VUV light, which allows for measurements at lower base pressures, due to the ability of completely separating the high pressure laser system from the rest.
Operation with a closed valve will lead to a reduction of the intensity of about 50\%, and is therefore rarely used.

The last integral part of the VUV trARPES setup is the pump beamline which contains various parts.
First, a motorized attenuator with thin film plates (TFP) is situated directly after a the exit port of residual infrared beam.
With it the fluenced used for creating the out of equilibrium excitation can be directly controlled by the measurement software, allowing for direct fluence dependent measurements.
The attenuator is used in a way that the beam reflected by the two TFPs is used for the pumping, ensuring the shortest possible pulse duration, while the transmitted beam is dumped on a beamblock.
After passing the attenuator, a pair of two 2-inch mirrors is used to compensate for the longer probe beam path.
The two 2-inch optics were chosen to allow for multiple bounces on each mirror, keeping the compensation as compact as possible.
A pickup mirror picks the infrared beam up from the beam compensation and directs it to a motorized delay stage, after which a focusing telescope is passed.
It consists of two convex lenses, that create a focal spot size of \qty{390}{\micro\meter} x \qty{360}{\micro\meter} on the sample.
The second lens is positioned on a \unit{\micro\meter} stage for fine adjustment of the spot size.
A BBO crystal can be inserted  close to the focal point of the first lens, enabling pumping of the sample with \qty{515}{\nano\meter} light.
If the second harmonic is used for pumping the following IR optics have to be switched for green light optimized optics and ensure that any residual IR light is removed from the green second harmonic pump beam.
The pump laser beam enters the recombination vacuum chamber through a window port, parallel to the probe beam, after which it can be directed to the sample with an insertable mirror.
Fine adjustment of the pump beam position is done with the help of a piezo-mirror mount before entering the recombination chamber.

Within the VUV probe arm a diagnostic chamber is situated, containing a photodiode which is used to measure the photon flux of the selected VUV beam after leaving the monochromator.
In addition this diagnostic chamber can be used to measure the full VUV spectrum which is generated in the HCPCF, by step wise rotation of the monochromator's grating and measuring the flux at each step.
Figure \ref{fig:vuv_spectrum} shows the output of the VUV fiber, covering the four harmonics that are able to pass the monochromator with high intensity.
Further information on the harmonics, the spectral shape and relative intensities for all monochromator exit slit combinations can be found in the already published paper, contained in this chapter.

The paper attempts a comparison two other complementary light sources as well as fixed wavelength VUV sources.
A more detailed explanation of the light source is then followed, with a full description of the monochromator used and the working principle of a normal incident monochromator with is advantages and disadvantages.
Additionally, a short description of the integration into the trARPES setup can be found.
After the description of the light source and its integration two sections about the achieved energy and time resolution fully characterize the trARPES setup with the VUV light source.
For the energy resolutions measurements on polychristalline gold have been performed for all combinations of the four harmonics (\qtylist{7.2;8.4;9.6;10.8}{\electronvolt}) and monochromator exit slits (\qtylist{500;100;32}{\micro\meter}), to determine the total energy resolution.
The same measurements have been performed on a Helium lamp to extract the linewidths of the harmonics in each case.
Additional measurements on \ce{Bi2Se3} and \ce{Au(111)} have been performed to demonstrate the energy and momentum resolution by resolving the Rashba split in gold and Dirac-cone in \ce{Bi2Se3}.
Two measurements on \ce{Bi2Se3} and \ce{TaTe2} have been performed to analyze the time resolution of the light source at \qty{10.8}{\electronvolt} at two monochromator exit slits (\qtylist{500;100}{\micro\meter}).
At the end of the paper a conclusion is made, putting the achieved performance of the system in context with literature and offer possible further ways to improve the system.

\begin{figure}[h!]
	\centering
	\includegraphics[width=0.7\linewidth]{images/vuv/VUV_spectrum_pp_w7}
	\caption{Spectrum of the monochromator output as measured on the photodiode of the diagnostics chamber situated behind the monochromator's exit slit. The four selectable harmonics ($6\omega$, $7\omega$, $8\omega$ \& $9\omega$) are shown at their respective central photon energy and their spectral widths. The grating is rotated during the scan and a photocurrent is measured on the diode. A responsivity curve used to calculate the photon flux in photons/s. Refer to \cite{hellbruck_high-resolution_2024} for the photonflux of the different harmonics and for different monochromator exit slit settings.}
	\label{fig:vuv_spectrum}
\end{figure}

\includepdf[pages=-]{main/vuv_paper.pdf}

\section{Ongoing Developments}

Since the full implementation and characterization of the VUV laser system further developments on the system have been introduced.
A main change has been the switch to a new manipulator.
The old manipulator had four adjustable axes (3 Cartesian coordinates plus rotational axis), with access to a fifth (in plane rotation) with low accuracy by rotating the sample with the help of a wobble stick.
In this setup only the rotational axis was motorized.
The new manipulator has access to all six degrees of freedom, with all of them being motorized, allowing for greater control and more complex scans.

The main advantages of this manipulator is the access to the in plane rotational degree of freedom, allowing for it's adjustment while having a live view of the band structure.
Furthermore it solves a problem when taking Fermi surface maps.
Due to the sample geometry, the center of rotation is often not the point imaged by the electron analyzer.
This mean that the illuminated spot slightly changes under sample rotation of a Fermi surface scan, resulting in the possibility to image adjacent domains.
To compensate for this the $x$ and $y$ coordinate have to adjusted when changing from one angle to the next to image the same sample position.
This process was automated due to the availability of the additional motorized degrees of freedom.
Instead, with the motorization of the new manipulator more complex movements are possible, where the $x$ and $y$ coordinates can be dynamically adjusted during a Fermi surface scan.

Another big advantage of the new manipulator is a better temperature control.
With an open cycle cryostate the temperature can be lower down to \qty{4}{\kelvin}.
Intermediate temperatures can be reached by adjusting the flow rate of the used coolant.
Additionally a heater unit makes fine adjustments of the temperature possible.
This would allow for more precise temperature scan, for example when crossing phase transitions or the exploration of softening of modes etc.

\begin{figure}[h]
	\centering
	\begin{subfigure}[b]{0.46\textwidth}
		\includegraphics[width=\textwidth]{images/vuv/manip_old}
		\caption{}
	\end{subfigure}
	\begin{subfigure}[b]{0.46\textwidth}
		\includegraphics[width=\textwidth]{images/vuv/manip_new}
		\caption{}
	\end{subfigure}
	\caption{Pictures of the (a) old and (b) new manipulator.}
	\label{fig:manip_full}
\end{figure}

A further development concerns the polarization control of the VUV beamline.
While control over the pump polarization is easily implemented with a IR waveplate, the situation is more difficult for the probe.
It is in principal possible to rotate the polarization of both IR and green beam before entering the fiber and create harmonics with the corresponding polarization.
But due to a polarization dependent rejector it is only possible to use the monochromator with p-polarization.
Using the monochromator at s-polarization does not remove the residual IR and green from the seed, which leads to a degradation of the optics and intense stray light passing through the monochromator.
Therefore it is necessary to instead rotate the polarization of the VUV light after the monochromator.
An example for the importance of polarization control can be seen in ARPES measurements on \ce{Bi}2212 at low probe energies (\qty{7}{\electronvolt}), where a strong dependence of the photoelectron yield from the light polarization is found, which stems from a strong polarization dependent matrix element \cite{fanciulli_spin_2018}.
Similar tests with a probe energy of \qty{10.8}{\electronvolt} have resulted in photoemission spectra with low photoelectron yield and no resolvable band structure, when measured with s-polarized light.
The question naturally arises if a similar polarization dependent effect is at play or if the dipol matrix element results in a low photoelectron yield independent of the polarization.
The problem in rotating the polarization at these photon energies is the lack of readily available half-wave plates (HWP).
Due to the low transmission in most materials a HWP has to be specifically made for this use case (manufactured by Kogakugiken Corp.).
The initial implementation and first tests are ongoing and will hopefully provide full control over the polarization at \qty{10.8}{\electronvolt}, and allow the study of the matrix element in \ce{Bi}2212.

\begin{figure}[t]
	\centering
	\begin{subfigure}[b]{0.45\textwidth}
		\includegraphics[width=\textwidth]{images/vuv/head_old}
		\caption{}
	\end{subfigure}
	\begin{subfigure}[b]{0.45\textwidth}
		\includegraphics[width=\textwidth]{images/vuv/head_new}
		\caption{}
	\end{subfigure}
	\\
	\begin{subfigure}[b]{0.45\textwidth}
		\includegraphics[width=\textwidth]{images/vuv/sample_coordinates}
		\caption{}
	\end{subfigure}
	\caption{Pictures of the (a) old and (b) manipulator heads. (c) Shows a sketch of the 6 available motorized axis from the point of a sample plate reference system.}
	\label{fig:manip_head}
\end{figure}

The most recent development concern one of the strengths of the VUV light source, which is the ability to better study low perturbation dynamics.
As mentioned in section \ref{sec:high_rep_vuv}, the nature of the high repetition rates make it possible to faster acquire the needed statistics.
This enables the observation of less pronounced features, which is typically the case for features occurring only after a small perturbation to the system.
Adding this to the fact that high pump fluences are also problematic at high repetition rates due to the thermalization issues described earlier, it seems that the best use case for the light source is the low perturbation regime.
To further strengthen this ability a new detection scheme is implemented, which measures pumped and unpumped signals successively.
The differences between these images should, in principle, show the pump induced features more pronounced and reduce fluctuations that just occur over time.

\begin{figure}
	\centering
	\includegraphics[width=0.65\linewidth]{images/vuv/chopper_electric_scheme}
	\caption{Sketch of the om/off detection scheme. The scheme shows the electric connections necessary to synchronize the camera with laser and chopper wheel. As an example typical repetition rates of an experiment where chosen. The VUV laser provides light pulses at a repetition rate of \qty{1}{\mega\hertz}. The A frequency divide (F-DIV) /delay generator divide the reference signal to a \qty{1}{\kilo\hertz} signal in  accordance with the possible trigger rate of the camera and a second trigger signal at half the repetition rate for chopper wheel, creating the pump on/off images. A photodiode measures the chopped beam and can display its repetition rate on an oscilloscope together with the chopper controller reference signal and the camera reference signal for comparison.}
	\label{fig:triggerscheme}
\end{figure}

This detection scheme is based on a fast acquisition camera, whose frame frequency can be adjusted up to \qty{1.5}{\kilo\hertz}.
The camera can be directly triggered by the laser trigger signal, or any similar reference signal.
In order to measure both on and off signals a chopper wheel is used, which is set to a frequency that matches the camera acquisition rate.
With the VUV laser set to \qty{1}{\mega\hertz}, both camera and chopper are set to \qty{1}{\kilo\hertz}.
This results in each acquired camera shot, reflecting the data of 1000 probe pulses, with pumped and unpumped shots following one after the other.
A more detailed sketch of the detection scheme is shown in Fig. \ref{fig:triggerscheme}.
The VUV laser is set to a repetition rate of \qty{1}{\mega\hertz}, with the trigger reference signal being send to a frequency divider (F-DIV) and Delay Generator.
The F-DIV can divide the electric input signal for multiple channels individually, while the delay generator is can be used to adjust the timing of the different signals.
For example, in a typical use case one channel is set to a division by a factor $1000$, resulting in a trigger rate of \qty{1}{\kilo\hertz}, which is send to the camera.
A second channel is set to a factor $2000$, which results in a \qty{500}{\hertz} trigger, which is sent to a chopper controller.
The chopper controller then synchronizes the chopper to turn at speed necessary to create pump on and off frames at the correct repetition rate.
A reference signal can be send from the chopper controller to the camera as a second trigger (AOL).
Additionally, an oscilloscope can be connected to visualize all three signals simultaneously.
The chopper controller reference can be attached to the oscilloscope instead of the camera (AOL), the chopped pump pulse is detected by a photodiode and its signal can be shown simultaneously on the oscilloscope.
On top of that, the camera has a reference out channel too, which can be shown on the oscilloscope to compare the delay between all signals.
These electric delays can be adjusted by the F-DIV/delay generator to make sure that the camera is triggered in sync with the chopper and that the chopper functions at a repetition rate corresponding to the laser repetition rate.


A typical acquisition scheme of a trARPES experiment looks the following:
\begin{itemize}[topsep=-0.5em]
	\setlength\itemsep{-0.5em}
	\item [$\bullet$] \qty{1000}{\hertz} repetition rate of each camera shot
	\item [$\bullet$] setting integration window for both on and off signal to \qty{200}{\milli\second}
	\item [$\bullet$] summing for \qty{1}{\second}, while sorting the pictures in corresponding on and off bin
	\item [$\bullet$] move stage to next delay position and repeat on and off picture
	\item [$\bullet$] after finishing delay list, start at first delay and repeat loop
\end{itemize}
The main challenge with the implementation of this detection scheme is accounting for all accumulated delays and detector dead times.
The delays can be investigated with the help of an oscilloscope and adjusted by a delay generator.
A major challenge in setting the delays correct was also accounting for the decay time of the phosphor screen at the end of the hemispherical analyzer.
It's decay time is in the \unit{\milli\second} range, therefore the start of each acquisition has to be carefully adjusted to avoid spillover signals from pump on to pump off frames.
Systematic testing of several electric delays has been done by only using the pump beam and observing the multiphoton photoemission signal, which should lead to the observation of a photoemission signal when the chopper lets the pump beam pass and an empty frame when not.
With this method an optimal acquisition window was found, and by additionally setting correct threshold setting of the camera any spillover can be avoided, while simultaneously avoiding a big reduction in signal.

Initial tests with this new detection scheme have been carried out with the Harmonium beamline on reference samples that have been shown to provide strong out of equilibrium signals like \ce{WSe2} \cite{puppin_excited-state_2022}.
While it could be shown that the the detection scheme has been correctly implemented, an improvement in signal quality has not been observed, which can be due to multiple reasons.
The correct implemented was shown by observing the excitonic state above $E_F$ only in the pump on frames.
But no signal improvement was observed when plotting the on/off frames compared to the only on frames.
A first reason for this can be, that the measurement was taken far out of equilibrium, where comparatively high signal is expected, but at the same time where pump induced artifacts might hinder the observation of nuanced features.
Second, no big changes to the band structure have been observed below the Fermi level, which is the region where signal improvements would be expected, while the region above Fermi containing the excitonic state has intrinsically no background and therefore shouldn't show improvements when plotting any on/off images.
The clear disadvantage is that the on/off detection scheme naturally reduces the statistics taken in the same time frame by a factor of 2.
Therefore improvements can only be expected if the noise levels of the traditional pump probe scheme are to high and prevent the observation of pump induced effects.
Further tests have to be carried out with samples that show changes in the low perturbation regime and in the occupied region of the band structure, to test the on/off scheme in a more appropriate use case.

\begin{figure}
	\centering
	\includegraphics[width=\textwidth]{images/vuv/on-off_bandmaps}
	\caption{The figure shows bandmaps and difference maps of the K-point of \ce{WSe2} measured with Harmonium using the pump on/off detection scheme. All maps are shown at a time delay of \qty{0}{\femto\second} between pump and probe.
	(a) shows the bandmap for pump on with the faint exciton state above $E_F$. (b) is the corresponding difference map of raw data and static background ($on-\overline{on}$). (c)\&(d) equivalent maps with pump off. (e) Difference map between on and off raw data.}
	\label{fig:on-offbandmaps}
\end{figure}
