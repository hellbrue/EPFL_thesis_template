\cleardoublepage
\chapter*{Reflections and Future Horizons}
\markboth{Reflections and Future Horizons}{Reflections and Future Horizons}
\addcontentsline{toc}{chapter}{Reflections and Future Horizons}

As I reach the end of this dissertation, I find myself reflecting on the impressions this journey has left on me, and perhaps on my own actions as well.
When I first arrived at EPFL, I began my research from a very naive perspective.
Thrown into a new environment during lockdowns and overwhelmed by a new research topic.
This is not to complain about the hardships of a PhD, but quite the contrary.
It was only after these first humbling lessons that I began to accept the things I didn’t know and stopped trying to prove what I did know to others.
Which in the end made me realize, there is more beauty in what we don't know.\\
In fact, this realization connects to a broader issue that has bothered me in recent years.
Our collective inability to say, "I don't know."
I am not sure whether this is a modern phenomenon or whether we are simply more exposed to it in a digital age, where it's essential to always have something to say.
Perhaps this tendency has found its way into general discourse, where we feel compelled to be experts on everything rather than embrace our ignorance and listen instead of speaking.\\

Why do I bring this up at this point?
While I beleive that, scientists are trained in environments of critical discourse, we too can fall into the trap of assuming that our expertise in one area extends into others, confusing intuition with knowledge.
It is precisely at the point where intuition and facts diverge that we must pay attention.
As a friend once put it, \textit{"The good, the bad, and the ugly."}
The good: Some of the best experiences during my PhD have come from engaging in discussions at conferences, learning about new ideas that might apply to my own projects, and receiving valuable feedback on my work.
However, I’ve also witnessed the opposite.
The bad: Moments when we ask questions without truly listening to the answers, or worse, when we approach a discussion with preformed opinions, seeking to prove others wrong rather than to find the truth collectively.
I am certainly guilty of this as well, perhabs we all are at some point, but I hope to learn to be more self-aware, and realize this in future exchanges.
Maybe it is a worthwhile exercise to accept our ignorance, as the only way of truly striving for discovery.\\
In the end, I believe I was fortunate to participate in many great discussions, sometimes even just as an observer.
These very interactions made me fall in love with the idea of science.\hfill\break

At the start of my PhD journey, after facing the initial frustrations of the seemingly endless literature on cuprate superconductors, I started to appreciate the complexity of physics.
From there, it has been a roller coaster of learning something new, only to realize that there’s even more I don’t know.
Especially learning about the possibility of manipulating matter with light has captivated me over the years, and I believe this path holds great promise for future materials research.
The idea that we can change material properties efficiently and at will opens doors for innovation, which could help overcome the technological challenges we will face.
To that end, ultrafast research is essential, not only for learning how to drive material properties but also for exploring the mechanisms of the materials themselves.\\
Here, I believe, the thesis has shown two great examples.
In the chapters \ref{ch:bi2212} and \ref{ch:tate2}, we learned about the existence of long lived metastable states, that exhibit different electronic properties from their equilibrium counterpart.
Not only were we able to observe the existence of these states, but each in case we were provided with information on the microscopic properties governing the materials.
Therefore, the field shows a unique possibility of combining fundamental and application oriented research.\\
However, sometimes it still feels like we're still in the beginnings of ultrafast science.
We often blast any material we can get our hands on with lasers just below the threshold of destruction, just to claim to have discovered a new way to efficiently alter the material.
While this may sound overly provocative, it highlights an important point in my opinion.
As with most advancements, the scientific process is often iterative, and it will take time to fully utilize the possibilities to control matter, but I believe we should occasionally take the time and step back to reflect on where we’re investing our time and resources.
Instead of brute-forcing progress, it might be more worthwhile to pause and carefully consider the next steps.
Perhaps the fast-paced nature of academia that requires the constant creation of output, makes it harder to take this more creative, but more time consuming approach, leading us to chase the next nearest idea.\hfill\break

But despite these challenges academia is facing, the past few decades have seen an incredible, and possibly unprecedented, amount of new discoveries.
For example, just around the time of writing this, both the Nobel Prizes in Physics and Chemistry were awarded for AI-driven research, emphasizing the profound impact this field will have.
Perhaps this innovation will have one of the most pronounced impacts for society as a whole.
But also the domain of controlling matter experienced important growth in recent years.
With improvements in laser technology, we are now able to drive materials more selectively.
Rather than simply using any available pump light, we can tune our driving source to be resonant with the material’s collective modes, vastly increasing efficiency and marking a shift towards a more targeted approach in ultrafast experiments.\\
Similarly, the rise of cavity systems coupled with complex condensed matter, offers a groundbreaking new approach, the prospect of manipulating material properties purely through vacuum fluctuations is an extraordinary goal.\\
Being able to observe the control of material properties via light myself, in the studies I’ve worked on has been a rewarding experience.
The idea of manipulating superconductors, switching them in and out of their superconducting phase, or even enhancing their critical temperature, is a topic that strongly resonates with me, as I hope truly transformative innovation will emerge from this research.\\

Finally, I believe science holds incredible power.
It drives innovation, uncovers truths, and inspires our imagination.
The scientific method, along with honest and healthy discourse, is maybe one of the greatest approaches humanity has created.
Seeing so many people around the world engaging with this approach fills me with optimism.
We should not forget these values and hold this approach dear to ourselves.
Hopefully we can take the opportunity and positively influence and inspire the people around us.
If we do so, I am confident that we can tackle the challenges ahead, even those as difficult and threatening as climate change, together as a society.


\vspace*{3cm}

\begin{raggedleft}
	Science may set limits to knowledge,\\
	but should not set limits to imagination\\
	--- Bertrand Russell\\
\end{raggedleft}