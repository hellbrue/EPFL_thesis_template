\cleardoublepage
\chapter*{Conclusion}
\markboth{Conclusion}{Conclusion}
\addcontentsline{toc}{chapter}{Conclusion}

At the end of this dissertation, I would like spell out my thoughts about these past four years and the work they entail.
The very first project I worked on during my thesis was introducing the new VUV laser beamline, which I discussed in detail in Chapter \ref{ch:vuv}.
While projects about instrumentation are often viewed as "unsexy", I believe they have a high value to a certain extent.
First of all, it was an immense project, where many different skills were needed, from electronics, to vacuum technologies, automation, photonics, CAD, and maybe mst importantly, planning the project and organizing it.
Being able to learn so many things in the beginning was highly valuable.
On top of that it taught me a lot about the system I was working with.
It turns out that you learn a lot about trARPES when you build a trARPES beamline.
But beyond all of this I was able to pair studying cuprates and wokring with the initial data, with being an engineer, which helped to reduce alot of frustrations, when one of the two hit a road block.

In the I, together with the ARPES team, succeeded in including the new beamline.
The first samples were taken and with the system being more and more reliable to use and workflows are starting to be established, it is the moment to utilize the new capabilities.
A lot of different projects ca be realized.
I believe two different topics should be addressed here.
First of all, many correlated systems show small features in terms of energy, here the high energy resolution of the system is invluable to study the time dynamics.
Around the world only few systems have the capability to resolve the ultrafast dynamics down to \qty{25}{\milli\electronvolt}, while being able to map the full FBZ.
This together with the soon implemented polarization control and the newly gained access to temperature control, will proof to be invaluable for investigating phase transitions or metastable states.
Having access everyday to such a system should be taken advantage of, with many interesting material system such as \ce{TaTe2} and \ce{Bi}2223 or similar.\\
Second is the direction of small perturbation dynamics.
Many trARPES measurements are based on using high fluence pump pulses, often out of necessity, as many effects simply need high fluence to be observed.
But of course many nuanced features will be destroyed at that point.
Here with new detection schemes. like pump on/pump off, it might be possible to observe dynamics resulting from small perturbations.
Here it will be necessary to look for other time resolved techniques that typically work in a lower fluence range, like transient reflectivity.
From there one might find possible candidates to be tested out in trARPES measurements.

\newpage

Another big project concerned the light induced Lifshitz transition in \ce{Bi}2212.
This project has been a part of my life for almost four years now and it taught me many lessons, to say the least.
First, starting a project on cuprates or high $T_c$ superconductors is overwhelming, to say the least.
But the amount of things you learn is second to none, it seems like with every day there is something new to unpack.
It is probably one of the reasons why it took so long to bring the data presented in chapter \ref{ch:bi2212} from \textit{"in preparation"} to a finalized draft.
Every time, we thought, we figure it out, we learned something new, which required attention, studying, new analysis, discussions etc.
And while tedious, it is what science should be about.
As we say in German: \textit{"Gut Ding will Weile haben"}.\\
I believe this project was incredibly fascinating.
Further studies, either with higher resolution trARPES, or on underdoped samples should be able to test the proposed ideas.
Making a underdoped sample suddenly superconducting just by irradiating it with light would be astonishing.
A whole plethora of possibilities exist here.\hfill\break

