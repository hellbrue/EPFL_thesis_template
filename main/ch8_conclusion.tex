\cleardoublepage
\chapter*{Reflections and Future Horizons}
\markboth{Reflections and Future Horizons}{Reflections and Future Horizons}
\addcontentsline{toc}{chapter}{Reflections and Future Horizons}

As I reach the conclusion of this dissertation, I find myself reflecting on the past four years and the work that shaped this journey.
The first project I undertook was introducing the new VUV laser beamline, as detailed in Chapter \ref{ch:vuv}.
While instrumentation projects are often seen as less \textit{"sexy"}, I’ve come to appreciate their unique value.
This project required and taught me many different skills, electronics, vacuum systems, photonics, CAD, automation, and perhaps most importantly, the ability to plan and manage it all \textit{(which I probably more often than not failed at)}.
Gaining such a broad range of experience early on was invaluable, and it deepened my understanding of the system itself.
It’s true: you learn a lot about trARPES when you build a trARPES beamline.\\
Beyond technical expertise, the experience allowed me to combine two roles, working as a scientist studying cuprates and analyzing data, while also functioning as an engineer.
When one aspect encountered a roadblock, the other helped balance frustrations.

Together with the ARPES team, we successfully completed the beamline, and now, with the first samples collected and the system becoming more reliable, it is the moment to utilize the new capabilities.
Many exciting projects are possible, but two directions stand out in my opinion.
First, many correlated systems have subtle energy features, and the beamline’s high energy resolution will be invaluable for studying their time dynamics.
There are few systems in the world capable of resolving ultrafast dynamics with a resolution down to \qty{25}{\milli\electronvolt}, while mapping the full Brillouin zone.
With the upcoming addition of polarization control and fine temperature adjustments, the system will provide insightful data for investigating phase transitions and metastable states.\\
The second direction involves exploring small perturbation dynamics.
High fluence pump pulses are often necessary in trARPES experiments to observe key effects, but they can also obscure finer details.
With new detection techniques, such as pump on/pump off, we might be able to capture dynamics from smaller, less disruptive perturbations.
Here it will be important to look to other time-resolved methods, such as transient reflectivity, which operate at lower fluences, to identify samples that can then be tested in trARPES.\hfill\break

Another major project involved the light-induced Lifshitz transition in \ce{Bi}2212, a project that has been part of my life for nearly four years, and it has taught me many lessons.
Starting a project on cuprates or high-$T_c$ superconductors can feel overwhelming.
However, the amount of knowledge you gain is immense, it feels like every day there’s something new to learn.
This is probably why it took so long to turn the data presented in Chapter \ref{ch:bi2212} from “in preparation” to a finished draft.
Each time we thought we had reached a conclusion, new discoveries would emerge, requiring further study, analysis, and discussion.
While this process can be demanding, it is also the essence of science.
As we say in German, \textit{"Gut Ding will Weile haben"} - good things take time.\\

This project has been truly captivating, and future studies, with higher resolution or underdoped samples, could further test the ideas we’ve developed.
The notion of making an underdoped sample superconducting by simply irradiating it with light remains an exciting possibility.
A whole plethora of possibilities exists here.\hfill\break

The work on \ce{Bi}2212 naturally led to a project on \ce{Bi}2223, which marked my first beamtime experience at the SLS PSI synchrotron.
While the long, overnight shifts can be exhausting, the experience provided new perspectives.
Initially, we had planned temperature-dependent scans, but due to a malfunctioning heater we were only able to work at base and room temperatures.
During beamtime, you realize the importance of sticking to a plan without becoming overly eager and chasing every potential new feature.
At the same time, it's crucial to remain open to exploring unexpected results.
Balancing this contradiction is challenging, and often, you won’t know if you succeeded until the end.\\
Despite my limited time with the \ce{Bi}2223 data, we’ve already made some fascinating observations that motivate further exploration.
With the ARPES machine at EPFL, we can now perform the temperature scans we originally envisioned without relying on synchrotrons.
Also time resolved measurements could turn out to be fruitful.
It would be very interesting to observe how the different layers, with different gap sizes and different effective dopings react to light irradiation.
Or if the interplay of phases differs from the ones observed in \ce{Bi}2212. \hfill\break


The final major project of my PhD focused on \ce{TaTe2}.
For me the project started with looking for samples with which I could test the new VUV beamline.
After we managed to measure the sample an characterize the time resolution of the new beamline, I was made aware of a mountain of data, taken by Michele Puppin and Alberto Crepaldi and co-workers, before I even arrived at EPFL.
Analyzing this data was a rewarding experience, uncovering such strong time dynamics is rare.
Restarting the collaboration allowed me to dive into a new class of materials and learn about charge density waves (CDWs), which was great opportunity due to the close connection of CDWs and superconductivity.\\
In this project, we found a range of intriguing phenomena.
From strong oscillations, rich ultrafast dynamics, to a metastable state.
Fourier analysis even revealed a previously unknown low-frequency, and potential amplitude mode, along with band-selective electron-phonon coupling.
The most rewarding part of this work was not just understanding these individual findings, but seeing how they fit together to form a cohesive story.\\
There’s still much more to explore in this project.
Temperature-dependent studies could offer valuable insights, particularly given how the lifetime of the metastable state in \ce{TaS2} depends on the temperature before light excitation.
It would be interesting to see if \ce{TaTe2} behaves similarly, and if so why this temperature dependence exists.
Exploring the time dynamics in a temperature range across the phase transition and investigating the amplitude mode further could also provide new, exciting information about their respective nature.\hfill\break

