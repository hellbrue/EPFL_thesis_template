\chapter{High resolution trARPES in the VUV regime}
\label{ch:vuv}

The study of strongly correlated quantum matter is a well-established field, with the discovery of high-temperature superconductors dating back more than 30 years \cite{bednorz_possible_1986,anderson_resonating_1987}.
Since then, the field has evolved significantly, with the advent of ultrafast spectroscopies being one of the most notable advancements.
Historically, research has focused predominantly on ground-state properties, while out-of-equilibrium studies were considered too messy for these already intricate systems.
However, a paradigm shift has taken place, largely due to several groundbreaking studies \cite{eesley_relaxation_1990, han_femtosecond_1990, giannetti_revealing_2011, smallwood_tracking_2012}, positioning ultrafast, out-of-equilibrium investigations as powerful tools in the study of correlated quantum systems \cite{orenstein_ultrafast_2012, maiuri_ultrafast_2020, boschini_time-resolved_2024, giannetti_ultrafast_2016, lloyd-hughes_2021_2021}.

These out-of-equilibrium techniques not only offer insight into the formation of phases and pairings relevant to ground-state phenomena but also reveal metastable excited states.
Moreover, they provide a means of controlling material properties through light-induced excitation.
The previous chapters on metastable states in \ce{Bi}2212 and \ce{TaTe2} exemplify this potential.

In the history of physics, many groundbreaking discoveries have been enabled by advancements in measurement technologies.
As tools improve, new features are revealed, expanding our understanding of complex phenomena.
Therefore, the development and refinement of experimental tools is a vital part of advancing the study of strongly correlated quantum systems, and of science as a whole.
Ultrafast techniques, in particular, have seen substantial progress, especially in the development of ultrafast light sources \cite{keller_recent_2003}.
From the early days of maintenance-intensive dye lasers, to the introduction of few-femtosecond titanium-doped sapphire (Ti:Sa) lasers, and now the more commonly used ultrafast fiber-based turn-key solutions, light sources have become brighter, more stable, and user-friendly.
Equally important are advances in detection schemes, vacuum systems, and sample quality, all of which contribute to the overall performance of these experimental setups.
The challenge lies in integrating these advancements into cohesive tools that improve the capabilities of ultrafast spectroscopic techniques.

In this chapter, I will introduce a technical development in time-resolved angle-resolved photoemission spectroscopy (trARPES), focusing on the expansion of the technique into the vacuum ultraviolet (VUV) regime at high repetition rates and high energy resolution, while still allowing for the exploration of the time domain with reasonably high detail.
The technique and instrumentation have been previously published \cite{hellbruck_high-resolution_2024}, but here I will expand on the rationale behind this particular trARPES setup, discussing its advantages, limitations, and its role within the LACUS facility, as well as ongoing improvements.

\section{The Harmonium beamline}

\begin{figure}
	\centering
	\begin{subfigure}[b]{0.45\textwidth}
		\includegraphics[width=\textwidth]{images/vuv/HHG_spectrum_30}
		\caption{}
	\end{subfigure}
	\begin{subfigure}[b]{0.45\textwidth}
		\includegraphics[width=\textwidth]{images/vuv/HHG_spectrum_100}
		\caption{}
	\end{subfigure}
	\caption{(a) Spectrum of the Harmonium beamline operating in the \qtyrange{28}{48}{\electronvolt} range, with higher harmonics generated in Argon. (b) Spectrum of Harmonium with higher harmonics generated in Neon. Both spectra are adapted from \cite{ojeda_harmonium_2015}.}
	\label{fig:hhgspectrum}
\end{figure}


Before delving into the new capabilities, it’s important to provide an overview of the existing setup.
The trARPES system at LACUS is based on a high-harmonics-based laser system paired with a Phoibos 150 2D hemispherical analyzer and a 4-axis manipulator for experimental control.
This beamline, known as Harmonium, operates using high harmonic generation (HHG) \cite{arrell_harmonium_2017}.
Figure \ref{fig:beamline_cad} shows both top and isometric views of the entire beamline, including the various light sources with their respective specifications.

A Titanium Sapphire (Ti:Sa) oscillator (Coherent "Vitara") generates infrared (IR) pulses at \qty{1.6}{\electronvolt} (\qty{780}{\nano\meter}) with a repetition rate of \qty{80}{\mega\hertz}. These pulses are amplified using one of two available systems, either the KMLabs Wyvern or the Coherent Astrella amplifier, to pulse energies of \qty{2}{\milli\joule} (Wyvern) or \qty{1}{\milli\joule} (Astrella), both operating at a repetition rate of \qty{6}{\kilo\hertz}.
This high peak power is essential for driving the highly non-linear process of HHG.

In HHG, an ultrashort laser pulse ionizes a noble gas, typically Argon or Neon \cite{rudawski_high-flux_2013}.
The ionized electron is accelerated by the electric field of the IR driving pulse, gaining kinetic energy.
Upon a possible recombination with its parent ion, the electron emits an extreme ultraviolet (EUV) photon.
The EUV spectrum produced by this process consists of multiple harmonics, with energies that are odd multiples of the seed pulse energy $(2n+1) \hbar\omega$, thereby forming a harmonic comb.

There are three main methods for gas delivery in HHG, with the generation process either happening in a gas cells, a gas jets, or a gas-filled fibers \cite{paul_quasi-phase-matched_2003, heckl_high_2009, sudmeyer_femtosecond_2008, roser_131w220fs_2005}.
In all cases, the emitted harmonic spectrum is governed by phase matching conditions, which can be optimized for different energy ranges.
Phase matching is achieved by minimizing the phase mismatch between the n-th harmonic and the fundamental
\begin{equation}
	\Delta k = k_n - k_1
\end{equation}
where $k$ is the respective wavenumber. 
Optimal phase matching is achieved when $\Delta k=0$.
For example, in different gases different phase matching conditions can be realized, allowing for better access to different spectral ranges.
Typically, Argon is used to create energies between \qtylist{20; 40}{\electronvolt} and Neon between \qtylist{40; 110}{\electronvolt} (see Fig. \ref{fig:hhgspectrum}) \cite{rudawski_high-flux_2013}.
Gas pressure, laser intensity, and target position relative to the focal point can all be adjusted to optimize phase matching and maximize harmonic output.
The small cross section of the HHG process, due to the low probability of the electron recombining with the ionized gas atom, as wells as the non-perturbative nature of the process are the reason for the need of high peak fluences at high repetition rates.
Moreover, high repetition rates are desirable for photoemission spectroscopy (PES) studies to mitigate space charge effects, which are discussed further in section \ref{sec:space_charge}.

Once the harmonic comb is generated, a single harmonic is selected for trARPES measurements using a monochromator.
This monochromator comprises two toroidal mirrors, a grating, and a slit module.
The first toroidal mirror collimates the beam onto the grating, where the harmonics are dispersed.
A second toroidal mirror refocuses the desired harmonic onto an adjustable slit.
The system offers multiple gratings, each with different amounts of grooves per \unit{cm} (\unit{g\per\cm}), with the different gratings being used for different energy ranges.
The gratings are illuminated at grazing incidence, with grooves aligned parallel to the propagation direction to minimize pulse front tilt of the emitted wavefront \cite{poletto_time-preserving_2010}.
The selected harmonic is then focused onto the sample by a third toroidal mirror.

\begin{figure}
	\centering
	\begin{subfigure}[b]{0.45\textwidth}
		\includegraphics[width=\textwidth]{images/vuv/laser_systems_top}
		\caption{}
	\end{subfigure}
	\begin{subfigure}[b]{0.45\textwidth}
		\includegraphics[width=\textwidth]{images/vuv/beamline_iso}
		\caption{}
	\end{subfigure}
	\caption{(a) Top view and (b) isometric view of the trARPES setup, featuring both the Harmonium and the VUV light sources.}
	\label{fig:beamline_cad}
\end{figure}

A portion of the amplified IR beam is split off before the HHG process to serve as the pump beam used to excite the sample.
After compensating for the probe beam path, a telescope system consisting of two curved mirrors focuses the pump beam onto the sample and adjusts the spot size.
The pump and probe beams co-propagate with a small angular offset, dictated by geometric constraints.
Additionally, a $\beta$-bariumborate (BBO) crystal can be introduced into the pump beam path to double the energy from \qty{1.6}{\electronvolt} to \qty{3.2}{\electronvolt}.

The Harmonium beamline can deliver a high photon flux of up to \qty{2.3e11}{\pps} (at \qty{36}{\electronvolt}) for harmonics between \qtyrange{20}{110}{\electronvolt}.
An energy resolution of approximately \qty{150}{\milli\electronvolt} is achievable, depending on the chosen harmonic and grating, as well as possible space charge effects, which are discussed in more detail in the following section.
The temporal resolution is less than \qty{100}{\femto\second}, as determined by the full width at half maximum (FWHM) of the temporal cross-correlation, measured using the laser-assisted photoemission (LAPE) effect \cite{ojeda_harmonium_2015}.
Although HHG pulses are inherently short, limited only by the spread in recombination times within the gas jet, the probe beam suffers from pulse front tilt due to the single grating monochromator.
The pulse front tilt leads to a increase in pulse duration over the spot size, which together with the pump pulse duration of $\approx$\qty{50}{\femto\second} results in the above time resolution \cite{arrell_harmonium_2017}.
The high time resolution is due to the monochromator design, using a time preserving single grating type monochromator \cite{poletto_time-preserving_2010}, which compensates the pulse front tilt, typically induced a reflective grating.
Time preserving monochromators and the effect of pulse front tilt are discussed in a following section.

\section{Vacuum space charge effects}
\label{sec:space_charge}

One of the key limiting factors for energy resolution in high-resolution or time-resolved ARPES experiments is vacuum space charge effects.
This phenomenon occurs when electrons traveling through a vacuum experience mutual repulsion due to Coulomb interactions.
When the electron density exceeds a certain threshold, these space charge effects significantly alter the distribution of the electron cloud.
This results in changes to the energy, momentum, and temporal information carried by the individual electrons, ultimately reducing measurement resolution and introducing artifacts.

\begin{figure}[b!]
	\centering
	\includegraphics[width=0.6\linewidth]{images/vuv/space_charge_lit}
	\caption{The figure shows the attainable sample current, corresponding to the amount of emitted photoelectrons, while maintaining a specified energy resolution. The dashed lines indicate the maximum achievable sample currents for different repetition rate scenarios, with a spot size of \qty{1}{\milli\meter} on the sample. This figure was adapted from \cite{corder_ultrafast_2018}, with markers representing various published ARPES setups.}
	\label{fig:spacechargelit}
\end{figure}

The degree of energy broadening caused by space charge can vary significantly depending on the kinetic energy of the electrons and the pulse duration of the light that generates the electron cloud.
Several studies have examined the impact of vacuum space charge on energy resolution for various light pulse energies and durations, as well as the additional effects of a pump pulse \cite{corder_ultrafast_2018,plotzing_spin-resolved_2016,hellmann_vacuum_2009,graf_vacuum_2010,frietsch_high-order_2013}.

In general, for one-photon photoemission processes involving light pulses in the \qtyrange{5}{100}{\electronvolt} range and sub-\unit{\pico\second} pulse durations, space charge effects exhibit a linear dependence on electron density.
This linear dependence manifests in both energy shifts and broadening of the photoemission spectra \cite{corder_ultrafast_2018,plotzing_spin-resolved_2016}.
The electron density, denoted by $\rho = N/D$, is determined by the number of emitted electrons $N$, and the spot size of the light on the sample $D$.
The measurable current, $I_\text{sample}$ at the sample can be used to calculate energy broadening and shifts according to
\begin{equation}
	\Delta E_{s,b} = m_{s,b} \frac{I_\text{sample}}{e f_\text{rep} D}
	\label{eq:spacecharge}
\end{equation}
where $f\text{rep}$ is the repetition rate of the light source, and $m_{s,b}$ are empirical scaling factors for energy broadening (b) and energy shift (s). These scaling factors have been determined as $m_b=$\qty{2.1e-6}{\electronvolt} and $m_s=$\qty{2.1e-6}{\electronvolt} \cite{plotzing_spin-resolved_2016}.

In addition, space charge effects can arise from photoemission caused by the pump laser pulses.
High-fluence pump pulses can generate additional photoelectrons through multiphoton processes \cite{al-obaidi_ultrafast_2015}.
Studies by Graf et al. \cite{graf_vacuum_2010} focused on the low-energy (\qty{6}{\electronvolt}) and few-picosecond regime, where they also observed a linear relationship between electron density and both energy shifts and broadening.
These effects can be described by the following expressions for energy shift and broadening

\noindent\begin{minipage}{.5\linewidth}
	\begin{equation}
		\Delta_E(N,D)=0.5\times \frac{N^{1.1}}{D^{1.1}}
	\end{equation}
\end{minipage}%
\begin{minipage}{.5\linewidth}
	\begin{equation}
		\Gamma_E(N,D)=13\times \frac{N^{0.7}}{D^{1.2}}
	\end{equation}
\end{minipage}

Given a constant number of electrons over a one-second interval, the repetition rate has a significant impact on energy spread, as seen from the inverse proportionality in the equations above.
For example, dividing $N$ photoelectrons into a million pulses (corresponding to a \unit{\mega\hertz} system) reduces space charge effects significantly compared to dividing them into only 6000 pulses, typical of a system like Harmonium.
Consequently, incorporating high-repetition-rate light sources in trARPES or ARPES experiments is crucial for improving overall system resolution.
Figure \ref{fig:spacechargelit} adapted from \cite{corder_ultrafast_2018}, shows the maximum possible photocurrent for different repetition rates, plotted against energy resolution for a beam spot size of \qty{1}{\milli\meter}.
Different markers are added from multiple light sources referenced in \cite{corder_ultrafast_2018}.
The figure illustrates very well the strong effect higher repetition rates have on the highest attainable number of photoelectrons for high energy resolutions.

In the case of the Harmonium beamline, a large portion of the EUV photons must be filtered out during trARPES experiments to mitigate space charge effects, which is yet another reason for moving toward high-repetition-rate systems.
Beyond the space charge issue affecting energy resolution, time resolution also faces limitations.
For example, Harmonium’s time resolution is constrained by the pump pulse duration, which could be improved by compressing the pulse to the few-femtosecond regime \cite{nisoli_compression_1997}.
Additionally, the pulse front tilt introduced by the monochromator presents further challenges to time resolution, a topic that will be covered in the next section.
	
\section{Time resolution limit due to pulse front tilt}

In trARPES experiments, the overall time resolution is determined by the cross-correlation of the pump and probe pulses.
As a result, minimizing the pulse duration of both the pump and probe pulses is crucial to achieving high temporal resolution.
Short pump pulses can be generated with relative ease through pulse compression techniques, such as white light generation followed by re-compression \cite{nisoli_compression_1997}. For typical HHG beamlines like Harmonium, the probe pulse is inherently shorter than the seed pulse often used for the optical pump.
In the case of a Gaussian pulse not experiencing any change in phase or frequency the time duration of such pulses can be expressed as
\begin{equation}
	\Delta\tau = \frac{2 \ln{2}}{\pi c}\frac{\lambda^2}{\Delta\lambda} = \frac{0.44}{c}\frac{\lambda^2}{\Delta\lambda}
\end{equation}
where $\lambda$ is the central wavelength, and $\Delta\lambda$ is the spectral width at half-height.
However, in HHG beamlines, a comb of harmonics is generated rather than a single pulse, requiring the use of a monochromator to select a single harmonic for ARPES experiments. Consequently, the effect of the monochromator on the pulse duration must be considered.

Monochromators used in HHG systems must accommodate the broad harmonic spectrum and allow for tuning across a wide frequency range.
Typically, such monochromators employ reflection gratings, which contain broad spectral bands when used at grazing incidence.
The problem of gratings arises from the fact that they induce a tilt of the pulse wavefront, after a light pulse is diffracted off of them.
The pulse front tilt is a result of the light rays being diffracted off of different neighboring grooves, which creates a path delay equal to $m\lambda$, with $m$ corresponding to the order of diffraction.
The total path difference across the full spatial pulse distribution can be summed to $m\lambda N$, with N totaling the number of illuminated grooves.
Depending on the properties of the chosen harmonic and the employed grating this contribution can be quite substantial, especially when the goal is a time resolution in the femtosecond regime.
This pulse front tilt can significantly impact time resolution, particularly when aiming for femtosecond-level precision.
Therefore, minimizing or compensating for this tilt is essential when designing a time-preserving monochromator.

One method for compensating pulse front tilt is to introduce a second grating, which counteracts the tilt introduced by the first grating \cite{villoresi_compensation_1999}.
While effective, this approach increases the complexity of the monochromator system and reduces the overall transmission.
An alternative approach involves using a single grating, where the total path difference is given by $m\lambda N=\lambda^2/\Delta\lambda$ \cite{samson_j_a_vacuum_1998} and therefore the pulse duration at its half-width equates to
\begin{equation}
	\Delta\tau = \frac{0.5}{c}\frac{\lambda^2}{\Delta\lambda}.
\end{equation}
This duration is only marginally longer than the Fourier-limited case \cite{poletto_time-preserving_2010,nugent-glandorf_laser-based_2002,poletto_time-compensated_2004,poletto_time-delay_2006}.
However, one drawback of this approach is that the energy resolution decreases as the initial pulse duration shortens prior to monochromatization.
In the Harmonium beamline, the single grating approach is used to achieve high temporal resolution while sacrificing some degree of energy resolution.
This compromise allows for enhanced time-domain precision, which is critical in many trARPES experiments, especially those that focus on capturing ultrafast dynamics.

\section{High repetition rate VUV trARPES}
\label{sec:high_rep_vuv}

Recent advances in HHG technology, particularly the extension into the multi-megahertz regime \cite{mills_xuv_2012,hadrich_high_2014,pronin_high-power_2015,saraceno_toward_2015,hadrich_single-pass_2016,carstens_high-harmonic_2016,zhao_efficient_2018}, have paved the way for their adaptation in time-resolved ARPES (trARPES) systems.
With these developments, trARPES systems have emerged that operate at repetition rates as high as \qty{88}{\mega\hertz} \cite{corder_ultrafast_2018,mills_cavity-enhanced_2019}.
These systems use cavity-enhanced seed pulses in resonators to achieve higher repetition rates while maintaining a high photon flux at high energies.

The primary benefit of high repetition rate HHG systems is the significant increase in statistical data collection over the same measurement period, along with a reduction in space charge effects, as discussed earlier.
However, two main challenges arise with this approach.
The first is the increased technical complexity in continuously operating these systems compared to simpler gas jet or fiber-based HHG setups.
While technical challenges can be resolved, the second and more fundamental issue concerns the high repetition rate of the pump pulses.

The key factor for studying out-of-equilibrium phenomena is the energy density, measured in \unit{\milli\joule\per\centi\meter^2}.
If this parameter is held constant, increasing the repetition rate from \qty{1}{\kilo\hertz} to \qty{100}{\mega\hertz} raises the average power by five orders of magnitude.
At these levels, thermalization becomes a critical limiting factor.
The higher average power places a greater thermal load on the sample, leading to heat accumulation, temperature rise, and accelerated sample degradation.
This thermalization effect creates a hard limitation on high repetition rate trARPES setups, restricting the number of samples that can be measured at elevated powers or requiring a reduction in peak power that may prevent the observation of out-of-equilibrium effects.
Finding an optimal compromise between repetition rate and manageable thermal load is key.
Operating at intermediate rates from \qtyrange{0.5}{2}{\mega\hertz} provides a balance, allowing for high statistical accuracy, sufficient pump fluences for studying strong perturbations, and reduced space charge effects.
This approach has been adopted in the new trARPES setup at LACUS.

Another challenge in trARPES is extending the spectral range of the probe pulses.
HHG systems can easily access photon energies between \qtyrange{15}{110}{\electronvolt} \cite{chini_coherent_2014,weissenbilder_how_2022}, while lower energies (below \qty{7}{\electronvolt}) can be reached through frequency conversion in nonlinear crystals such as BBO and KBBF (Potassium Beryllium Fluoroborate).
However, there remains a spectral gap between \qtyrange{7}{15}{\electronvolt}.
To address this, third harmonic generation (THG) in crystals is typically used, creating UV pulses around \qty{3.6}{\electronvolt}, which are then tripled again in a xenon-filled gas cell to generate harmonics around \qty{11}{\electronvolt}.
This method, though effective, offers limited tunability.
But information from ARPES, like any other spectroscopic technique, heavily depends on the employed probe energy due to the dipole matrix element, making it very desirable to tune the probe energy.
Additionally, the tunability of photon energy allows for selection of different momentum regions.
The new VUV light source overcomes this limitation, providing tunability in the \qtyrange{7}{10.8}{\electronvolt} range with \qty{1.2}{\electronvolt} increments by using a highly cascaded harmonic generation (HCHG) process instead of double THG process \cite{couch_ultrafast_2020}.

\begin{table}[t]
	\centering
	\begin{tabular}{ *{4}{c} }
		\hline
		Rep. rate (\unit{\kilo\hertz})		& Osc. Power (\unit{\milli\watt}) 	& Booster Current (\unit{\ampere}) 		 & Booster Power (\unit{\milli\watt})	\\ \hline\hline
		1000								& 250							  	& 2.69								  	 & 1000									\\ \hline
		Rod Current (\unit{\ampere}) 		& Rod Power (\unit{\watt})			& IR Input VUV (\unit{\milli\watt})		 & Green Input VUV (\unit{\milli\watt})	\\ \hline\hline
		8.17								& 12					  			& 1600								   	 & 1600									\\ \hline
	\end{tabular}
	\caption{Table of laser parameters for everyday use. The repetition rate can be adjusted in the VUV laser software. The oscillator output power and diode currents for both the booster and rod amplifier are displayed in the software. The average power at different positions (booster, rod, and VUV input) must be measured manually using a power meter.}
	\label{tab:laser_param}
\end{table}

In the HCHG process, two driving pulses are required, the fundamental and its second harmonic.
For this setup, the fundamental pulse has a wavelength of \qty{1030}{\nano\meter}, typical for \ce{Yb}-doped fiber-based oscillators.
The oscillator ("YLMO" by Menlo Systems) operates at a repetition rate of \qty{50}{\mega\hertz}, producing $<$\qty{80}{\femto\second} pulses with a pulse energy exceeding \qty{40}{\nano\joule}.
In order to drive the HCHG process higher pulse energies are necessary, which are reached in two subsequent amplification steps.
The oscillator output is monitored within the laser software and a typical average power of \qtyrange{220}{260}{\milli\watt} can be observed.
After the pulses leave the fiber oscillator they pass through a first stretcher, increasing the pulse duration, allowing for better amplification via chirped pulse amplification (CPA) \cite{strickland_compression_1985,maine_generation_1988,strickland_chirped_2021}.
CPA allows pulses to be amplified without damaging the gain medium, by distributing the pulse energy in a greater temporal range. Additionally, an electro-optic pulse picker reduces the repetition to \qtyrange{0.5}{2}{\mega\hertz}, depending on the needs of the trARPES experiment.
The pulses are the pre-amplified by the fiber based booster amplifier from \unit{\nano\joule} to a few \unit{\micro\joule}, at a boster diode current of \qty{2.69}{\ampere}.

A second stretcher is used to further increase the pulse duration of the IR pulses with the help of a transmission grating.
Typically an average power of \qtyrange{950}{1050}{\milli\watt} can be measured, corresponding to a pulse power of \qtyrange{0.95}{1.05}{\micro\joule}.
A fiber based rod-amplifier ("AeroGain" by NKT Photonics) then increases the pulse energies from few \unit{\micro\joule} to $>$\qty{10}{\micro\joule}.
The amplification is done by coupling the pre-amplified pulses as a seed into a single-mode gain fiber, which is pumped by a counter-propagating continuous-wave (CW) diode at \qty{976}{\nano\meter} with current of \qty{8.17}{\ampere}.
The amplified pulses are then re-compressed by a pair of gratings to \qty{200}{\femto\second} at up to \qty{12}{\micro\joule}.
Table \ref{tab:laser_param} contains an overview of the laser parameters, at various stages of amplification.

\begin{figure}
	\centering
	\includegraphics[width=0.7\linewidth]{images/vuv/vuv_laser}
	\caption{Picture of the open VUV laser system, showing the various modules.}
	\label{fig:vuvlaser}
\end{figure}

In a typical experiment, approximately \qty{5}{\micro\joule} of pulse energy is used for the highly cascaded harmonic generation (HCHG) process, with the remaining energy reserved for optical pumping.
The \qty{5}{\micro\joule} pulse is split into two beams.
One beam passes through a BBO crystal, which generates the second harmonic at \qty{515}{\nano\meter}.
The second harmonic beam is then directed over a delay stage, after which it is recombined with the fundamental beam so that both pulses co-propagate into the hollow-core photonic crystal fiber (HCPCF).
The delay stage ensures that the fundamental and second harmonic pulses are synchronized when they enter the fiber, which is crucial for generating the third harmonic and initiating the HCHG process.
A waveplate and polarizing prism pair is used to fine-tune the power balance between the two beams.
Typically, the highest VUV output is achieved when the power of the fundamental and second harmonic beams is equal as they enter the fiber.
Additionally, a waveplate in the green (second harmonic) beam path ensures that both beams are co-linearly polarized before entering the fiber.

Earlier experiments demonstrating the cascaded harmonic generation process utilized capillary waveguides with large diameters (greater than \qty{100}{\micro\meter}) \cite{misoguti_generation_2001}, but those setups relied on Ti:Sa oscillators, which produce much shorter pulse durations and higher pulse energies \cite{misoguti_generation_2001,durfee_phase_2002,misoguti_nonlinear_2005}.
In contrast, the light source in this setup employs negative-curvature hollow-core photonic crystal fibers (HCPCF) with a smaller diameter (\qty{50}{\micro\meter}), allowing the use of fiber lasers with longer pulse durations \cite{couch_ultrafast_2020}.
These HCPCF fibers have a complex structure based on photonic crystals, which confine the light within the photonic band gap of the material \cite{kolyadin_negative_2015, wei_negative_2017}.
The negative curvature of the core boundary further enhances light confinement, hence the term "negative-curvature HCPCF" \cite{wei_negative_2017}.
Figure \ref{fig:hchgsketch} (a) (taken from \cite{couch_ultrafast_2020}) shows a sketch of the generation process together with a view of the structured face of the HCPCF.

The HCPCF is filled with xenon gas at a pressure of $\approx$\qty{100}{\kilo\pascal} (measured from vacuum) in which the third harmonic is created in a four-wave-mixing process, combining two photons of the second harmonic and creating one photon of the fundamental and third harmonic.
Once the third harmonic is generated it can combine with the fundamental or second harmonic in additional four-wave mixing processes, successively creating the higher harmonics up to the 15th order (see Fig. \ref{fig:hchgsketch} (b)).
From the fourth harmonic onward, multiple pathways can generate each harmonic.
For example, the 7th harmonic $\omega_7$ can be generated through any of the following processes:
\begin{equation}
\begin{aligned}
	\omega_7 &= \omega_6 + \omega_2 - \omega_1 \\
	\omega_7 &= \omega_5 + \omega_3 - \omega_1 \\
	\omega_7 &= \omega_4 + \omega_4 - \omega_1
\end{aligned}
\label{eq:vuv_pathway}
\end{equation}
With further examples being listed in the report by Couch et al. \cite{couch_ultrafast_2020}.

For harmonic generation to occur, phase matching conditions must first be satisfied for each harmonic.
As an example, the phase mismatch for the four-wave mixing process that generates the third harmonic in a gas-filled hollow-core fiber can be described by the equation
\begin{equation}
	\Delta k = k_1 + k_3 - 2k_2 = 2\pi N \left( \frac{\delta_3}{\lambda_3} + \frac{\delta_1}{\lambda_1} - \frac{2\delta_2}{\lambda_2}\right) - \frac{u}{4\pi a^2} (\lambda_3 + \lambda_1 -2\lambda_2)
	\label{eq:phasematching_vuv}
\end{equation}
with the first term representing the contribution which depends on the xenon pressure in the HCPCF, due to the pressure dependent refractive index, and the second term stemming from the confinement in the waveguide.
The refractive index of noble gases is close to 1 and can be described by the simplified Sellmeier expression \cite{bideau-mehu_measurement_1981}
\begin{equation}
	n-1\simeq\frac{Ne^2}{8\pi^2\epsilon_0mc^2}\sum_{i}^{}\frac{f_i}{\lambda_i^{-2}-\lambda^{-2}}
\end{equation}
Equation \ref{eq:phasematching_vuv} can be generalized for any harmonic, depending on the creation pathway mentioned in equation \ref{eq:vuv_pathway}, with $\lambda_n$ representing the wavelength of the nth harmonic, $k_n$ the corresponding wavevector, N the number density of the gas (in this case xenon), $\delta_n$ the wavelength dependent gas-dispersion \cite{bideau-mehu_measurement_1981}, u a mode dependent parameter \cite{couch_ultrafast_2020,durfee_iii_guided-wave_1999}) and a the waveguide diameter (here \qty{50}{\micro\meter}). 

Phase mismatch can be minimized for each harmonic by slightly adjusting the xenon gas pressure.
The third harmonic requires the highest pressure, with progressively lower pressures needed for each subsequent higher harmonic.
To meet the phase matching conditions for all harmonics, a typical xenon pressure of \qtyrange{100}{125}{\kilo\pascal} is applied to one end of the fiber, creating a pressure gradient along its length.
Couch et al. \cite{couch_ultrafast_2020} confirmed the $\chi^3$ dependence of the HCHG process by performing numerical simulations using nonlinear Schrödinger equations within the PyNLO package.
The simulations, which modeled the harmonic flux output, were in good agreement with experimental measurements, further supporting the process's $\chi^3$ nonlinear behavior \cite{hult_fourth-order_2007,ycas_g_pynlopynlo_2024}.

\begin{figure}
	\centering
	\includegraphics[width=0.7\linewidth]{images/vuv/hchg_sketch}
	\caption{(a) Sketch of the VUV system. IR pulses are generated by a \ce{Yb} fiber oscillator paired with a fiber amplifier. A BBO crystal generates the second harmonic from the fundamental IR pulse, and both are focused into a xenon-filled HCPCF, where higher harmonics are produced via an HCHG process. A monochromator then selects the desired harmonic. (b) Scheme illustrating possible four-wave mixing pathways that create harmonics up to the 9th order in the HCHG process. Adapted from \cite{couch_ultrafast_2020}.}
	\label{fig:hchgsketch}
\end{figure}

Couch et al. \cite{couch_ultrafast_2020} have also reported that high resolution spectra of the higher harmonics have a higher bandwidth full-width at half-maximum (FWHM) than the driving pulse (\qty{40}{\milli\electronvolt} instead of \qty{8}{\milli\electronvolt}), indicating that the HCHG process creates shorter pulse duration than the one of the driving pulse, similar to HHG \cite{gagnon_soft_2007}.
In our study \cite{hellbruck_high-resolution_2024}, we found that the time resolution in ARPES experiments was longer than the driving pulse duration.
However, we also discovered that this resolution, and consequently the VUV pulse duration, was constrained by the pulse front tilt introduced by the monochromator grating.
This indicates that the actual pulse duration might indeed be shorter than the driving pulse.
Further measurements using a time-compensated monochromator could help determine the exact pulse duration at the fiber output.

Finally, a key advantage of this laser system is its reliability and ease of operation.
The VUV laser employs a fiber-based oscillator and two fiber-based amplifiers, making the setup not only more compact but also significantly more stable.
For example, the oscillator requires no realignment, and the amplifier only needs minor adjustments every few months.
This greatly reduces the time spent on maintenance, allowing more focus on the experiment itself.
Additionally, the integration of fibers in fixed cartridges, without the need for manual adjustments, makes the VUV generation process quite stable.
The photon flux can be re-optimized with minimal adjustments, even after several days of continuous operation.

This new light source opens up possibilities for trARPES measurements in energy ranges that were previously inaccessible with the HHG beamline.
It offers higher energy resolution than the Harmonium system, albeit with some compromise in time resolution, making it a valuable complement to the existing HHG beamline.
The higher repetition rates enable faster data acquisition without sacrificing statistical accuracy, while still allowing for high pump fluence measurements.
However, the system is particularly well-suited for exploring the low-perturbation regime, an area that posed significant challenges with the Harmonium setup.

\section{Paper and Supplementary}

The previous sections introduced the capabilities of each instrument, while this chapter will focus on integrating the new light source into the existing lab setup and fully characterizing its performance.
A paper detailing the complete trARPES setup, incorporating both light sources and featuring a comprehensive characterization of the VUV light source, has already been published.
This section will refer to that paper for the characterization details and will expand on specific aspects of the implementation.
Additionally, supplementary information relevant to the paper will be provided.

trARPES is a versatile technique that is heavily dependent on the light source used, due to the influence of the matrix element on the measurements.
The earlier discussion outlined the various light source options and the key parameters required for trARPES.
In that context, I explained why the VUV laser was chosen and how it complements the existing Harmonium beamline.
Together, these light sources offer a wide range of options for trARPES.
To maximize the potential of this combination, it is crucial to ensure seamless interoperability between the two beamlines.

One of the primary challenges during the planning and construction of the new setup was making sure that switching between the two beamlines could be done quickly, ideally within a minute, without introducing extended downtime or requiring significant realignment.
Additionally, the modifications needed to avoid impacting the Harmonium beamline.
To achieve these goals, a new vacuum chamber was installed in the HHG beamline.
This chamber houses motorized optics that can be inserted to direct the VUV pump and probe beams to the sample.
The optics include a flat IR mirror for the pump beam and two \ce{Al} mirrors with \ce{MgF2} coating for focusing and redirecting the probe beam.
Each optic is mounted on an individual motorized stage, allowing for rapid switching between beamlines without breaking the vacuum or requiring further adjustments beyond routine alignment.

Another challenge arose from the xenon gas used in the VUV generation, which leads to a pressure of approximately \qty{1e-4}{\milli\bar} in the monochromator.
Reducing this pressure to the low \qty{1e-10}{\milli\bar} range in the ARPES chamber is essential for preventing sample degradation.
To address this, multiple differential pumping stages were introduced, including turbo pumps on both the recombination chamber and the beamline connecting to the laser, as well as an ion pump and a combination of ion/NEG pump directly attached to the beamline tube.
A gate valve equipped with a \ce{MgF2} window allows for complete separation of the two vacuum systems, enabling measurements at lower base pressures.
However, using this valve results in a 50\% reduction in intensity and is therefore used sparingly.

The VUV trARPES setup also includes a pump beamline with several key components.
A motorized attenuator with thin film plates (TFP), positioned after the exit port of the residual IR beam, allows for precise control over the fluence used for creating the out-of-equilibrium excitation.
This setup enables fluence-dependent measurements to be automated via the measurement software.
The attenuator is used in a way that the beam reflected by the two TFPs is used for the pumping, ensuring the shortest possible pulse duration, while the transmitted beam is dumped on a beamblock.
Following the attenuator, two 2-inch mirrors compensate for the longer probe beam path.
The two 2-inch optics were chosen to allow for multiple bounces on each mirror, keeping the compensation as compact as possible.
The beam is then directed to a motorized delay stage, followed by a focusing telescope consisting of two convex lenses that focus the beam to a spot size of \qty{390}{\micro\meter} x \qty{360}{\micro\meter} on the sample.
Fine adjustments to the spot size can be made using a \unit{\micro\meter} stage on which the second lens is placed.
A BBO crystal can be inserted near the focal point to enable second harmonic pumping at \qty{515}{\nano\meter}.
When using the second harmonic for pumping, IR-optimized optics are replaced with green-light-optimized optics, and any residual IR light is removed from the pump beam.
The pump laser beam enters the recombination vacuum chamber through a window port, parallel to the probe beam, after which it can be directed to the sample with an insertable mirror.
Fine adjustment of the pump beam position is done with the help of a piezo-mirror mount before entering the recombination chamber.

Within the VUV probe arm, a diagnostic chamber contains a photodiode for measuring the photon flux of the selected VUV beam after it passes through the monochromator.
This chamber can also be used to measure the full VUV spectrum generated by the HCPCF by rotating the monochromator’s grating in steps and recording the flux at each position.
Figure \ref{fig:vuv_spectrum} shows the output of the VUV fiber, capturing the four harmonics that pass through the monochromator with high intensity.
More detailed information on the harmonics, spectral shape, and relative intensities can be found in the published paper referenced in this chapter.

The paper also compares the VUV light source to two complementary sources and fixed-wavelength VUV systems.
It provides a detailed description of the monochromator and its working principles, including the advantages and disadvantages of a normal-incidence monochromator.
The integration of the VUV source into the trARPES setup is also discussed briefly.
After the description of the light source and its integration two sections about the achieved energy and time resolution fully characterize the trARPES setup with the VUV light source.
For the energy resolutions measurements on polychristalline gold have been performed for all combinations of the four harmonics (\qtylist{7.2;8.4;9.6;10.8}{\electronvolt}) and monochromator exit slits (\qtylist{500;100;32}{\micro\meter}), to determine the total energy resolution.
The same measurements have been performed on a Helium lamp to extract the linewidths of the harmonics in each case.
Additional measurements on \ce{Bi2Se3} and \ce{Au(111)} have been performed to demonstrate the energy and momentum resolution by resolving the Rashba split in gold and Dirac-cone in \ce{Bi2Se3}.
Two measurements on \ce{Bi2Se3} and \ce{TaTe2} have been performed to analyze the time resolution of the light source at \qty{10.8}{\electronvolt} at two monochromator exit slits (\qtylist{500;100}{\micro\meter}).
The paper concludes by putting the performance of the system into context with existing literature and offers suggestions for further improvement.

\begin{figure}[h!]
	\centering
	\includegraphics[width=0.7\linewidth]{images/vuv/VUV_spectrum_pp_w7}
	\caption{Spectrum of the monochromator output as measured on the photodiode in the diagnostics chamber located behind the monochromator’s exit slit. The four selectable harmonics ($6\omega$, $7\omega$, $8\omega$, and $9\omega$) are shown at their respective central photon energies and spectral widths. The grating is rotated during the scan, and a photocurrent is measured on the diode. A responsivity curve is used to calculate the photon flux in photons/s. For photon flux values of the different harmonics and various monochromator exit slit settings, refer to \cite{hellbruck_high-resolution_2024}.}
	\label{fig:vuv_spectrum}
\end{figure}

\section*{Statement on my Contribution}

In the following, I include the publication: "L. Hellbr\"uck, M. Puppin, F. Guo, D. D. Hickstein, S. Benhabib, M. Grioni, J. H. Dil, H. M. R{\o}nnow, and F. Carbone. High-resolution MHz time- and angle-resolved photoemission spectroscopy based on a tunable vacuum ultraviolet source. Review of Scientific Instruments 95, 033007 (2024)" \cite{hellbruck_high-resolution_2024}.
This publication is an integral part of my thesis.
I began working on this project in early 2021, with the goal of building a new trARPES beamline and integrating it into the existing setup.
The initial steps involved working on the VUV laser purchased from KMLabs and collaborating with the company's laser engineers to get the system operational.
Next, I established a protocol for the daily use of the laser, fine-tuning parameters, and optimizing the output.
Meanwhile, I planned the integration of the laser, which required the insertion of a new vacuum chamber as well as the construction of two optical beam paths.
Upon completing the system, I gathered initial static and time-resolved data on various samples and performed a full characterization of the new ARPES setup.
I conducted the data analysis and led the writing of the paper, in collaboration with M. Puppin, H. M. R{\o}nnow, and F. Carbone.
All co-authors contributed to various aspects of the project, as outlined in the author contributions section of the publication.

\includepdf[pages=-]{main/vuv_paper.pdf}

\section{Ongoing Developments}

Since the full implementation and characterization of the VUV laser system, several developments have been introduced to further enhance its capabilities.
One significant upgrade has been the replacement of the old manipulator.
The previous manipulator offered four adjustable axes, three Cartesian coordinates and one rotational axis—with limited access to a fifth degree of freedom (in-plane rotation), which could only be adjusted with a wobble stick and low precision.
Only the rotational axis was motorized in that setup.
The new manipulator, however, provides access to all six degrees of freedom, with each axis fully motorized.
This allows for greater control and enables more complex scanning procedures.

One of the main advantages of the new manipulator is the ability to adjust the in-plane rotational degree of freedom while directly observing the band structure in real time.
Additionally, the new system addresses a recurring issue with Fermi surface mapping.
Due to sample geometry, the center of rotation often does not coincide with the point imaged by the electron analyzer, causing slight shifts in the illuminated spot during sample rotation.
This mean that the illuminated spot slightly changes under sample rotation of a Fermi surface scan, resulting in the possibility to image adjacent domains.
To compensate for this the $x$ and $y$ coordinate have to adjusted when changing from one angle to the next to image the same sample position.
This process was automated due to the availability of the additional motorized degrees of freedom.
Instead, with the motorization of the new manipulator more complex movements are possible, where the $x$ and $y$ coordinates can be dynamically adjusted during a Fermi surface scan.

The new manipulator also offers improved temperature control.
It features an open-cycle cryostat that can cool samples down to \qty{4}{\kelvin}, with intermediate temperatures being achievable by adjusting the coolant flow rate.
A built-in heater enables fine adjustments, which allows for precise temperature control during scans, such as when crossing phase transitions or exploring mode softening.

\begin{figure}[h]
	\centering
	\begin{subfigure}[b]{0.46\textwidth}
		\includegraphics[width=\textwidth]{images/vuv/manip_old}
		\caption{}
	\end{subfigure}
	\begin{subfigure}[b]{0.46\textwidth}
		\includegraphics[width=\textwidth]{images/vuv/manip_new}
		\caption{}
	\end{subfigure}
	\caption{Pictures of the (a) old and (b) new manipulator.}
	\label{fig:manip_full}
\end{figure}

Another important development is the addition of polarization control for the VUV beamline.
While controlling the pump polarization can be easily achieved with an IR waveplate, managing the probe polarization is more complex.
In principle, the polarization of both the IR and green beams can be rotated before they enter the fiber, producing harmonics with the desired polarization.
However, due to a polarization-dependent rejector, the monochromator can only operate with p-polarized light.
Using s-polarized light results in residual IR and green light passing through, degrading the optics and causing intense stray light.
To resolve this, the polarization of the VUV light must be rotated after the monochromator.

Polarization control is essential, as evidenced by ARPES measurements on \ce{Bi}2212 at low probe energies (\qty{7}{\electronvolt}), which showed a strong dependence of the photoelectron yield on light polarization due to polarization-dependent matrix elements \cite{fanciulli_spin_2018}.
Recent tests using a probe energy of \qty{10.8}{\electronvolt} also revealed a low photoelectron yield and no resolvable band structure with s-polarized light.
The question naturally arises if a similar polarization dependent effect is at play or if the dipol matrix element results in a low photoelectron yield independent of the polarization.
The challenge in rotating the polarization at these photon energies stems from the lack of readily available half-wave plates (HWPs) with sufficient transmission.
A custom HWP was specifically manufactured for this use case (manufactured by Kogakugiken Corp.).
The initial implementation and first tests are ongoing and will hopefully provide full control over the polarization at \qty{10.8}{\electronvolt}, and allow the study of the matrix element in \ce{Bi}2212.

\begin{figure}[t]
	\centering
	\begin{subfigure}[b]{0.45\textwidth}
		\includegraphics[width=\textwidth]{images/vuv/head_old}
		\caption{}
	\end{subfigure}
	\begin{subfigure}[b]{0.45\textwidth}
		\includegraphics[width=\textwidth]{images/vuv/head_new}
		\caption{}
	\end{subfigure}
	\\
	\begin{subfigure}[b]{0.45\textwidth}
		\includegraphics[width=\textwidth]{images/vuv/sample_coordinates}
		\caption{}
	\end{subfigure}
	\caption{Pictures of the (a) old and (b) new manipulator heads. (c) Shows a sketch of the 6 available motorized axes from the point of a sample plate reference system.}
	\label{fig:manip_head}
\end{figure}

Another recent development focuses on leveraging one of the strengths of the VUV light source: its ability to study low-perturbation dynamics.
As mentioned in section \ref{sec:high_rep_vuv}, the nature of the high repetition rates make it possible to faster acquire the needed statistics.
This enables the observation of less pronounced features, which is typically the case for features occurring only after a small perturbation to the system.
Adding this to the fact that high pump fluences are also problematic at high repetition rates due to the thermalization issues described earlier, it seems that the best use case for the light source is the low perturbation regime.
To enhance this capability, a new detection scheme has been implemented, which alternates between measuring pumped and unpumped signals.
This approach should make pump-induced features more pronounced by reducing noise and fluctuations.

\begin{figure}
	\centering
	\includegraphics[width=0.65\linewidth]{images/vuv/chopper_electric_scheme}
	\caption{Sketch of the on/off detection scheme. The diagram illustrates the electrical connections necessary to synchronize the camera with the laser and chopper wheel. As an example, typical repetition rates for an experiment were chosen. The VUV laser provides light pulses at a repetition rate of \qty{1}{\mega\hertz}. The frequency divider (F-DIV) and delay generator divide the reference signal to a \qty{1}{\kilo\hertz} signal, in accordance with the possible trigger rate of the camera, and a second trigger signal at half the repetition rate for the chopper wheel, creating the pump on/off images. A photodiode measures the chopped beam and displays its repetition rate on an oscilloscope alongside the chopper controller reference signal and the camera reference signal for comparison.}
	\label{fig:triggerscheme}
\end{figure}

This detection scheme uses a fast acquisition camera, which can operate at frame rates of up to \qty{1.5}{\kilo\hertz}.
The camera can be triggered directly by the laser or any other suitable reference signal.
To capture both pumped and unpumped signals, a chopper wheel is employed, rotating at a frequency that matches the camera's acquisition rate.
With the VUV laser set to \qty{1}{\mega\hertz}, both the camera and chopper are synchronized to \qty{1}{\kilo\hertz}.
This means that each camera frame collects data from 1000 probe pulses, alternating between pumped and unpumped shots following one after the other.

A more detailed schematic of the detection system is shown in Fig. \ref{fig:triggerscheme}.
The VUV laser operates at \qty{1}{\mega\hertz}, and its trigger signal is sent to a frequency divider (F-DIV) and a delay generator.
The F-DIV can split the input signal into multiple channels, each with different division factors, while the delay generator allows for precise timing adjustments.
For instance, in a typical setup, one channel may divide the signal by $1000$, resulting in a \qty{1}{\kilo\hertz} trigger for the camera. Another channel could be set to divide the signal by $2000$, generating a \qty{500}{\hertz} trigger for the chopper controller.

The chopper controller then synchronizes the wheel's rotation to create alternating "pump on" and "pump off" frames at the correct frequency.
A reference signal can be send from the chopper controller to the camera as a second trigger (AOL).
Additionally, an oscilloscope can be connected to visualize all three signals simultaneously.
The chopper controller reference can be attached to the oscilloscope instead of the camera (AOL), the chopped pump pulse is detected by a photodiode and its signal can be shown simultaneously on the oscilloscope.
On top of that, the camera has a reference out channel too, which can be shown on the oscilloscope to compare the delay between all signals.
These electric delays can be adjusted by the F-DIV/delay generator to make sure that the camera is triggered in sync with the chopper and that the chopper functions at a repetition rate corresponding to the laser repetition rate.

A typical acquisition scheme of a trARPES experiment looks the following:
\begin{itemize}[topsep=-0.5em]
	\setlength\itemsep{-0.5em}
	\item [$\bullet$] \qty{1000}{\hertz} repetition rate of each camera shot
	\item [$\bullet$] setting integration window for both on and off signal to \qty{200}{\milli\second}
	\item [$\bullet$] summing for \qty{1}{\second}, while sorting the pictures in corresponding on and off bin
	\item [$\bullet$] move stage to next delay position and repeat on and off picture
	\item [$\bullet$] after finishing delay list, start at first delay and repeat loop
\end{itemize}
The main challenge with the implementation of this detection scheme is accounting for all accumulated delays and detector dead times.
These delays were investigated using an oscilloscope and adjusted with a delay generator.
A significant issue arose from the decay time of the phosphor screen at the end of the hemispherical analyzer, which is in the millisecond range.
This required careful alignment of the acquisition start time to avoid signal spillover from "pump on" to "pump off" frames.
Systematic testing of several electrical delays was performed by using only the pump beam and monitoring the multiphoton photoemission signal.
The goal was to observe a signal when the pump beam was allowed to pass by the chopper and an empty frame when it was blocked.
Using this method, an optimal acquisition window was determined, and by fine-tuning the camera's threshold settings, spillover was successfully avoided without a significant reduction in the overall signal strength.

Initial tests of this detection scheme were conducted on the Harmonium beamline using reference samples, such as \ce{WSe2}, which are known to produce strong out-of-equilibrium signals \cite{puppin_excited-state_2022}.
While the scheme was successfully implemented, no noticeable improvement in signal quality was observed.
This could be due to several factors.
First, the measurements were taken far from equilibrium, where comparatively high signal intensity is expected, but pump-induced artifacts may obscure more subtle features.
Second, no significant changes to the band structure were detected below the Fermi level, where improvements from the on/off detection scheme would be most apparent.
The region above the Fermi level, which contains the excitonic state, inherently lacks background noise and, therefore, would not show much improvement from the on/off imaging technique.

One clear downside of the on/off detection scheme is that it effectively halves the statistical data collected in the same timeframe compared to traditional methods.
This means improvements will only be noticeable if the noise levels in the traditional pump-probe scheme are too high to observe pump-induced effects.
Further testing is necessary with samples that display changes in the low perturbation regime and within the occupied region of the band structure to assess whether the on/off detection scheme provides better results in these scenarios.

\begin{figure}
	\centering
	\includegraphics[width=\textwidth]{images/vuv/on-off_bandmaps}
	\caption{The figure shows bandmaps and difference maps of the K-point of \ce{WSe2}, measured with Harmonium using the pump on/off detection scheme. All maps are shown at a time delay of \qty{0}{\femto\second} between pump and probe. (a) displays the bandmap for pump on, with the faint exciton state visible above $E_F$. (b) represents the corresponding difference map between the raw data and the static background ($on-\overline{on}$). (c) and (d) show the equivalent maps with the pump off. (e) presents the difference map between the on and off raw data.}
	\label{fig:on-offbandmaps}
\end{figure}
