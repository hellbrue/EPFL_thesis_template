\cleardoublepage
\chapter*{Introduction}
\markboth{Introduction}{Introduction}
\addcontentsline{toc}{chapter}{Introduction}


Correlated matter refers to a class of materials in which their properties can no longer be explained solely by the traditional quantum mechanical description of electrons.
In this area of condensed matter physics, many well-established concepts, such as band theory \cite{bethe_sommerfeld_bloch} and Landau’s theory of Fermi liquids \cite{landau}, begin to break down.
As a result, the strong electron-electron interactions in these systems lead to collective behaviors that give rise to novel and exotic phases of matter.
Examples of these rich emergent phenomena include high-temperature superconductivity, heavy fermion behavior, and the Mott insulating state.

Understanding these materials poses a significant theoretical challenge due to the strong correlations between electrons, compounded by the many-body nature of these systems.
Traditional tools like density functional theory (DFT) are often insufficient to capture the complexity of these interactions.
However, over the last few decades, significant progress has been made.
The development of new theoretical frameworks, such as the Hubbard model, the t-J model, and more recently, dynamical mean-field theory (DMFT), has enhanced our ability to describe these complex systems.
Parallel advances in material research have uncovered exciting new states of matter, including quantum spin liquids, superconductivity in magic-angle twisted bilayer graphene, moiré superlattices, and fractional Chern insulators.
Additionally, the study of these materials has greatly benefited from advances in experimental techniques.
Methods such as scanning tunneling microscopy (STM) and angle-resolved photoemission spectroscopy (ARPES) have provided new insights into the microscopic properties and electronic structures of correlated systems.

With the advent of time-resolved techniques and their continuous improvement, it has become possible not only to explore the dynamics and reformation processes of these materials but also to study their properties out of equilibrium.
This has led to a strong interest in using light as a means to manipulate and control the properties of correlated matter, with the hope of discovering new, potentially metastable phases that cannot be accessed adiabatically.
Prominent examples of such light-driven effects include light-induced superconductivity, Floquet engineering, light-driven magnetic switching, and photoinduced metal-to-insulator transitions.

The discoveries in this field have already led to numerous technological advances, with many promising applications still on the horizon.
Superconductors, in particular, are already widely used in various areas of daily life, from medical applications like MRI machines to transportation with maglev trains, the generation of extreme magnetic fields for nuclear fusion, and particle accelerators such as the Large Hadron Collider (LHC), as well as in current quantum computers.
Other types of correlated materials also show potential for use in technologies such as Mott transistors, advanced magnetic sensors, devices based on magnetoresistance and spin memory, next-generation photovoltaics, and many other applications.

\begin{center}
	\rule{0.3\textwidth}{.8pt}
\end{center}

But apart from a direct impact the field has on specific applications, a broader questions arises.
What is the role of fundamental physics research towards society?
Or to paraphrase one of my favorite comedy groups: "What has physics ever done for us?"

In fact we often try to justify our research by looking at the next possible, immediate application for a given subject.
And while it is important to keep applications and improved technologies in mind it should not always be the main driving force.
Fundamental research directly feeds into the basic human urge of curiosity and making new discoveries, attempting to just understand the universe better than previously.

It is not really possible to gauge the value of fundamental research in itself, because any impact felt by the general population is often many decades away.
For this it has been shown, that societies struggle with long term thinking, trading in short term costs for long term benefits.
And it even holds true when the long term benefits vastly outweigh the costs.
Maybe the most prominent example is climate change, a problem which we are aware of, for which we have the means to solve, but as a society choose not to.

Therefore thinking about the benefits fundamental research has might be even harder.
Not many people would have thought about a direct application, that the invention of general relativity would bring.
Instead it took a century and the development of atomic clocks to experimentally confirm the theory.
But nowadays, even though some might not be aware of it, both are centerpieces of our every day life and the world would work much different without the existence of GPS.

While not every study may have the profound impact of general relativity, predicting which current research will ultimately address future challenges is difficult.
As Maria Skłodowska-Curie once said:
Maria Skłodowska-Curie said
\begin{quote} 
	\centering 
	 Scientific work must not be considered from the point of view of the direct usefulness of it. It must be done for itself, for the beauty of science.
\end{quote}
In the end a healthy balance should exist between the aim for discoveries in their, maybe romantic and pure sense, and application-oriented research aimed at solving urgent and immediate challenges.

At this point, one might wonder why I am reflecting on the value of research.
Naturally, I do not intend to place the studies presented in this thesis alongside the great accomplishments mentioned earlier.
However, after dedicating the past four years to these topics, it is only natural to reflect on the deeper reasons for becoming a scientist.
Although I have come to appreciate the importance of considering the practical applications of research, I firmly believe that pursuing the pure joy of discovery holds immense value.
It not only influences the research of tomorrow but also has the power to inspire society at large.

\begin{center}
	\rule{0.3\textwidth}{.8pt}
\end{center}

It is exactly the combination of these two aspects of research that followed me throughout the last four years, working on the formation of metastable states in correlated matter, what there defining properties are and how to control them, what fundamental mechanisms are at play and how to advance existing metrology to gather higher quality data.

In \textbf{Chapter 1} I will present a brief introduction into ARPES, the technique which enabled me to study the band structure, and with it the electronic properties of correlated matter systems.
I will introduce both the basic principles behind the techniques at equilibrium, as well as the extension to the time domain.
In \textbf{Chapter 2}, I discuss the high temperature superconductor \ce{Bi}2212 and report on a photoinduced and metastable Lifshitz transition in the material and I will discuss how light influences the microscopic properties, such that this manipulation is possible.
For this I will provide a short summary on the relevant information of cuprates and the current understanding of high $T_c$ superconductors.
In \textbf{Chapter 3}, I present equilibrium ARPES data together with a preliminary analysis on the tri-layer cuprate \ce{Bi}2223.
In \textbf{Chapter 4}, I will discuss the transition metal dichalcogenide \ce{TaTe2} and the basic principle of charge density waves.
Here I will report on the out of equilibrium dynamics of the material and how Fourier analysis of trARPES data can provide information on the band selective electron-phonon coupling. Additionally I will report on a light-induced metastable state existing in the low temperautre phase.
Finally in \textbf{Chapter 5} I will present a new trARPES setup.
I was able to integrate a new vacuum ultraviolet light source into an existing beamtime, allowing to combine two complementary light sources enabling both high time and energy resolution in ARPES as well as unprecedented tunability.
I will introduce the new lgiht source and its integration into the existing beamline, as well as discussing the benefits and vision for the new combined system.
