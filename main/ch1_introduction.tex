\cleardoublepage
\chapter*{Introduction}
\markboth{Introduction}{Introduction}
\addcontentsline{toc}{chapter}{Introduction}


Correlated matter refers to a class of materials in which their properties can no longer be explained solely by the quantum mechanical description of independent electrons.
In this area of condensed matter physics, many well-established concepts, such as band theory \cite{bethe_theorie_1928,sommerfeld_zur_1928,bloch_bemerkung_1929} and Landau’s theory of Fermi liquids, begin to break down.
As a result, the strong electron-electron interactions in these systems lead to collective behaviors that give rise to novel and exotic phases of matter.
Examples of these emergent phenomena include high-temperature superconductivity, heavy fermion behavior, and the Mott insulating state.

Understanding these materials poses a significant theoretical and experimental challenge due to the strong correlations between electrons, compounded by the many-body nature of these systems.
Computational tools like density functional theory (DFT) are often insufficient to capture the complexity of these interactions.
However, over the last few decades, significant progress has been made.
The development of new theoretical frameworks, such as the t-J model, and more recently, dynamical mean-field theory (DMFT), has enhanced our ability to describe these complex systems.
Parallel advances in material research have uncovered exciting new states of matter, including quantum spin liquids \cite{broholm_quantum_2020}, superconductivity in magic-angle twisted bilayer graphene \cite{oh_evidence_2021}, moiré superlattices \cite{andrei_marvels_2021}, and fractional Chern insulators \cite{zeng_thermodynamic_2023}.
Additionally, the study of these materials has greatly benefited from advances in experimental techniques.
New spectroscopic tools such as scanning tunneling microscopy (STM) and angle-resolved photoemission spectroscopy (ARPES) have provided new insights into the microscopic properties and electronic structures of correlated systems.
\begin{figure}
	\centering
	\includegraphics[width=1\textwidth]{images/inro/sc_app}
	\caption{Examples of applications of correlated matter, in particular superconductors.}
	\label{fig:scapp}
\end{figure}

With the advent of time-resolved techniques and their continuous improvement, it has become possible not only to explore the dynamics and reformation processes of these materials but also to study their properties out of equilibrium \cite{giannetti_ultrafast_2016}.
This has led to a strong interest in using light as a means to manipulate and control the properties of correlated matter, with the hope of discovering new, potentially metastable phases that cannot be accessed adiabatically.
Prominent examples of such light-driven effects include light-induced superconductivity, Floquet engineering, light-driven magnetic switching, and photoinduced metal-to-insulator transitions \cite{fausti_light-induced_2011,takubo_photoinduced_2008}.

The discoveries in this field have already led to numerous technological advances, with many promising applications still on the horizon.
Superconductors, in particular, are already widely used in various areas, from medical applications like MRI machines to transportation with maglev trains, the generation of extreme magnetic fields for nuclear fusion, and particle accelerators such as the Large Hadron Collider (LHC), as well as in current quantum computers.
Other types of correlated materials also show potential for use in technologies such as Mott transistors, advanced magnetic sensors, devices based on magnetoresistance and spin memory, next-generation photovoltaics, and many other applications.

\begin{center}
	\rule{0.3\textwidth}{.8pt}
\end{center}

But apart from a direct impact the field has on specific applications, a broader question arises.
What is the role of fundamental physics research towards society?
Or to paraphrase one of my favorite comedy groups: "What has physics ever done for us?"

In fact we often try to justify our research by looking at the next possible, immediate application for a given subject.
And while it is important to keep applications and improved technologies in mind, it should not always be the only driving force.
Fundamental research directly feeds into the basic human urge of curiosity and making new discoveries, attempting to just understand the universe better than previously.

It is not really possible to gauge the value of fundamental research in itself, because any impact felt by the general population is often many decades away.
For this it has been shown, that societies struggle with long term thinking, trading in short term costs for long term benefits.
And it even holds true when the long term benefits vastly outweigh the costs.
Maybe the most prominent example is climate change, a problem which we are aware of, for which we have the means to solve, but as a society choose not to.

\begin{figure}
	\centering
	\includegraphics[width=1\textwidth]{images/inro/fundamental}
	\caption{Impact of fundamental research, from unimaginable applications to our understanding of the universe.}
	\label{fig:fundamental}
\end{figure}


Therefore, thinking about the benefits fundamental research would bring might be even harder.
Not many people would have thought about a direct application that the invention of general relativity would bring.
Instead it took a century and the development of atomic clocks to fully exploit the knowledge of relativity.
But nowadays, even though some might not be aware of it, both are centerpieces of our every day life and the world would work much different without the existence of GPS.

While not every study may have the profound impact of general relativity, predicting which current research will ultimately address future challenges is difficult.
As Maria Skłodowska-Curie once said,
\begin{quote} 
	\centering 
	 Scientific work must not be considered from the point of view of the direct usefulness of it. It must be done for itself, for the beauty of science.
\end{quote}
In the end a healthy balance should exist between the aim for discoveries in their possibly romantic and pure sense, and application-oriented research aimed at solving urgent and immediate challenges.

At this point one might wonder why I am reflecting on the value of research.
Naturally, I do not intend to place the studies presented in this thesis alongside the great accomplishments mentioned earlier.
However, after dedicating the past four years to these topics, it is only natural to reflect on the deeper reasons for becoming a scientist.
Although I have come to appreciate the importance of considering the practical applications of research, I firmly believe that pursuing the pure joy of discovery holds immense value.
It not only influences the research of tomorrow but also has the power to inspire society at large.

\begin{center}
	\rule{0.3\textwidth}{.8pt}
\end{center}

It is precisely the combination of these two aspects of research — exploring fundamental mechanisms and addressing practical applications — that has shaped the course of my graduate studies, focusing on out of equilibrium dynamics and the formation of metastable states in correlated matter.
My efforts have centered on understanding their microscopic properties, how to control or manipulate them, and how to improve existing metrology to obtain higher quality data.

In \textbf{Chapter 1}, I will provide an introduction to ARPES, the technique that allowed me to study the band structure and electronic properties of correlated matter systems.
I will explain both the fundamental principles of ARPES in equilibrium and its extension to the time domain.
In \textbf{Chapter 2}, I will discuss the high-temperature superconductor \ce{Bi2Sr2CaCu2O8} (\ce{Bi}2212), where I report on a photoinduced and metastable Lifshitz transition in the material.
I will explore how light influences the microscopic properties of this system, enabling such manipulation.
This will include a brief overview of cuprates and the current understanding of high-Tc superconductors.
In \textbf{Chapter 3}, I present equilibrium ARPES data, along with a preliminary analysis of the tri-layer cuprate \ce{Bi2Sr2Ca2Cu3O10} (\ce{Bi}2223), to further explore the electronic properties of these complex systems and adress the correlation between \ce{CuO2} layers and $T_c$.
\textbf{Chapter 4} focuses on the transition metal dichalcogenide \ce{TaTe2} and the fundamental principles of charge density waves.
I will discuss the out-of-equilibrium dynamics of this material and demonstrate how Fourier analysis of time-resolved ARPES (trARPES) data can provide insight into band-selective electron-phonon coupling.
Additionally, I will report on the discovery of a light-induced metastable state in the low-temperature phase of \ce{TaTe2}.
Finally, in \textbf{Chapter 5}, I will present a new trARPES setup that I contributed to developing.
By integrating a new vacuum ultraviolet light source into an existing beamline, I was able to combine two complementary light sources, enabling both high time and energy resolution in ARPES, along with unprecedented tunability.
I will introduce this new light source, explain how it was integrated into the beamline, and discuss the potential benefits and future vision for this combined system.