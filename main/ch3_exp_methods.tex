\chapter{Time \& angle resolved photoemission spectroscopy}

The development and understanding of quantum materials play a huge role in the advancement of modern technologies.
In many of these materials, interactions and correlations of electrons have a huge impact on the general material properties such as the formation of exotic phases.
Understanding what the impact of the electrons on these phases is and how the interactions can lead to the emergent phenomena is crucial.

Angle resolved photoemission spectroscopy stands out here as a unique technique, capable of measuring the single particle spectral function and by extension mapping out the band structure in the full 3D Brillouin zone (BZ).
The technique relies on the photoelectric effect, first observed by Hertz in 1887 and later described by Einstein in 1905 and is fundamentally a photon-in electron-out technique.
In recent years ARPES has matured as a technique, as can be seen in the extensive reviews on the topic, especially in the context of complex matter \cite{}, and was adapted to several different sub-techniques addressing certain needs, such as spin-ARPES, $\mu$- or $n$-ARPES, or tr-ARPES.

Focusing on time- and angle- resolved PES (trARPES), rapid development led to a sound understanding of this technique over the last decades, which is summarized in some of the recent reviews dedicated to trARPES \cite{}.
Due to the techniques ability of measuring the transient changes to the electronic band structure, as well as interactions with other electrons or bosons (e.g. phonons) it is a powerful tool to observe out of equilibrium dynamics.
Further it can observe lightinduced states and establish a hierarchy in formation process, or the melting process of equilibrium orders.
This provides profound inside into the understanding of correlated material, as well as path a way of manipulating and controlling such materials.

In this chapter I will give a brief description of the technique and where the measured signal originates from.
This will first be done by looking at the general equilibrium case, I will then extend the description to the time domain by looking mainly at the quasi-equilibrium regime, and briefly touching onto the out of equilibrium case.

\section{General description}

Before focusing on the time domain approach it is necessary to introduce the equilibrium case first.
As mentioned before ARPES is a photon-in, electron out technique relying on the photoelectric effect.
Therefore, if an incoming photon has an energy $h\nu$ that is greater than the workfunction of the material $\phi$, an electron may leave the sample with a kinetic energy corresponding to the energy difference.
If electrons are bound stronger than at the Fermi-level, the binding energy has to be considered as well, which leads to energy conservation formula as
\begin{equation}
	E_\text{kin} = h\nu - \phi - E_\text{Bin}.
\end{equation}
The photoelectrons that leave the sample will be collected by spectrometer, such as a hemispherical analyzer and the initial density of states can be observed.
In addition, it is possible to relate the angle at which the photoelectrons leave the sample and their respective kinetic energy to the crystal moment, that the electrons possessed inside the crystal due to momentum conservation rules.
The relation is given by
\begin{equation}
	\hbar \mathbf{k}_\parallel = \sqrt{2m_eE_\text{kin}} \sin\theta
\end{equation}
and is only valid for the in-plane momentum component.
The out of plane component $\mathbf{k}_\perp$ is not conserved, due to an additional potential resulting from the broken symmetry at the sample surface.
In fact $\mathbf{k}_\perp$ depends on the inner potential $V_0$, that is typically unknown before a measurement.
The equation for the out of plane component is given by
\begin{equation}
	\hbar \mathbf{k}_\perp = \sqrt{2m_e\left(E_\text{kin}\cos^2\theta+V_0\right)}
\end{equation}

Instead the perpendicular momentum component can be obtained by performing photonenergy dependent scans.
\begin{figure}
	\centering
	\includegraphics[width=0.7\linewidth]{images/trarpes/arpes_sketch}
	\caption{}
	\label{fig:arpes_sketch}
\end{figure}

For ARPES measurements at equilibrium, within the sudden approximation and for single bands, the measured intensity is calculated via Fermi's golden rule, considering the matrix element $M(\mathbf{k}, \omega)$, the Fermi-Dirac distribution $f(\mathbf{k}, \omega)$ and the spectral function $A(\mathbf{k}, \omega)$.
Here the sudden approximation describes the scenario in which the photoemission process occurs suddenly, meaning that no interaction between the created photoelectron and the system occur post photon absorption.
The intensity is then given by
\begin{equation}
	I(\mathbf{k}, \omega) = A(\mathbf{k}, \omega)\left|M(\mathbf{k}, \omega)\right|^2f(\omega)
	\label{eq:arpes_signal}
\end{equation}
The dipole matrix element corresponds to the probability of a transition between an initial and corresponding final state.
It not only contains the probability for a direct transition between an initial and finial state $\Phi_i$ and $\Phi_f$, but also information that is sensitive to the sample geometry, light polarization and ARPES setup due to the light-matter interaction Hamiltonian,
with the matrix element being given by
\begin{equation}
	M = \braket{\Phi_f\left|H_\text{int}\right|\Phi_i} =\braket{\Phi_f\left| \mathbf{A} \cdot \mathbf{p} \right|\Phi_i}
\end{equation}
with the vector potential $\mathbf{A}$ and the electron momentum operator $\mathbf{p}$.
This way depending on the parity of the initial state and vector potential, the intensity of bands can be fully suppressed depending on the probe light polarization.

Additionally $A(\mathbf{k}, \omega)=-(1/\pi)\text{Im}G(\mathbf{k}, \omega)$ is the single-particle spectral function, containing both information on the bare band dispersion of the single-particle $\epsilon_\mathbf{k}$ and its self energy $\Sigma(\mathbf{k}, \omega)$, representing the many body correlation effects influencing the single-particle. \cite{} mahan
The Green's function captures the inherent nature of the photoemission process leaving the system in an excited state and not strictly speaking at equilibrium.
It specifically treats the fact that an electron is removed from the system and the probability of how the system will evolve after the removal.
It is due to that fact that correlations and interactions are already embedded into the information of the measured photoelectron.

The complete spectral function is described by
\begin{equation}
	A(\mathbf{k}, \omega)= -\frac{1}{\pi} \frac{\Sigma''(\mathbf{k}, \omega)}{\left[ \omega - \epsilon_\mathbf{k} - \Sigma'(\mathbf{k}, \omega) \right]^2 + \left[ \Sigma''(\mathbf{k}, \omega) \right]^2}
\end{equation}
with $\Sigma'(\mathbf{k}, \omega)$ and $\Sigma''(\mathbf{k}, \omega)$ being the real and imaginary part of the self-energy respectively.
In general, the real part of the self-energy leads to shift of the bare band dispersion $\tilde{\epsilon}_k \rightarrow \epsilon_k + \Sigma'(\mathbf{k}, \omega)$, whereas the imaginary part results in a broadening, compared to a case where no interactions are present.
The lifetime of the excited state, resulting from the removal of a single electron can be calculated from the imaginary part of the self-energy $\tau=\hbar/\left[2\Sigma''(\mathbf{k}, \omega)\right]$.
A typical way of analyzing ARPES data with the purpose of extracting information on the self-energy is extracting energy distribution curves (EDC) for certain momenta, as well as looking at the momentum distribution curves (MDC) for certain energies.
Analyzing the position of and width of the corresponding spectral features provides insight into the self-energy.
Also, intimate knowledge of the bare-band dispersion is beneficial for the interpretation of the ARPES signal.

A more deteiles description of the equilibrium ARPES information can be found in the review by Damascelli et al. \cite{}.
After discussing the basic ARPES principles and the measured signal in the equilibrium case in this section, the following section will extend these fundamentals into the time domain.

\section{TR-ARPES: a stroboscopic time domain approach}

In order to extend ARPES into the time domain a stroboscopic pump-probe approach is used.
Here, a first pulse, called pump pulse, creates the excitation in a system, and a second probe pulse measures the excited state.
A delay is introduced between pump and probe, and by varying that delay the transient evolution of the system can be observed.
In this case the probe pulse is fixed in time and used as an internal clock.
The pump pulse is then delayed in respect to the probe pulse with the help of a translation stage.

The here described technique uses light pulses with an energy above the materials workfunction as a probe pulse, and a pulse with an energy below the workfunction creates the excitation.
In principle trARPES measurements using two photon absorption process, where both light pulses have an energy below the workfunction, are also possible, but will not be the topic of this chapter.
More insight into two photon photoemission (2PPE) trARPES can be found in several dedicated reviews \cite{}.

The description and interpretation of the ARPES signal after pump excitation is not trivial.
In a first approximation the equilibrium intensity from equation \ref{eq:arpes_signal} can be extended into the time domain
\begin{equation}
	I(\mathbf{k}, \omega, \Delta t) = A(\mathbf{k}, \omega, \Delta t)\left|M(\mathbf{k}, \omega, \Delta t)\right|^2f(\mathbf{k}, \omega, \Delta t)
\end{equation}
considering the the delay  $\Delta t$ between pump and probe \cite{} freericks kemper 2021.
The approximation assumes that the electrons thermalized into a quasi-equilibrium state, which often occurs after a time span of few tens or hundreds of femtosecond, depending on the material and light properties.
For example it has been shown that the adiabatic band dispersion can survive even at time zero and for high pump fluences \cite{} boschini review + neufeld 2022

Disentangling the various contributions is experimentally even more challenging than in the equilibrium case.
The Fermi-Dirac function, which at equilibrium solely depends on the energy, suddenly becomes momentum and time dependent $f(\omega) \rightarrow f(\omega, \mathbf{k}, \Delta t)$.
This is especially true at short delays $\Delta t \sim 0$, where the direct photoexcitation leads to a discrete excitation of unoccupied states, which is highly anisotropic in momentum space.
After thermalization processes set in the Fermi-Dirac function starts to populate the entire BZ becoming again more isotropic.
The state at these time scales can be described in an effective temperature model, in which an effective, time dependent electronic temperature is contained in the Fermi-Dirac equation.

\begin{figure}
	\centering
	\includegraphics[width=0.9\linewidth]{images/trarpes/example_bandstructure}
	\caption{}
	\label{fig:example_bandstructure}
\end{figure}


Additionally one has to consider the effect of a potentially time dependent matrix element.
At equilibrium the matrix element connects an initial and final state by the light-matter interaction Hamiltonian $H_\text{int} \propto \mathbf{A\cdot\mathbf{p}}$.
Here, $\mathbf{A}$ describes the electromagnetic vector potential and $\mathbf{p}$ the electron momentum operator.
Therefore, in ARPES both the measurement geometric, light polarization and sample geometry are relevant for the signal, which contains orbital information of the initial state of the emitted photoelectron.
Typically, any time evolution of the matrix element is ignored in trARPES measurements, but it has been shown that such an evolution can occur \cite{} boschini freericks and krish.
This shows that polarization dependent studies should be considered for trARPES experiments.

The third contribution to the signal at equilibrium was the spectral function $A(\mathbf{k}, \omega)$.
In trARPES measurements the spectral function $A(\mathbf{k}, \omega, \Delta t)$ encompasses transient changes of many body interactions, which are contained in the self-energy, or coupling to bosonic modes (e.g. phonons)
For example the coupling to phonons will result in a oscillatory modulation of the bare band dispersion, which can be seen in chapter \ref{ch_tate2}.
A detailed description with examples for the complexity of the trARPES signal is given in the review by Boschini et al. \ref{} boschini

This description of the trARPES signal is valid only in the quasi-equilibrium scenario.
At time delays corresponding to an overlap between pump and probe, and until initial scattering leads to thermalization, an out of equilibrium approach has to be chosen.
Modeling these effects is a big theoretical challenge.
This becomes especially evident when considering effects like Floquet-Bloch states, which do not exist without the presence of a light pulse.

In general trARPES is a powerful technique, capable of observing many transient effects such as band structure changes, renormalizations due to changes in many-body effects, couplings with phonons.
It can not only provide insight into the electronic properties at equilibrium but also observe pure light-induced phenomena.
The chapters \ref{ch:bi2212} and \ref{ch:tate2} will discuss two studies on \ce{Bi}2212 and \ce{TaTe2} that demonstrate the importance of trARPES for out of equilibrium studies.