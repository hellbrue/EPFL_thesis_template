\chapter{Time \& angle resolved photoemission spectroscopy}

The development and understanding of quantum materials are pivotal in advancing next-generation technologies.
In many of these materials, electron interactions and correlations significantly influence their overall properties, often leading to the emergence of exotic phases.
Gaining insight into how electron and their interactions shape these phases and give rise to emergent phenomena is essential for understanding and potentially manipulating these materials.

Angle-resolved photoemission spectroscopy (ARPES) is a powerful technique uniquely suited for probing these interactions.
It measures the single-particle spectral function, allowing for a detailed mapping of the electronic band structure across the full 3D Brillouin zone (BZ).
ARPES is based on the photoelectric effect, first observed by Hertz in 1887 \cite{hertz_ueber_1887} and later described by Einstein in 1905 \cite{einstein_uber_1905}, and fundamentally operates as a photon-in, electron-out method.
Over the years, ARPES has matured, as highlighted in numerous reviews, particularly in the context of complex materials \cite{damascelli_angle-resolved_2003,damascelli_angle-resolved_2003,lu_angle-resolved_2012,gedik_photoemission_2017,lv_angle-resolved_2019,zhang_angle-resolved_2022}.
Various sub-techniques have been developed to meet specific experimental needs, such as spin-ARPES, $\mu$- or $n$-ARPES (with \unit{\micro\meter} or \unit{\nano\meter} spatial resolution), and time-resolved ARPES (trARPES).

Focusing on time-resolved ARPES (trARPES), the technique has seen rapid development over recent decades, leading to a deepened understanding, as summarized in recent reviews \cite{smallwood_ultrafast_2016,huang_high-resolution_2022,boschini_time-resolved_2024}.
trARPES allows for the measurement of transient changes in the electronic band structure and the interactions between electrons and bosons (e.g., phonons).
This makes it a valuable tool for studying out-of-equilibrium dynamics, observing light-induced states, and unraveling the hierarchy of phase formation or the melting of equilibrium orders.
As a result, trARPES provides profound insights into the behavior of correlated matter and opens pathways for manipulating and controlling their properties.

In this chapter, I will provide a brief description of the technique and explain the origin of the measured signals.
The discussion will begin with an analysis of the equilibrium case, followed by an extension into the time domain, focusing on the quasi-equilibrium regime and briefly touching on out-of-equilibrium phenomena.

\section{General description}

ARPES is a photon-in, electron-out technique based on the photoelectric effect.
Figure \ref{fig:arpes_sketch} (a) illustrates a typical ARPES experiment.
When an incoming photon with energy $h\nu$ exceeds the material's work function $\phi$, an electron can be ejected with kinetic energy determined by the difference between its binding energy $E_\text{B}$, $\phi$, and $h\nu$.
This relationship is captured by the energy conservation equation
\begin{equation}
	E_\text{kin} = h\nu - \phi - E_\text{B}.
	\label{eq:e_cons}
\end{equation}
The photoelectrons emitted from the sample are collected by a spectrometer, such as a hemispherical analyzer, allowing the joint density of states (JDOS) to be observed.
Figure \ref{fig:arpes_sketch} (b) illustrates the relationship between the binding energy $E_\text{B}$ of electrons within the crystal and the kinetic energy $E_\text{kin}$ of the detected photoelectrons.

In addition to energy, the angle at which the photoelectrons are emitted and their kinetic energy can be used to determine the crystal momentum of the electrons inside the material, based on momentum conservation rules.
For the in-plane momentum component, this relation is given by
\begin{equation}
	\hbar \mathbf{k}\parallel = \sqrt{2m_eE\text{kin}} \sin\theta,
	\label{eq:mom}
\end{equation}
where $\theta$ is the emission angle.
However, the out-of-plane momentum component $\mathbf{k}\perp$ is not conserved due to the broken periodicity at the surface of the material.
Using the free-electron final state approximation, $\mathbf{k}\perp$ depends on the inner potential $V_0$, which is typically unknown prior to the measurement.
The expression for $\mathbf{k}\perp$ is
\begin{equation}
	\hbar \mathbf{k}\perp = \sqrt{2m_e\left(E_\text{kin}\cos^2\theta+V_0\right)}.
\end{equation}
If the free-electron approximation does not hold, the perpendicular momentum component can still be determined by performing photon energy-dependent scans and knowledge of the final state dispersion \cite{strocov_intrinsic_2003}.

%\begin{figure}
%	\centering
%	\begin{subfigure}[b]{0.49\textwidth}
%		\includegraphics[width=\linewidth]{images/trarpes/arpes_sketch}
%		\caption{}
%	\end{subfigure}
%	\hfill
%	\begin{subfigure}[b]{0.4\textwidth}
%		\includegraphics[width=\linewidth]{images/trarpes/dos}
%		\caption{}
%	\end{subfigure}
%	\caption{(a) Sketch of the trARPES setup. The red pulse excites the sample, while the purple pulse probes the excitation and generates photoelectrons according to Eqs. \ref{eq:e_cons} and \ref{eq:mom}. (b) Visualization of the energy relation within the material’s band structure (left), as a function of binding energy, to the detected kinetic energy $E_\text{kin}$ (right) according to Eq. \ref{eq:e_cons}. Adapted from Ref. \cite{hufner_photoelectron_1995}.}
%	\label{fig:arpes_sketch}
%\end{figure}
\begin{figure}
	\centering
	\includegraphics[width=\linewidth]{images/trarpes_pp/arpes_sketch}
	\caption{(a) Sketch of the trARPES setup. The red pulse excites the sample, while the purple pulse probes the excitation and generates photoelectrons according to Eqs. \ref{eq:e_cons} and \ref{eq:mom}. (b) Visualization of the energy relation within the material’s band structure (left), as a function of binding energy, to the detected kinetic energy $E_\text{kin}$ (right) according to Eq. \ref{eq:e_cons}. Adapted from Ref. \cite{hufner_photoelectron_1995}.}
	\label{fig:arpes_sketch}
\end{figure}

In equilibrium ARPES measurements, under the sudden approximation and for single-band systems, the measured intensity can be expressed using Fermi's golden rule.
This takes into account the matrix element $M(\mathbf{k}, \omega)$, the Fermi-Dirac distribution $f(\mathbf{k}, \omega)$, and the spectral function $A(\mathbf{k}, \omega)$.
The sudden approximation assumes that the photoemission process occurs rapidly, without further interaction between the photoelectron and the remaining system after photon absorption.
The intensity is then given by
\begin{equation}
	I(\mathbf{k}, \omega) = A(\mathbf{k}, \omega)\left|M(\mathbf{k}, \omega)\right|^2f(\omega),
	\label{eq:spectral}
\end{equation}
where the matrix element $M(\mathbf{k}, \omega)$ represents the probability of the transition between an initial and final state
It not only contains the probability for a direct transition between an initial and final state $\Phi_i$ and $\Phi_f$, but also information that is sensitive to the sample geometry, light polarization and ARPES setup due to the light-matter interaction Hamiltonian $H_\text{int}$.
The matrix element is defined as
\begin{equation}
	M = \braket{\Phi_f\left|H_\text{int}\right|\Phi_i} = \braket{\Phi_f\left| \mathbf{A} \cdot \mathbf{p} \right|\Phi_i},
\end{equation}
where $\mathbf{A}$ is the electromagnetic vector potential and $\mathbf{p}$ is the electron momentum operator.
Since the matrix element depends on the parity of the initial state and the vector potential, the ARPES signal is strongly influenced by the measurement geometry, the light's polarization, and the orbital character of the bands.
This can lead to a full suppression or enhancement of certain bands in specific experimental configurations \cite{gierz_illuminating_2011, cao_mapping_2013,zhu_layer-by-layer_2013,schuler_polarization-modulated_2022}.

The spectral function $A(\mathbf{k}, \omega) = -(1/\pi)\text{Im}G(\mathbf{k}, \omega)$ contains information about both the bare band dispersion $\epsilon_\mathbf{k}$ and the self-energy $\Sigma(\mathbf{k}, \omega)$, which accounts for many-body effects influencing the single-particle \cite{mahan_many-particle_2000}.
The Green's function captures the inherent nature of the photoemission, measuring the single-particle excitation of a many-body system.
Specifically, this function describes how an $N$ electron system responds to a single electron removal and highlights how correlations and interactions are imprinted on the measured photoemission spectrum.

The full spectral function is described by
\begin{equation} A(\mathbf{k}, \omega) = -\frac{1}{\pi} \frac{\Sigma''(\mathbf{k}, \omega)}{\left[ \omega - \epsilon_\mathbf{k} - \Sigma'(\mathbf{k}, \omega) \right]^2 + \left[ \Sigma''(\mathbf{k}, \omega) \right]^2},
\end{equation}
where $\Sigma'(\mathbf{k}, \omega)$ and $\Sigma''(\mathbf{k}, \omega)$ are the real and imaginary parts of the self-energy, respectively.
The real part $\Sigma'(\mathbf{k}, \omega)$ shifts the bare band dispersion, while the imaginary part $\Sigma''(\mathbf{k}, \omega)$ introduces an energy broadening, compared to a non-interacting case.
The lifetime of the excited state, resulting from the removal of a single electron can be calculated from the imaginary part of the self-energy $\tau=\hbar/\left[2\Sigma''(\mathbf{k}, \omega)\right]$.

To extract self-energy information from ARPES data, energy distribution curves (EDCs) and momentum distribution curves (MDCs) are commonly analyzed.
By examining the position and width of spectral features in these curves, insight into the self-energy can be gained \cite{norman_extraction_1999, freericks_what_2021,kurleto_about_2021}.
Intimate knowledge of the bare-band dispersion is also beneficial for interpreting ARPES data.

For a more detailed discussion on equilibrium ARPES, refer to reviews by Damascelli et al. and Sobota et al. \cite{damascelli_angle-resolved_2003,sobota_angle-resolved_2021}.
The next section will build on these fundamentals by extending equilibrium ARPES principles into the time domain.

\section{TR-ARPES: a stroboscopic time domain approach}

To extend ARPES into the time domain, a stroboscopic pump-probe approach is used.
A pump pulse initiates an excitation in the system, followed by a probe pulse that measures the excited state.
By introducing a delay between the pump and probe pulses and varying this delay, the transient evolution of the system can be tracked over time.
In this setup, the pump pulse serves as a temporal reference, with the dynamics observed relative to it.
Either the pump or probe pulse can be delayed using a translation stage to adjust the timing between them.

The earliest trARPES experiments were conducted using two-photon photoemission (2PPE), where both pulses had energies below the work function, and only their combined energy was sufficient to emit photoelectrons.
In the context of this thesis instead, extreme ultraviolet (EUV) probe pulses will be used to measure the entire Brillouin zone, while an infrared pump pulse creates the excitation.
More details on 2PPE-based trARPES can be found in various dedicated reviews \cite{damascelli_multiphoton_1996,bartoli_nonlinear_1997,hofer_time-resolved_1997,bovensiepen_elementary_2012,cui_transient_2014}.

Describing and interpreting the ARPES signal following a pump excitation is complex.
In a first approximation, the equilibrium intensity equation \ref{eq:arpes_signal} can be extended into the time domain as follows
\begin{equation}
	I(\mathbf{k}, \omega, \Delta t) = A(\mathbf{k}, \omega, \Delta t)\left|M(\mathbf{k}, \omega, \Delta t)\right|^2f(\mathbf{k}, \omega, \Delta t),
\end{equation}
where $\Delta t$ is the delay between the pump and probe pulses \cite{freericks_what_2021}.
This approximation assumes that the electrons thermalize into a quasi-equilibrium state with a well-defined electronic temperature, typically after tens to hundreds of femtoseconds, depending on the material and the properties of the light.
For instance, adiabatic band dispersion has been shown to persist even at time zero and under high pump fluences \cite{boschini_time-resolved_2024,neufeld_time-_2022}.

Disentangling the various contributions to the signal is more challenging than in the equilibrium case.
The electron distribution, which at equilibrium is represented by a momentum-independent Fermi-Dirac function, becomes both momentum- and time-dependent, with $f(\omega) \rightarrow f(\omega, \mathbf{k}, \Delta t)$.
This is particularly significant at short delays ($\Delta t \sim 0$), where direct photoexcitation leads to anisotropic excitation of unoccupied states in momentum space.
As the system undergoes thermalization to bring the system back to a Fermi-Dirac distribution, the electron distribution becomes more isotropic, populating the full Brillouin zone.
This state can be described using an effective temperature model, where a time-dependent electronic temperature is incorporated into the Fermi-Dirac distribution.
Figure \ref{fig:example_bandstructure} sketches an example of such electron population dynamics and the corresponding time-dependent ARPES signal.

\begin{figure}
	\centering
	\includegraphics[width=\linewidth]{images/trarpes_pp/example_bandstructure}
	\caption{Illustration of the trARPES process with exemplary transient band strucutre evolution. Adapted from Ref. \cite{boschini_time-resolved_2024}. (Left) shows the sequence from equilibrium, to excitation when the pump pulse arrived at the sample and finally photoemission from the excited state when the probe pulse arrives. (Right) Shows a sequence of band maps for an exemplary Dirac cone (top) and the corresponding difference maps to equilibrium (bottom) for four different time steps. Before $t_0$ the equilibrium band structure can be seen. At $t_0$ a direct transition, as indicated by the red dashed arrow occurs, corresponding to the energy of the pump pulse, at timescales corresponding to the electron-electron scattering time $\tau_{ee}$ the initial distribution begins to thermalize, this population decays at longer timescales.}
	\label{fig:example_bandstructure}
\end{figure}

Additionally, the matrix element in trARPES may become time-dependent, adding another layer of complexity.
As in equilibrium ARPES, the matrix element is sensitive to the geometry of the experiment, the light polarization, and the sample's properties, and it encodes information about the orbital character of the initial state.
Recent experiments have shown that the matrix element can be modified during the pump-probe process \cite{boschini_role_2020,freericks_constant_2016}, indicating that polarization-dependent studies should be considered for trARPES experiments.

The third key contribution to the ARPES signal is the spectral function $A(\mathbf{k}, \omega)$.
In trARPES, the spectral function $A(\mathbf{k}, \omega, \Delta t)$ reflects transient changes in many-body interactions, such as those contained in the self-energy, or interactions with bosonic modes like phonons.
For instance, coupling to phonons can cause an oscillatory modulation of the bare band dispersion, as explored in Chapter \ref{ch:tate2}.
A comprehensive review of the complexity of trARPES signals, with examples, is provided by Boschini et al. \cite{boschini_time-resolved_2024}.

The description of the trARPES signal discussed here applies to quasi-equilibrium scenarios.
However, when pump and probe pulses overlap and until scattering processes lead to thermalization, an out-of-equilibrium approach must be used \cite{schuler_theory_2021, freericks_what_2021, neufeld_time-_2022}.
Modeling these effects presents significant theoretical challenges, particularly when considering phenomena such as Floquet-Bloch states, which only exist in the presence of a light pulse.

Overall, trARPES is a powerful tool for observing a variety of transient effects, including band structure changes, renormalizations due to many-body interactions, and coupling with phonons.
It provides not only insights into equilibrium electronic properties but also a view on purely light-induced phenomena such as Floquet states, though these will not be covered in detail in this thesis.
Chapters \ref{ch:bi2212} and \ref{ch:tate2} will focus on two studies of \ce{Bi}2212 and \ce{TaTe2}, highlighting the importance of trARPES in out-of-equilibrium investigations.